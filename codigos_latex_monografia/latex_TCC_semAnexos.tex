%% abtex2-modelo-trabalho-academico.tex, v-1.9.2 laurocesar
%% Copyright 2012-2014 by abnTeX2 group at http://abntex2.googlecode.com/ 
%%
%% This work may be distributed and/or modified under the
%% conditions of the LaTeX Project Public License, either version 1.3
%% of this license or (at your option) any later version.
%% The latest version of this license is in
%%   http://www.latex-project.org/lppl.txt
%% and version 1.3 or later is part of all distributions of LaTeX
%% version 2005/12/01 or later.
%%
%% This work has the LPPL maintenance status `maintained'.
%% 
%% The Current Maintainer of this work is the abnTeX2 team, led
%% by Lauro César Araujo. Further information are available on 
%% http://abntex2.googlecode.com/
%%
%% This work consists of the files abntex2-modelo-trabalho-academico.tex,
%% abntex2-modelo-include-comandos and abntex2-modelo-references.bib
%%

% ------------------------------------------------------------------------
% ------------------------------------------------------------------------
% abnTeX2: Modelo de Trabalho Academico (tese de doutorado, dissertacao de
% mestrado e trabalhos monograficos em geral) em conformidade com 
% ABNT NBR 14724:2011: Informacao e documentacao - Trabalhos academicos -
% Apresentacao
% ------------------------------------------------------------------------
% ------------------------------------------------------------------------

\documentclass[
	% -- opções da classe memoir --
	12pt,				% tamanho da fonte
	openright,			% capítulos começam em pág ímpar (insere página vazia caso preciso)
	twoside,			% para impressão em verso e anverso. Oposto a oneside
	a4paper,			% tamanho do papel. 
	% -- opções da classe abntex2 --
	%chapter=TITLE,		% títulos de capítulos convertidos em letras maiúsculas
	%section=TITLE,		% títulos de seções convertidos em letras maiúsculas
	%subsection=TITLE,	% títulos de subseções convertidos em letras maiúsculas
	%subsubsection=TITLE,% títulos de subsubseções convertidos em letras maiúsculas
	% -- opções do pacote babel --
	english,			% idioma adicional para hifenização
	brazil				% o último idioma é o principal do documento
	]{abntex2}

% ---
% Pacotes básicos 
% ---
\usepackage{lmodern}			% Usa a fonte Latin Modern			
\usepackage[T1]{fontenc}		% Selecao de codigos de fonte.
\usepackage[utf8]{inputenc}		% Codificacao do documento (conversão automática dos acentos)
\usepackage{lastpage}			% Usado pela Ficha catalográfica
\usepackage{indentfirst}		% Indenta o primeiro parágrafo de cada seção.
\usepackage{color}				% Controle das cores
\usepackage{graphicx}			% Inclusão de gráficos
\usepackage{microtype} 			% para melhorias de justificação
% ---
		
% ---
% Pacotes adicionais, usados apenas no âmbito do Modelo Canônico do abnteX2
% ---
\usepackage{lipsum}				% para geração de dummy text
% ---

\usepackage{mathtools}
\usepackage{amssymb}
\newcommand{\ffrac}[2]{\ensuremath{\frac{\displaystyle #1}{\displaystyle #2}}}

% ---
% Pacotes de citações
% ---
\usepackage[brazilian,hyperpageref]{backref}	 % Paginas com as citações na bibl
\usepackage[alf]{abntex2cite}	% Citações padrão ABNT
\usepackage{wrapfig}
\usepackage{lscape}
\usepackage{rotating}
\usepackage{epstopdf}
% --- 
% CONFIGURAÇÕES DE PACOTES
% --- 

% ---
% Configurações do pacote backref
% Usado sem a opção hyperpageref de backref
\renewcommand{\backrefpagesname}{Citado na(s) página(s):~}
% Texto padrão antes do número das páginas
\renewcommand{\backref}{}
% Define os textos da citação
\renewcommand*{\backrefalt}[4]{
	\ifcase #1 %
		Nenhuma citação no texto.%
	\or
		Citado na página #2.%
	\else
		Citado #1 vezes, nas páginas #2.%
	\fi}%
% ---

% ---
% Informações de dados para CAPA e FOLHA DE ROSTO
% ---
\titulo{Diga-me quem segues e lhe direi quem és:\\
        Estimação de ideologia feminista no Twitter usando ponto ideal bayesiano\\
        Anexos}
\autor{Camila Lainetti de Morais}
\local{São Paulo}
\data{2021}
\orientador[Orientadora:]{Márcia D'Elia Branco}
%\coorientador{Equipe \abnTeX}
\instituicao{%
  Universidade de São Paulo
  \par
  Instituto de Matemática e Estatística}
\tipotrabalho{Trabalho de Conclusão de Curso}
% O preambulo deve conter o tipo do trabalho, o objetivo, 
% o nome da instituição e a área de concentração 
%\preambulo{Modelo canônico de trabalho monográfico acadêmico em conformidade com
%as normas ABNT apresentado à comunidade de usuários \LaTeX.}
% ---


% ---
% Configurações de aparência do PDF final

% alterando o aspecto da cor azul
%\definecolor{blue}{RGB}{41,5,195}

% informações do PDF
\makeatletter
\hypersetup{
     	%pagebackref=true,
		pdftitle={\@title}, 
		pdfauthor={\@author},
    	pdfsubject={\imprimirpreambulo},
	    pdfcreator={LaTeX with abnTeX2},
		pdfkeywords={abnt}{latex}{abntex}{abntex2}{trabalho acadêmico}, 
		colorlinks=true,       		% false: boxed links; true: colored links
    	linkcolor=blue,          	% color of internal links
    	citecolor=blue,        		% color of links to bibliography
    	filecolor=magenta,      		% color of file links
		urlcolor=blue,
		bookmarksdepth=4
}
\makeatother
% --- 

% --- 
% Espaçamentos entre linhas e parágrafos 
% --- 

% O tamanho do parágrafo é dado por:
\setlength{\parindent}{1.3cm}

% Controle do espaçamento entre um parágrafo e outro:
\setlength{\parskip}{0.2cm}  % tente também \onelineskip

% ---
% compila o indice
% ---
\makeindex
% ---


% ----
% Início do documento
% ----
\begin{document}

% Retira espaço extra obsoleto entre as frases.
\frenchspacing 

% ----------------------------------------------------------
% ELEMENTOS PRÉ-TEXTUAIS
% ----------------------------------------------------------
% \pretextual

% ---
% Capa
% ---
% \imprimircapa
% ---

% ---
% Folha de rosto
% (o * indica que haverá a ficha bibliográfica)
% ---
\imprimirfolhaderosto
% ---

% ---
% Inserir a ficha bibliografica
% ---

% Isto é um exemplo de Ficha Catalográfica, ou ``Dados internacionais de
% catalogação-na-publicação''. Você pode utilizar este modelo como referência. 
% Porém, provavelmente a biblioteca da sua universidade lhe fornecerá um PDF
% com a ficha catalográfica definitiva após a defesa do trabalho. Quando estiver
% com o documento, salve-o como PDF no diretório do seu projeto e substitua todo
% o conteúdo de implementação deste arquivo pelo comando abaixo:
%
% \begin{fichacatalografica}
%     \includepdf{fig_ficha_catalografica.pdf}
% \end{fichacatalografica}
%\begin{fichacatalografica}
%	\vspace*{\fill}					% Posição vertical
%	\hrule							% Linha horizontal
%	\begin{center}					% Minipage Centralizado
%	\begin{minipage}[c]{12.5cm}		% Largura
%	
%	\imprimirautor
%	
%	\hspace{0.5cm} \imprimirtitulo  / \imprimirautor. --
%	\imprimirlocal, \imprimirdata-
%	
%	\hspace{0.5cm} \pageref{LastPage} p. : il. (algumas color.) ; 30 cm.\\
%	
%	\hspace{0.5cm} \imprimirorientadorRotulo~\imprimirorientador\\
%	
%	\hspace{0.5cm}
%	\parbox[t]{\textwidth}{\imprimirtipotrabalho~--~\imprimirinstituicao,
%	\imprimirdata.}\\
%	
%	\hspace{0.5cm}
%		1. Palavra-chave1.
%		2. Palavra-chave2.
%		I. Orientador.
%		II. Universidade xxx.
%		III. Faculdade de xxx.
%		IV. Título\\ 			
%	
%	\hspace{8.75cm} CDU 02:141:005.7\\
%	
%	\end{minipage}
%	\end{center}
%	\hrule
%\end{fichacatalografica}
% ---
% ---

% ---
% Inserir folha de aprovação
% ---

% Isto é um exemplo de Folha de aprovação, elemento obrigatório da NBR
% 14724/2011 (seção 4.2.1.3). Você pode utilizar este modelo até a aprovação
% do trabalho. Após isso, substitua todo o conteúdo deste arquivo por uma
% imagem da página assinada pela banca com o comando abaixo:
%
% \includepdf{folhadeaprovacao_final.pdf}
%
%\begin{folhadeaprovacao}
%
%  \begin{center}
%    {\ABNTEXchapterfont\large\imprimirautor}
%
%    \vspace*{\fill}\vspace*{\fill}
%    \begin{center}
%      \ABNTEXchapterfont\bfseries\Large\imprimirtitulo
%    \end{center}
%    \vspace*{\fill}
%    
%    \hspace{.45\textwidth}
%    \begin{minipage}{.5\textwidth}
%        \imprimirpreambulo
%    \end{minipage}%
%    \vspace*{\fill}
%   \end{center}
%        
%   Trabalho aprovado. \imprimirlocal, 24 de novembro de 2012:
%
%   \assinatura{\textbf{\imprimirorientador} \\ Orientador} 
%   \assinatura{\textbf{Professor} \\ Convidado 1}
%   \assinatura{\textbf{Professor} \\ Convidado 2}
   %\assinatura{\textbf{Professor} \\ Convidado 3}
   %\assinatura{\textbf{Professor} \\ Convidado 4}
%      
%   \begin{center}
%    \vspace*{0.5cm}
%    {\large\imprimirlocal}
%    \par
%    {\large\imprimirdata}
%    \vspace*{1cm}
%  \end{center}
%  
%\end{folhadeaprovacao}
% ---

% ---
% Dedicatória
% % ---
% \begin{dedicatoria}
%   \vspace*{\fill}
%   \centering
%   \noindent
%   \textit{Este trabalho é dedicado a todas as mulheres que lutaram por seus direitos antes de mim e que permitiram o meu próprio caminho até a universidade.} \vspace*{\fill}
% \end{dedicatoria}
% ---

% ---
% Agradecimentos
% ---
% \begin{agradecimentos}
% Agradeço às feministas que lutaram para incluir mulheres em todos os espaços, inclusive os acadêmicos, e também às colegas feministas atuais, que continuam na luta pela manutenção e expansão dos nossos direitos.

% Agradeço à minha orientadora Márcia D'Elia Branco por todos os direcionamentos, indicações de leitura, críticas e comentários perspicazes.

% Agradeço ao meu companheiro, André, por todo o suporte e carinho dados ao longo da confecção deste trabalho, e também pela paciência e pela revisão do texto.

% Agradeço aos meus pais, Denise e José Carlos, e à minha família, vó Izaura, vó Nair, tia Cristina e sogra Odúlia, pelo carinho, pela compreensão nas indisponibilidades, e pela torcida de que tudo viria a dar certo no final.

% Agradeço aos meus amigos, em especial Caio Vinícius e Araras, pelo apoio e pela flexibilidade de sempre poder esperar mais uma semana até a próxima sessão de RPG.

% Agradeço aos meus colegas de classe, sempre muito generosos, que tanto me ajudaram a compreender os conteúdos das disciplinas do curso.

% Agradeço ao corpo docente e aos funcionários do Instituto de Matemática e Estatística da Universidade de São Paulo, sobretudo aos professores do Bacharelado em Matemática Aplicada e Computacional. A dedicação, a disponibilidade e o apoio que encontrei no instituto superaram todas as minhas expectativas.

% Agradeço também aos administradores da Rede Linux, principalmente ao Pedro T. R. Pinheiro, por toda a ajuda com o uso da rede de forma remota, inclusive em horários ``inusitados''.

% Agradeço, por fim, à Alexandra Asanovna Elbakyan e a todos aqueles que trabalham pela popularização da ciência, ainda tão inacessível para tantas pessoas.
% \end{agradecimentos}
% ---

% ---
% Epígrafe
% ---
% \begin{epigrafe}
%     \vspace*{\fill}
% 	\begin{flushright}
% 		\textit{``Metade vítimas, metade cúmplices, como todo mundo''
% 		(J.-P. Sartre)}
% 	\end{flushright}
% \end{epigrafe}
% ---

% ---
% RESUMOS
% ---

% \setlength{\absparsep}{18pt} % ajusta o espaçamento dos parágrafos do resumo
% \begin{resumo}
%  Em um contexto no qual a última campanha presidencial vitoriosa teve como um eixo central o combate à “ideologia de gênero”, faz-se necessário compreender o posicionamento de figuras públicas e cidadãos em relação aos temas de gênero e feminismo. Valendo-se do modelo de estimação bayesiana de ponto ideal proposto por \citeonline{barbera2015}, este trabalho analisa um conjunto de influenciadores e cidadãos brasileiros ativos no Twitter, metrificando seus posicionamentos ideológicos no tocante ao feminismo, com o objetivo de compreender mais sobre os grupos feministas e antifeministas, a relação entre eles, e suas possíveis divisões internas. A observação das estimações de ponto ideal dos influenciadores aponta que existem dois \textit{clusters}, um feminista e outro antifeminista, bastante separados e com poucos seguidores em comum. Notamos ainda uma subdivisão interna entre feministas negras e não negras, e uma subdivisão interna entre antifeministas que atuam principalmente no âmbito religioso evangélico e aqueles cuja principal atuação concentra-se no âmbito político. Em relação aos cidadãos, observamos que aqueles que seguem muitos influenciadores incluídos no estudo têm, em geral, posicionamentos estimados mais moderados que aqueles das figuras públicas. Construímos ainda, para a validação das conclusões obtidas, uma análise de rede de relacionamentos dos influenciadores.

% \textbf{Palavras-chaves}: gênero; feminismo; antifeminismo; posicionamento ideológico; análise bayesiana.
% \end{resumo}

% resumo em inglês
%\begin{resumo}[Abstract]
% \begin{otherlanguage*}{english}
%   This is the english abstract.
%
%   \vspace{\onelineskip}
% 
%   \noindent 
%   \textbf{Key-words}: latex. abntex. text editoration.
% \end{otherlanguage*}
%\end{resumo}

% ---
% inserir lista de ilustrações
% ---
%\pdfbookmark[0]{\listfigurename}{lof}
%\listoffigures*
%\cleardoublepage
% ---

% ---
% inserir lista de tabelas
% ---
%\pdfbookmark[0]{\listtablename}{lot}
%\listoftables*
%\cleardoublepage
% ---

% ---
% inserir lista de abreviaturas e siglas
% ---
%\begin{siglas}
%  \item[ABNT] Associação Brasileira de Normas Técnicas
%  \item[abnTeX] ABsurdas Normas para TeX
%\end{siglas}
% ---

% ---
% inserir lista de símbolos
% ---
%\begin{simbolos}
%  \item[$ \Gamma $] Letra grega Gama
%  \item[$ \Lambda $] Lambda
%  \item[$ \zeta $] Letra grega minúscula zeta
%  \item[$ \in $] Pertence
%\end{simbolos}
% ---

% ---
% inserir o sumario
% ---
\pdfbookmark[0]{\contentsname}{toc}
\tableofcontents*
%\cleardoublepage
% ---


% ----------------------------------------------------------
% ELEMENTOS TEXTUAIS
% ----------------------------------------------------------
\textual

% ----------------------------------------------------------
% Introdução (exemplo de capítulo sem numeração, mas presente no Sumário)
% ----------------------------------------------------------
%\chapter*[Introdução]{Introdução}
%\addcontentsline{toc}{chapter}{Introdução}
% ----------------------------------------------------------

%Este documento e seu código-fonte são exemplos de referência de uso da classe
%\textsf{abntex2} e do pacote \textsf{abntex2cite}. O documento 
%exemplifica a elaboração de trabalho acadêmico (tese, dissertação e outros do
%gênero) produzido conforme a ABNT NBR 14724:2011 \emph{Informação e documentação
%- Trabalhos acadêmicos - Apresentação}.
%
%A expressão ``Modelo Canônico'' é utilizada para indicar que \abnTeX\ não é
%modelo específico de nenhuma universidade ou instituição, mas que implementa tão
%somente os requisitos das normas da ABNT. Uma lista completa das normas
%observadas pelo \abnTeX\ é apresentada em \citeonline{abntex2classe}.
%
%Sinta-se convidado a participar do projeto \abnTeX! Acesse o site do projeto em
%\url{http://abntex2.googlecode.com/}. Também fique livre para conhecer,
%estudar, alterar e redistribuir o trabalho do \abnTeX, desde que os arquivos
%modificados tenham seus nomes alterados e que os créditos sejam dados aos
%autores originais, nos termos da ``The \LaTeX\ Project Public
%License''\footnote{\url{http://www.latex-project.org/lppl.txt}}.
%
%Este documento deve ser utilizado como complemento dos manuais do \abnTeX\ 
%\cite{abntex2classe,abntex2cite,abntex2cite-alf} e da classe \textsf{memoir}
%\cite{memoir}. 


% ----------------------------------------------------------
% PARTE
% ----------------------------------------------------------
%\part{Preparação da pesquisa}
% ----------------------------------------------------------

% ---
% Capitulo com exemplos de comandos inseridos de arquivo externo 
% ---
%\include{abntex2-modelo-include-comandos}
% ---

% ----------------------------------------------------------
% PARTE
% ----------------------------------------------------------
%\part{Referenciais teóricos}
% ----------------------------------------------------------
%\lipsum[31-33]

% ----------------------------------------------------------
% ELEMENTOS PÓS-TEXTUAIS
% ----------------------------------------------------------
 \postextual
% ----------------------------------------------------------

% ----------------------------------------------------------
% Referências bibliográficas
% ----------------------------------------------------------

% ----------------------------------------------------------
% Glossário
% ----------------------------------------------------------
%
% Consulte o manual da classe abntex2 para orientações sobre o glossário.
%
%\glossary

% ----------------------------------------------------------
% Apêndices
% ----------------------------------------------------------

% ---
% Inicia os apêndices
% ---
%\begin{apendicesenv}

% Imprime uma página indicando o início dos apêndices
%\partapendices

% ----------------------------------------------------------
%\chapter{Quisque libero justo}
% ----------------------------------------------------------

%\lipsum[50]

% ----------------------------------------------------------
%\chapter{Nullam elementum urna vel imperdiet sodales elit ipsum pharetra ligula
%ac pretium ante justo a nulla curabitur tristique arcu eu metus}
% ----------------------------------------------------------
%\lipsum[55-57]

%\end{apendicesenv}
% ---


% ----------------------------------------------------------
% Anexos
% ----------------------------------------------------------

% ---
% Inicia os anexos
% ---
 \begin{anexosenv}

% Imprime uma página indicando o início dos anexos
 \partanexos

% ---
 \chapter{Breve justificativa dos influenciadores escolhidos}\label{justificativa}

 \section*{Influenciadoras feministas do Twitter}
 Feministas que apoiam todos os eixos que as caracterizam\footnote{Ver caracterização feita no capítulo 1 do texto principal.}, ou uma quantidade expressiva dos mesmos, e também mídias, coletivos e partidos que os apoiam.
 
\begin{enumerate}

\subsection*{Pessoas públicas}

 \item Amanda Audi - @amandafaudi
 
 Jornalista formada pela Universidade Federal do Paraná (UFPR) e repórter do Intercept Brasil, Amanda Audi\footnote{\url{https://twitter.com/amandafaudi}. Último acesso em 16/02/2021.} discute principalmente política e direitos humanos, mas também aborda questões feministas, com as quais se identifica. Amanda já colaborou com mídias como Folha de S.Paulo, O Globo, Gazeta do Povo, TV Globo e Poder360, e atualmente é diretora-executiva do Congresso em Foco.

 \item Ana Flor - @Tdetravesti
 
 Militante do Levante Popular da Juventude de Recife\footnote{Coletivo político de juventude de esquerda. Ver \url{https://levante.org.br/quem-somos/}. Último acesso em 16/02/2021.} e graduanda de pedagogia, Ana Flor\footnote{\url{https://twitter.com/Tdetravesti}. Último acesso em 16/02/2021.} escreve para a mídia Blogueiras Feministas (Ver \ref{blogfeministas}) e para o Medium.
 
 \item Andreza - @andrezadelgado
 
 Produtora cultural, \textit{youtuber}, \textit{podcaster}, colunista da UOL, co-criadora da PerifaCon, da PerifaGamer e da Copa das Favelas, Andreza Delgado\footnote{\url{https://twitter.com/andrezadelgado}. Último acesso em 16/02/2021.} utiliza diversos meios de comunicação e mídias sociais para democratizar o acesso à cultura nerd para jovens de periferia, além de discutir raça, gênero e política.

 \item Anielle Franco - @aniellefranco\label{aniellefranco}
 
 Educadora, jornalista, escritora, feminista preta, mestranda e diretora do Instituto Marielle Franco, Anielle Franco\footnote{\url{https://twitter.com/aniellefranco}. Último acesso em 16/02/2021.} é irmã de Marielle Franco (Ver \ref{luyfranco}).

\newpage

 \item Antonia Pellegrino - @apellegrino\label{pellegrino}
 
 Diretora, produtora, escritora e roteirista premiada pela Academia Brasileira de Letras, Antonia Pellegrino\footnote{\url{https://twitter.com/apellegrino}. Último acesso em 16/02/2021.} é uma ativista feminista que usa sua sua atuação no cinema e na televisão para abordar temas ligados à gênero, raça e política. Foi roteirista do documentário \textit{Democracia em Vertigem}, dirigido por Petra Costa (ver \ref{PetraCosta}), indicado ao Oscar de Melhor Documentário de longa-metragem, que aborda o tema democracia no Brasil. Era amiga pessoal de Marielle Franco (ver \ref{luyfranco}) e é esposa de Marcelo Freixo, deputado de quem Marielle foi assessora parlamentar.

 \item Áurea Carolina - @aureacarolinax
 
 Cientista social especialista em Gênero e Igualdade pela Universidade Autônoma de Barcelona e Mestra em Ciência Política pela Universidade Federal de Minas Gerais (UFMG), Áurea Carolina\footnote{\url{https://twitter.com/aureacarolinax}. Último acesso em 18/02/2021.} é também política, ativista feminista negra e pela juventude.

 \item Bia Ferreira - @iglesbiteriana
 
 Cantora, compositora, multi-instrumentista, Bia Ferreira\footnote{\url{https://twitter.com/iglesbiteriana}. Último acesso em 16/02/2021.} usa sua música (e suas redes sociais) para abordar raça, feminismo e política.
 
 \item Benedita da Silva - @dasilvabenedita
 
 Política, professora e ativista feminista negra, Benedita da Silva\footnote{\url{https://twitter.com/dasilvabenedita}. Último acesso em 16/02/2021.} é reconhecida defensora da mulher, da igualdade racial, da trabalhadora doméstica, das minorias, dos direitos humanos e das comunidades faveladas.

 \item Bruna de Lara - @delarabru
 
 Bruna de Lara\footnote{\url{https://twitter.com/delarabru}. Último acesso em 16/02/2021.} é jornalista formada pela Escola de Comunicação da Universidade Federal do Rio de Janeiro (UFRJ), coautora do livro ``$\#$MeuAmigoSecreto: feminismo além das redes'' e trabalha atualmente no Intercept Brasil. Trabalhou na revista piauí e foi uma das vencedoras do 9º Prêmio Jovem Jornalista Fernando Pacheco Jordão, oferecido pelo Instituto Vladimir Herzog. Em 2018, recebeu uma menção honrosa no 35º Prêmio Direitos Humanos de Jornalismo pela reportagem ``As mães que tiveram seus filhos assassinados pelo Estado decidiram fazer o trabalho da polícia: investigar'', publicada no Intercept.
 
 \newpage

 \item Bueno. - @winniebueno
 
 Criadora da winnieteca, uma iniciativa que promove a conexão de pessoas negras com a literatura, através da ponte entre pessoas dispostas a doarem livros com quem precisa de leituras específicas e não possui meios de acessá-las. Também é doutoranda em Sociologia pela Universidade Federal do Rio Grande do Sul (UFRGS), onde pesquisa o pensamento de Patricia Hill Collins, acadêmica e feminista negra renomada, especializada em raça, classe e gênero. Sobre seu tema de estudo, Winnie Bueno\footnote{\url{https://twitter.com/winniebueno}. Último acesso em 16/02/2021.} publicou em 2020 o livro Imagens de Controle - Um conceito do Pensamento de Patricia Hill Collins, pela editora Zouk.

 \item Carina Vitral - @carinavitral
 
 Ativista feminista e pelos direitos dos estudantes e ex-presidenta da União Nacional dos Estudantes (UNE), Carina Vitral\footnote{\url{https://twitter.com/carinavitral}. Último acesso em 16/02/2021.} é política e atualmente exerce cargo de primeira suplente de deputado estadual por São Paulo.

 \item Cida Falabella - @FalabellaCida
 
 Cida Falabella\footnote{\url{https://twitter.com/FalabellaCida}. Último acesso em 16/02/2021.} é atriz, diretora, professora e política. Foi co-vereadora de Belo Horizonte em um mandato coletivo chamado \textit{Gabinetona}. Cida é ainda ativista feminista e cultural.

 \item Clara Averbuck - @claraaverbuck
 
 Escritora feminista, colunista do jornal Zero Hora e da Revista Fórum, e editora do Lugar de Mulher. Clara Averbuck\footnote{\url{https://twitter.com/claraaverbuck}. Último acesso em 18/02/2021.} utiliza sua escrita para quebrar tabus de gênero, em especial com a personagem Camila, declarada um dos alter egos da autora.

 \item Cris Bartis - @CrisBartis\label{crisbartis}
 
 Comunicadora e \textit{podcaster}, Cris Bartis\footnote{\url{https://twitter.com/CrisBartis}. Último acesso em 18/02/2021.} é co-criadora do \textit{podcast} Mamilos, programa que aborda gênero, política e assuntos relacionados.

 \item cynara menezes - @cynaramenezes
 
 Jornalista, blogueira e escritora, ex-repórter de grandes mídias como Folha de São Paulo e Carta Capital, Cynara Menezes\footnote{\url{https://twitter.com/cynaramenezes}. Último acesso em 16/02/2021.} exerce hoje o jornalismo independente. Em 2013 fundou o \textit{blog} Socialista Morena, onde aborda temas como feminismo, arte, cultura, política, direitos humanos, mídia e maconha.

 \item Dani Monteiro - @danimontpsol\label{danimonteiro}
 
 Dani Monteiro\footnote{\url{https://twitter.com/danimontpsol}. Último acesso em 18/02/2021.} é deputada estadual do Rio de Janeiro, eleita em 2018, ativista negra, e estudante de Ciências Sociais da UERJ. Foi assessora da vereadora Marielle Franco (Ver \ref{luyfranco}).

 \item Debora Baldin - @deborabaldin$\_$
 
 Formada em Relações Internacionais, \textit{youtuber}, \textit{podcaster} e comunicadora, Debora Baldin\footnote{\url{https://twitter.com/deborabaldin_}. Último acesso em 16/02/2021.} usa seu canal do Youtube, \textit{podcasts} e suas redes sociais para falar sobre feminismo, política e cultura.

 \item Debora Diniz - @Debora$\_$D$\_$Diniz
 
 Antropóloga, acadêmica, pesquisadora, professora, escritora e diretora de documentários, coordenadora e fundadora do Anis -- Instituto de Bioética, Direitos Humanos e gênero, Debora Diniz\footnote{\url{https://twitter.com/Debora_D_Diniz}. Último acesso em 16/02/2021.} é uma ativista feminista autoexilada depois de inúmeras ameaças de morte por defender a descriminalização do aborto no país.

 \item Deputada Isa Penna $\#$VacinaJá - @IsaPennaPsol
 
 Isa Penna\footnote{\url{https://twitter.com/IsaPennaPsol}. Último acesso em 16/02/2021.} é advogada, ativista feminista, dos direitos LGBT e ecossocialista. Atualmente, é deputada estadual de São Paulo.

 \item Eliane Brum - @brumelianebrum
 
 Jornalista, escritora e documentarista, Eliane Brum\footnote{\url{https://twitter.com/brumelianebrum}. Último acesso em 16/02/2021.} publicou sete livros no Brasil, dirigiu e co-dirigiu quatro documentários, e escreveu diversos artigos, reportagens e textos para revistas e jornais, como El País. Eliane aborda temas políticos, históricos, de gênero, de meio ambiente, entre outros.

 \item Elika Takimoto - @elikatakimoto
 
 Elika Takimoto\footnote{\url{https://twitter.com/elikatakimoto}. Último acesso em 16/02/2021.} é vencedora do Prêmio Saraiva Literatura, na categoria crônicas. É doutora em Filosofia pela UERJ, mestre em História das Ciências e das Técnicas e Epistemologia pela UFRJ. Também é graduada em Física pela UFRJ, Professora e Coordenadora de Física do CEFET/RJ, e escreveu mais de dez livros. É ativista feminista filiada ao Partido dos Trabalhadores (PT).
\newpage
 \item Elza Soares - @ElzaSoares
 
 Cantora e compositora renomada, Elza Soares\footnote{\url{https://twitter.com/elzasoares}. Último acesso em 28/02/2021.} foi eleita pela Revista Rolling Stone Brasil como uma das cem maiores vozes da música brasileira, e também pela rádio BBC de Londres como a cantora brasileira do milênio. Elza se considera feminista, e é considerada por seu público ícone tanto do movimento feminista quanto do movimento negro.

 \item Erica Malunguinho - @malunguinho
 
 Deputada, artista e educadora, Erica Malunguinho\footnote{\url{https://twitter.com/malunguinho}. Último acesso em 16/02/2021.} é nascida e criada em Pernambuco, em uma família de militantes de movimentos populares. Enquanto pesquisadora cultural, suas pesquisas são do âmbito das artes, culturas e políticas a partir dos fundamentos de raça e de gênero. Como artista e cidadã, construiu solitária e coletivamente diversas ações performáticas que afrontavam as estruturas de poder. É mestra em estética e história da arte. Fundou o quilombo urbano Aparelha Luzia, em 2016.

 \item ERIKA HILTON - @ErikakHilton
 
 Erika Hilton\footnote{\url{https://twitter.com/ErikakHilton}. Último acesso em 16/02/2021.} é vereadora eleita da cidade de São Paulo. Negra e transvestigênere, foi a mulher mais bem votada em 2020 em todo o país. É também ativista dos Direitos Humanos, na luta por equidade para a população negra, no combate à discriminação contra a comunidade LGBTQIA+\footnote{Lésbicas, Gays, Transexuais e Travestis, Queers, Intersexos, Assexuais e outros grupos e variações de sexualidade e gênero.}, pela valorização das iniciativas culturais jovens e periféricas.

 \item Erika Kokay - @erikakokay
 
 Psicóloga, política e sindicalista, Erika Kokay\footnote{\url{https://twitter.com/erikakokay}. Último acesso em 16/02/2021.} presidiu o Sindicato dos bancários de Brasília e a Central Única dos Trabalhadores (CUT) do Distrito Federal. Foi eleita diversas vezes deputada federal, cargo que exerce hoje. É co-coordenadora da Frente Parlamentar Feminista Antirracista com Participação Popular.

 \item Fernanda Melchionna - @fernandapsol
 
 Fernanda Melchionna\footnote{\url{https://twitter.com/fernandapsol}. Último acesso em 16/02/2021.} é bibliotecária, bancária e política, eleita diversas vezes vereadora de Porto Alegre, atuando hoje como deputada federal pelo Rio Grande do Sul. Ativista feminista, chegou a palestrar na Universidade da Califórnia em Berkeley sobre o tema.
 
 \item Flávia Oliveira - @flaviaol
 
 Jornalista, colunista do jornal O Globo e da CBN, comentarista do GloboNews e \textit{podcaster}, Flávia Oliveira\footnote{\url{https://twitter.com/flaviaol}. Último acesso em 16/02/2021.} é ativista feminista negra, e mãe da também ativista Isabela Reis (ver \ref{belareis}).

 \item Gabi Coelho - @gabicoelho
 
 Repórter e coordenadora da equipe de colunistas voluntários do jornal comunitário Voz das Comunidades, no Rio de Janeiro, articuladora e comunicadora social no Observatório do Funk e \textit{podcaster}. Cursa Jornalismo na Pontifícia Universidade Católica de Minas Gerais (PUC-Minas) e organiza ações sociais em favelas de Belo Horizonte. Gabi Coelho\footnote{\url{https://twitter.com/gabicoelho}. Último acesso em 16/02/2021.} é também idealizadora e produtora de um documentário em andamento, sobre mulheres fotógrafas de favela no Complexo do Alemão e Morro do Santa Marta.

 \item gabrielle - @gabriolaz
 
 Graduanda em Direito pela Pontifícia Universidade Católica de de São Paulo (PUC-SP), militante abolicionista penal, membra da Frente Estadual pelo Desencarceramento de São Paulo e Amparar, além de \textit{podcaster}, Gabrielle Nascimento\footnote{\url{https://twitter.com/gabriolaz}. Último acesso em 16/02/2021.} discute principalmente sobre abolicionismo penal e questões de raça e gênero.

 \item Isabela Reis - @bela$\_$reis\label{belareis}
 
 Jornalista formada pela UFRJ, ativista negra antirracista, \textit{podcaster} e escritora, Isabela Reis\footnote{\url{https://twitter.com/bela_reis}. Último acesso em 16/02/2021.} aborda principalmente os temas mulher, negritude e política nas suas principais redes sociais --- Instagram e YouTube.
 
 \item Jandira Feghali - @jandira$\_$feghali
 
 Médica e sindicalista, Jandira Feghali\footnote{\url{https://twitter.com/jandira_feghali}. Último acesso em 16/02/2021.} foi eleita diversas vezes deputada federal pelo Rio de Janeiro, e tem ampla participação em projetos em defesa da mulher. Foi relatora do projeto de \textit{lei Maria da Penha}, é autora do texto da lei que concede licença maternidade à mãe adotante, e também é autora da lei que garante cirurgia reparadora de mama em casos de câncer.

 \item Joanna Maranhão - @Jujuca1987
 
 Ex-atleta olímpica de natação, Joanna Maranhão\footnote{\url{https://twitter.com/Jujuca1987}. Último acesso em 16/02/2021.} é também ativista feminista, e usa sua voz para promover direitos da mulher e denunciar o assédio às mulheres no esporte.

 \item Jout Jout - @joutfuckinjout
 
 Julia Tolezano, conhecida como Jout Jout\footnote{\url{https://twitter.com/joutfuckinjout}. Último acesso em 16/02/2021.}, é \textit{youtuber}, escritora e jornalista. Atualmente trabalha na televisão, no canal pago GNT. Costuma abordar temas como feminismo e mulheres em seu canal.

 \item Ju Wallauer - @jwallauer
 
 Juliana Wallauer\footnote{\url{https://twitter.com/jwallauer}. Último acesso em 18/02/2021.} é profissional do marketing e \textit{podcaster}, co-criadora do \textit{podcast} Mamilos juntamente com Cris Bartis (ver \ref{crisbartis}).

 \item Jurema Werneck - @juremawerneck
 
 Feminista, doutora em Comunicação e Cultura pela Universidade Federal do Rio de Janeiro (UFRJ), médica, escritora, ativista do movimento de mulheres negras brasileiro e dos direitos humanos, e diretora executiva da Anistia Internacional -- Brasil. Em 2006, Jurema Werneck\footnote{\url{https://twitter.com/juremawerneck}. Último acesso em 16/02/2021.} publicou o livro ``Saúde das Mulheres Negras: Nossos Passos Vêm de Longe''.

 \item Laerte Coutinho - @LaerteCoutinho1
 
 Quadrinista, cartunista e chargista, Laerte\footnote{\url{https://twitter.com/LaerteCoutinho1}. Último acesso em 18/02/2021.} é uma ativista feminista trans reconhecida pela Ordem do Mérito Cultural brasileira.
 
 \item Laura Carvalho - @lauraabcarvalho
 
 Economista, acadêmica, escritora e \textit{podcaster}, Laura Carvalho\footnote{\url{https://twitter.com/lauraabcarvalho}. Último acesso em 16/02/2021.} já usou suas redes sociais para se manifestar favoravelmente à descriminalização do aborto e outras pautas feministas.

 \item Leci Brandão - @lecibrandao
 
 Cantora, compositora e uma das mais importantes intérpretes de samba da música popular brasileira. Leci Brandão\footnote{\url{https://twitter.com/lecibrandao}. Último acesso em 16/02/2021.} é também política e ativista feminista negra e lésbica, tendo sido eleita diversas vezes deputada estadual por São Paulo.

 \item Letícia Parks - @letparks
 
 Letícia Parks\footnote{\url{https://twitter.com/letparks}. Último acesso em 16/02/2021.} é professora formada em Letras pela Universidade de São Paulo (USP), militante do Movimento Revolucionário de Trabalhadores (MRT), uma das organizadoras do livro ``A Revolução e o Negro'', colunista do Esquerda Diário, feminista socialista e fundadora do grupo de negras e negros anticapitalistas Quilombo Vermelho.

 \item Linn da Quebrada - @linndaquebrada
 
 Cantora, atriz, roteirista, compositora e apresentadora com quase 1 milhão de ouvintes em 91 países em 2020, Linn da Quebrada\footnote{\url{https://twitter.com/linndaquebrada}. Último acesso em 16/02/2021.} usa sua arte para abordar temas como corpo, vida, vivência feminina e transexual, gênero e sexualidade.

 \item $☭$Lisa Débora Perpétua da Faísca - @passarosErosas
 
 Influenciadora feminista do twitter, posicionada como ativista feminista, Lisa\footnote{\url{https://twitter.com/passarosErosas}. Último acesso em 16/02/2021.} (provavelmente um \textit{codinome}) tem mais de trinta mil seguidores e é seguida por influenciadoras como Débora Diniz, Mônica Francisco, Lola Aronovich, Marcia Tiburi e Cynara Menezes, e também pelas mídias Não me Kahlo, Blogueiras Feministas e Portal Catarinas.

 \item Lola Aronovich - @lolaescreva
 
 Professora universitária e blogueira, Lola Aronovich\footnote{\url{https://twitter.com/lolaescreva}. Último acesso em 16/02/2021.} utiliza seu \textit{blog} \textit{Escreva Lola escreva} há mais de uma década para difundir textos sobre política, mídia, direitos humanos e outros assuntos, quase sempre abordando questões feministas. O \textit{blog} já foi reconhecido como o maior do assunto no Brasil, e até hoje é uma das referências do movimento feminista brasileiro.

 \item louie ponto - @louieponto
 
 \textit{Youtuber} e ativista lésbica, Louie Ponto\footnote{\url{https://twitter.com/louieponto}. Último acesso em 16/02/2021.} é conhecida pelos seus vídeos que abordam gênero e sexualidade.

 \item Luciana Boiteux - @luboiteux
 
 Luciana Boiteux\footnote{\url{https://twitter.com/luboiteux}. Último acesso em 18/02/2021.} é advogada, acadêmica, professora, ativista feminista, militante dos direitos humanos e política.

 \item Luciana Genro - @lucianagenro
 
 Política e advogada, Luciana Genro\footnote{\url{https://twitter.com/lucianagenro}. Último acesso em 16/02/2021.} é uma das fundadoras do PSOL, juntamente com Heloísa Helena e outros dissidentes do PT, e atual dirigente do partido. Eleita em 2018, Luciana Genro é deputada estadual pelo Rio Grande do Sul, e tem propostas de combate à desigualdade e de defesa de pautas de minorias, como as feministas.

 \item Luiza Erundina - @luizaerundina
 
 Política e assistente social, Luiza Erundina\footnote{\url{https://twitter.com/luizaerundina}. Último acesso em 16/02/2021.} foi deputada federal por São Paulo diversas vezes, chegando a ser eleita a primeira prefeita mulher de São Paulo, em 1988, contando com Paulo Freire no cargo de Secretário da Educação. Foi de Erundina a iniciativa da Lei nº 12.612 que elegeu Freire Patrono da Educação brasileira. Atualmente, Erundina é deputada federal por São Paulo, integrando a Frente Parlamentar Feminista Antirracista com Participação Popular desde sua fundação em 2019.

 \item Manuela - @ManuelaDavila
 
 Jornalista, escritora, ativista feminista e política progressista, filiada ao Partido Comunista do Brasil (PCdoB), Manuela d'Ávila\footnote{\url{https://twitter.com/ManuelaDavila}. Último acesso em 16/02/2021.} foi deputada federal pelo Rio Grande do Sul entre 2007 a 2015, deputada estadual de 2015 a 2019 e candidata a vice-presidente da República em 2018, chegando a ir para o segundo turno contra Jair Bolsonaro e Hamilton Mourão.

 \item Manuela Barem - @manubarem
 
 Jornalista, escritora e especialista em comunicação digital, Manuela Barem\footnote{\url{https://twitter.com/manubarem}. Último acesso em 18/02/2021.} é também fundadora e ex-editora chefe do BuzzFeed-Brasil, mídia onde abordava temas como causas LGBT, feminismo, racismo e saúde mental.

 \item marcia tiburi - @marciatiburi
 
 Marcia Tiburi\footnote{\url{https://twitter.com/marciatiburi}. Último acesso em 16/02/2021.} é filósofa, acadêmica, professora, artista visual, escritora e ativista feminista. Seus principais temas de estudo e de escrita são política, arte, ética, estética, filosofia do conhecimento e feminismo.

 \item Margarida Salomão - @JFMargarida
 
 Margarida Salomão\footnote{\url{https://twitter.com/JFMargarida}. Último acesso em 16/02/2021.} é acadêmica, professora, sindicalista, política e atual prefeita de Juiz de Fora, tendo sido também reitora da Universidade Federal de Juiz de Fora (UFJF) e dirigente da CUT de Juiz de Fora. Suas principais pautas no congresso foram relativas à educação, mas tem ampla defesa também das pautas ligadas ao direito da mulher.

 \item Marília Arraes - @MariliaArraes
 
 Advogada e política, Marília Arraes\footnote{\url{https://twitter.com/MariliaArraes}. Último acesso em 16/02/2021.} é atualmente deputada federal por Pernambuco. Defensora de pautas feministas, Marília é atualmente titular das comissões da procuradoria da mulher e da secretaria da mulher, e da subcomissão de seguridade da mulher na Câmara dos Deputados.

 \item Marília Moscou - @MariliaMoscou
 
 Socióloga, mestra e doutora em educação pela Universidade de Campinas (Unicamp), Marília Moscou\footnote{\url{https://twitter.com/MariliaMoscou}. Último acesso em 16/02/2021.} atualmente reside em Berlim, onde é \textit{Research Fellow} da Fundação Alexander von Humboldt em parceria com o festival de cinema Berlin Feminist Film Week, dedicando-se à pesquisa sobre não-monogamia, gênero e violência doméstica. Compartilha os resultados desse projeto em suas redes sociais e no \textit{podcast} \textit{Libre}. Também é poeta e editora. Militante do Partido Comunista Brasileiro (PCB), escreve e faz comentário político \textit{online} desde 2010.

 \item MC Carol - @mc$\_$caroloficial
 
 MC Carol\footnote{\url{https://twitter.com/mc_caroloficial}. Último acesso em 16/02/2021.} é cantora, compositora e ativista feminista. Suas músicas abordam temáticas sociais, feminismo, humor e também sexualidade de forma explícita.

 \item Milly Lacombe - @millylacombe
 
 Escritora, roteirista, jornalista, Milly Lacombe\footnote{\url{https://twitter.com/millylacombe}. Último acesso em 16/02/2021.} é uma ativista feminista e lésbica. Escreve a ``Coluna do Meio'' na revista mensal TPM e é autora e co-autora de livros como ``Tudo É Só Isso''.

 \item Monica Benicio - @monica$\_$benicio\label{monicabenicio}
 
 Vereadora eleita do Rio de Janeiro, arquiteta e urbanista com foco em Direito à cidade, militante de direitos humanos, lésbica, feminista e periférica, Monica Benicio\footnote{\url{https://twitter.com/monica_benicio}. Último acesso em 16/02/2021.} é viúva de Marielle Franco (Ver \ref{luyfranco}).

 \item Mônica Bergamo - @monicabergamo
 
 Jornalista e colunista da Folha de São Paulo e da rádio BandNews, Mônica Bergamo\footnote{\url{https://twitter.com/monicabergamo}. Último acesso em 16/02/2021.} aborda diversos temas, de política a entretenimento. Costuma colocar em pauta temas feministas e denunciar casos de agressões contra mulheres e seus direitos.

 \item Mônica Francisco - @MonicaFPsol
 
 Feminista negra, periférica, militante de Direitos Humanos e ex-assessora de Marielle Franco, Mônica Francisco\footnote{\url{https://twitter.com/MonicaFPsol}. Último acesso em 16/02/2021.} também é pesquisadora, pastora e vereadora eleita do Rio de Janeiro.

 \item Monica Seixas - @MonicaSeixas
 
 Codeputada estadual por São Paulo, eleita pela Mandata Ativista, Monica Seixas\footnote{\url{https://twitter.com/MonicaSeixas}. Último acesso em 18/02/2021.} é ativista feminista negra e ecossocialista, jornalista e redatora.

 \item Natália Bonavides - @natbonavides
 
 Advogada popular, ativista feminista e política, Natália Bonavides\footnote{\url{https://twitter.com/natbonavides}. Último acesso em 16/02/2021.} atualmente é deputada federal pelo PT.

 \item Nátaly Neri - @natalyneri
 
 Cientista social e \textit{youtuber} com quase oitocentos mil inscritos\footnote{Verificado em 16/02/2021.}, Nátaly Neri\footnote{\url{https://twitter.com/natalyneri}. Último acesso em 16/02/2021.} usa as mídias sociais --- Youtube, Twitter e Instagram --- para falar sobre raça, gênero, sociedade, sustentabilidade e outros temas, como veganismo.

 \item negaluy - @luyarafranco\label{luyfranco}
 
 Luyara Franco\footnote{\url{https://twitter.com/luyarafranco}. Último acesso em 16/02/2021.} é estudante de Educação Física pela Universidade do Estado do Rio de Janeiro (UERJ), feminista e integrante da diretoria do Instituto Marielle Franco, que leva o nome da sua mãe. Marielle foi feminista, socióloga e política, assassinada no segundo ano do seu mandato de vereadora do Rio de Janeiro, em 2018.

 \item Paola Carosella - @PaolaCarosella
 
 Paola Carosella\footnote{\url{https://twitter.com/PaolaCarosella}. Último acesso em 16/02/2021.} é empresária, chef de cozinha, e foi jurada do \textit{talent show} culinário MasterChef. Em suas redes sociais e em entrevistas, Paola costuma defender o feminismo e se posiciona alinhada à defesa dos direitos da mulher.

 \item Perpétua Almeida - @perpetua$\_$acre
 
 Perpétua Almeida\footnote{\url{https://twitter.com/perpetua_acre}. Último acesso em 16/02/2021.} é professora, bancária e política, e atual deputada federal pelo Estado do Acre. Feminista, se posicionou favoravelmente à descriminalização do aborto, e defende direitos das mulheres nas redes sociais e no congresso.

 \item Petra Costa - @petracostal\label{PetraCosta}
 
 Petra Costa\footnote{\url{https://twitter.com/petracostal}. Último acesso em 16/02/2021.} é cineasta brasileira e membra da Academia de Artes e Ciências Cinematográficas desde 2018. Dirigiu Democracia em Vertigem, indicado ao Oscar de melhor documentário, com roteiro de Antonia Pellegrino (ver \ref{pellegrino}). Petra navega entre temas íntimos e questões sociais e políticas em seus filmes, e também aborda questões de gênero, como em Elena.

 \item Preta Ijimú - @Nailahnv
 
 Nailah Neves\footnote{\url{https://twitter.com/Nailahnv}. Último acesso em 16/02/2021.} é pesquisadora e professora voluntária do Maré -- Núcleo de Estudos em Cultura Jurídica e Atlântico Negro da Faculdade de Direito da Universidade de Brasília (UnB), embaixadora da Juventude do Escritório das Nações Unidas sobre Drogas e Crime (UNODC), consultora em Políticas de promoção de igualdade racial e de gênero, mestre em direitos humanos e cidadania na linha de pesquisa de Direitos Humanos, Democracia, Construção de Identidades/Diversidades e Movimentos Sociais pela UnB, além de \textit{podcaster}.

 \item Renata Souza - @renatasouzario
 
 Deputada estadual do Rio de Janeiro, jornalista, acadêmica, feminista negra, militante dos direitos humanos, primeira mulher negra presidente da Comissão de Defesa dos Direitos Humanos e Cidadania da ALERJ e ex chefe de gabinete do mandato de Marielle Franco (Ver \ref{luyfranco}), Renata Souza\footnote{\url{https://twitter.com/renatasouzario}. Último acesso em 18/02/2021.} chegou a se candidatar à prefeitura do Rio de Janeiro em 2020, alcançando mais de oitenta e cinco mil votos válidos.

 \item Rita Lisauskas - @RitaLisauskas
 
 Jornalista, blogueira, repórter e apresentadora, Rita Lisauskas\footnote{\url{https://twitter.com/RitaLisauskas}. Último acesso em 16/02/2021.} trabalhou em diversas mídias como TV Cultura, SBT, Record e Rede Bandeirantes. Atualmente é comentarista do quadro Liberdade de Opinião na CNN Brasil, com atuação também no Estadão e na Rádio Eldorado. Nas redes sociais, se posiciona abertamente a favor dos direitos da mulher e do feminismo.

 \item Rosana Pinheiro-Machado - @$\_$pinheira
 
 Antropóloga, acadêmica e professora de Desenvolvimento Internacional da Universidade de Bath no Reino Unido, Rosana Pinheiro-Machado\footnote{\url{https://twitter.com/_pinheira}. Último acesso em 16/02/2021.} também é colunista do jornal The Intercept. Ela escreve principalmente sobre feminismo e política.

 \item Sabrina Fernandes - @safbf\label{sabfer}
 
 Sabrina Fernandes\footnote{\url{https://twitter.com/safbf}. Último acesso em 18/02/2021.} é socióloga, acadêmica, escritora, professora, \textit{podcaster} e \textit{youtuber}, além de militante feminista, ecossocialista e marxista. Em seu canal Tese Onze, ela aborda temas de gênero e políticos em geral.

 \item Sâmia Bomfim - @samiabomfim
 
 Ex-servidora pública da Universidade de São Paulo (USP), política e ativista feminista, Sâmia Bomfim\footnote{\url{https://twitter.com/samiabomfim}. Último acesso em 16/02/2021.} foi vereadora de São Paulo e é, atualmente, deputada federal por São Paulo.

 \item Sher Machado - @transcurecer
 
 Presidente do Centro Acadêmico de Física do Instituto Federal do Rio de Janeiro (IFRJ), representante discente do Colegiado de Campus, conselheira do Conselho Superior do IFRJ, estudante do curso de licenciatura em Física, secretária geral do CapacitransRJ, embaixadora do Ceres Trans e guia do Wakanda Streamers, Sher Machado\footnote{\url{https://twitter.com/transcurecer}. Último acesso em 16/02/2021.} acumula contribuições à comunidade negra e trans. Além das atividades contínuas citadas, Sher ainda criou a Copa Rebecca Heineman, campeonato do jogo League of Legends\footnote{Jogo \textit{online} competitivo multijogador. Ver \url{https://br.leagueoflegends.com/pt-br/how-to-play/}. Último acesso em 16/02/2021.} dedicado às pessoas trans.

 \item Sonia Guajajara - @GuajajaraSonia
 
 Sonia Guajajara\footnote{\url{https://twitter.com/GuajajaraSonia}. Último acesso em 18/02/2021.} é ativista indígena e feminista, coordenadora da Articulação dos Povos Indígenas do Brasil (APIB) e também é formada em Letras e em Enfermagem, especialista em Educação especial pela Universidade Estadual do Maranhão (UFMA).

 \item Sueli Carneiro - @SueliCarneiro
 
 Intelectual, acadêmica, filósofa, ativista negra antirracista, Sueli Carneiro\footnote{\url{https://twitter.com/SueliCarneiro}. Último acesso em 16/02/2021.} é fundadora e atual diretora do Geledés —- Instituto da Mulher Negra, primeira organização negra e feminista independente de São Paulo e considerada uma das principais autoras do feminismo negro do país.

 \item Suzane Jardim - @jardim$\_$suzane
 
 Bacharela e licenciada em História pela USP, mestranda em Ciências Humanas e Sociais pela Universidade Federal do ABC (UFABC), pesquisadora do sistema carcerário brasileiro e dos processos raciais de criminalização, também atua na iniciativa privada e em ONGs como facilitadora de processos em educação antirracista e como consultora e pesquisadora para cinema, quadrinhos, animações, documentários e outros projetos culturais. Suzane Jardim\footnote{\url{https://twitter.com/jardim_suzane}. Último acesso em 16/02/2021.} ainda é ativista voltada para a disseminação de conhecimento e avanço de pautas relacionadas aos Direitos Humanos com foco em questão racial, de gênero, criminologia e segurança pública.

 \item Talíria Petrone - @taliriapetrone
 
 Talíria Petrone\footnote{\url{https://twitter.com/taliriapetrone}. Último acesso em 16/02/2021.} é professora, mestre em serviço social, política e ativista feminista negra. Eleita vereadora em 2016, atualmente é deputada estadual do Rio de Janeiro. Escreveu o prefácio para o livro ``Feminismo para os $99\%$: um manifesto''.

 \item Tati Nefertari - @TatiNefertari
 
 Coordenadora da Biblioteca Comunitária Assata Shakur, membra da org. Ujima Povo Preto, ativista negra antirracista e graduanda em Pedagogia pela Universidade de São Paulo (USP), Tati Nefertari\footnote{\url{https://twitter.com/TatiNefertari}. Último acesso em 16/02/2021.} utiliza também as redes sociais para discutir raça e feminismo negro.

 \item Tatiana Roque - @tatiroque
 
 Acadêmica e professora do Instituto de Matemática da Universidade Federal do Rio de Janeiro (UFRJ), Tatiana Roque\footnote{\url{https://twitter.com/tatiroque}. Último acesso em 16/02/2021.} é também coordenadora do Fórum de Ciência e Cultura da UFRJ e vice-presidente da Rede Brasileira de Renda Básica. Foi candidata a deputada federal pelo PSOL, em 2018. Tatiana tem também um canal no YouTube, onde aborda política, ciência e filosofia.

 \subsection*{Mídias, Coletivos e Partidos}

 \item Blogueiras Feministas - @blogfeministas\label{blogfeministas}
 
 \textit{Blog} político, feminista e coletivo, a mídia \textit{Blogueiras Feministas}\footnote{\url{https://twitter.com/blogfeministas}. Último acesso em 16/02/2021.} já contou com a publicação de mais de 70 colaboradores diferentes. Fundado durante o primeiro turno das eleições de 2010, o \textit{blog} permanece ativo, sendo espaço de discussão e crítica feminista, em geral sob a perspectiva da \textit{interseccionalidade}\footnote{Termo utilizado para caracterizar diferentes experiências de opressão, interligadas e influenciadas por diferentes questões: raça, gênero, classe, capacidades físicas e mentais, etnia, entre outros. O próprio \textit{blog} trata desse assunto em um de seus textos, que pode ser acessado em \url{https://blogueirasfeministas.com/2014/07/24/feminismo-intersecional-que-diabos-e-isso-e-porque-voce-deveria-se-preocupar/}. Último acesso em 20/02/2021.}.
 
 \item Blogueiras Negras - @BlogNegras
 
 \textit{Blog} político e feminista negro, a mídia \textit{Blogueiras Negras}\footnote{\url{https://twitter.com/BlogNegras}. Último acesso em 16/02/2021.} é escrita por mulheres negras e afrodescendentes. Criado no dia 8 de março de 2012, Dia Internacional da Mulher, o \textit{blog} conta hoje com uma comunidade de aproximadamente duzentas autoras feministas negras que se organizam de forma \textit{online} e \textit{offline}, produzindo conhecimento a partir de suas vivências e experiências.

 \item Geledés Instituto da Mulher Negra - @geledes
 
 Organização da sociedade civil que se posiciona em defesa de mulheres e negros, o Instituto Geledés\footnote{\url{https://twitter.com/geledes}. Último acesso em 16/02/2021.} posiciona-se também contra todas as demais formas de discriminação, como homofobia, lesbofobia, preconceitos regionais, de credo, opinião e de classe social. O portal da organização é um espaço de expressão pública das ações realizadas pela organização e de seus compromissos com a defesa da cidadania e dos direitos humanos, e a denúncia permanente dos entraves que persistem para a concretização da justiça social e da igualdade de direitos e de oportunidades em nossa sociedade. O portal possui principalmente postagens sobre questões de gênero e raciais, sobre discriminação e preconceitos e sobre a África e sua diáspora.

 \item Instituto Marielle Franco - @inst$\_$marielle
 
 O Instituto Marielle Franco\footnote{\url{https://twitter.com/inst_marielle}. Último acesso em 16/02/2021.} é uma organização sem fins lucrativos, criada pela família de Marielle, com a missão de conectar mulheres negras, LGBTQIA+ e periféricas e inspirá-las a seguir movendo as estruturas da sociedade por um mundo mais justo e igualitário. Promoveu ações como a Plataforma Antirracista nas Eleições, que reunia ações e ferramentas antirracistas, e o Prêmio Marielle Franco de Ensaios Feministas, que premiou escritoras feministas cis e trans pelo seu trabalho.
 
 \item Não Me Kahlo - @NAOKAHLO
 
 Coletivo nascido nas redes sociais, em um grupo de discussão no Facebook que chegou a reunir cerca de 3.000 mulheres, Não me Kahlo\footnote{\url{https://twitter.com/NAOKAHLO}. Último acesso em 16/02/2021.} rapidamente se expandiu para uma página de difusão de ideias no próprio Facebook e em outras redes, como Twitter e Instagram, e para um \textit{blog} colaborativo, que aborda temas como questões de gênero, machismo, racismo, lgbtfobia. Em 2015, foram iniciadoras da campanha $\#$MeuAmigoSecreto, que viralizou nas redes denunciando assédios e violências sofridas por mulheres vindas de homens próximos, como parentes ou professores. Em 2016, lançaram o livro ``$\#$MeuAmigoSecreto: feminismo além das redes'', abordando a campanha.

 \item ONU Mulheres Brasil - @ONUMulheresBR
 
 A ONU Mulheres, criada em 2010, tem o objetivo de unir, fortalecer e ampliar os esforços mundiais em defesa dos direitos humanos das mulheres. A organização atua como secretariado da Comissão da ONU sobre a Situação das Mulheres (CSW), que se reúne, no mês de março, em Nova Iorque, há mais de 60 anos. A comissão é uma das principais instâncias de negociação e de monitoramento de compromissos internacionais sobre direitos humanos das mulheres. Os encontros anuais contam com autoridades dos mecanismos das mulheres, sociedade civil e especialistas. O escritório do Brasil da ONU Mulheres\footnote{\url{https://twitter.com/ONUMulheresBR}. Último acesso em 16/02/2021.} opera em Brasília.

 \item Revista AzMina - @revistaazmina
 
 Revista feminista independente, a Revista AzMina\footnote{\url{https://twitter.com/revistaazmina}. Último acesso em 16/02/2021.} foi criada em 2015 através de financiamento coletivo. Atualmente organizada dentro do Instituto AzMina, uma organização sem fins lucrativos, a revista digital expandiu-se para novos horizontes de luta pela igualdade de gênero, com um aplicativo de enfrentamento à violência doméstica, uma plataforma de monitoramento legislativo dos direitos das mulheres, além de palestras e consultorias.

 \item Tese Onze - @teseonze
 O Tese Onze\footnote{\url{https://twitter.com/teseonze}. Último acesso em 16/02/2021.} é um canal do youtube com divulgação em outras redes sociais, cujo foco é apresentar contrapontos ao senso comum e trazer análises sobre sociologia e política com viés marxista, abordando, também, questões feministas. O canal é produzido e apresentado por Sabrina Fernandes (ver \ref{sabfer}), doutora em sociologia, feminista marxista, ecossocialista e vegana.
 
 \end{enumerate}
 
 \subsection*{Comentários}
 Outras pessoas públicas, coletivos e mídias foram configuradas para representar influenciadoras feministas nessa análise, mas acabaram não sendo incluídas por motivos alheios à vontade da pesquisadora, como saída voluntária do Twitter ou filtro do algoritmo usado na estimação do modelo. Por volta de cinquenta contas de influenciadoras feministas não foram analisadas devido a esses motivos. Entre as pessoas públicas, podemos citar a arquiteta feminista negra Stephanie Ribeiro, que excluiu ou suspendeu sua conta no Twitter durante o levantamento de dados, e a deputada federal Joenia Wapichana\footnote{\url{https://twitter.com/JoeniaWapichana}. Último acesso em 16/02/2021.}, ativista indígena, que não atingiu os requisitos do código de ser seguida por duzentos entre os dez mil cidadãos que mais seguem contas diversas dos influenciadores selecionados para o estudo (feministas e antifeministas). Pelo mesmo motivo da exclusão da conta de Joenia Wapichana, acabaram não analisados coletivos e mídias como QG Feminista\footnote{\url{https://twitter.com/QGfeminista}. Último acesso em 16/02/2021.}, Feminismo com Classe\footnote{\url{https://twitter.com/ComFeminismo}. Último acesso em 16/02/2021.} e revista Capitolina\footnote{\url{https://twitter.com/capitolinafala}. Último acesso em 16/02/2021.}.
 
 É importante considerarmos que não há uma lista ``oficial'' de feministas e que esse grupo não está sendo considerado como o único ou o melhor possível. A escolha foi feita, de partida, de acordo com influenciadoras que a pesquisadora já conhecia, e acabou se expandindo através da ajuda de colegas e intensa pesquisa sobre figuras públicas que tinham as características reunidas na seção 1.2.4 do capítulo 1. A pesquisadora tentou incluir no grupo de pesquisa ativistas feministas das mais diversas intersecções, bandeiras, vertentes e características possíveis: negras, trans, lésbicas, entre outras. Infelizmente, porém, provavelmente nem todas as intersecções foram cobertas. Acreditamos, ainda assim, que essa amostra é uma representação possível, ainda que incompleta, dos diversos movimentos feministas, e é compatível com os objetivos de estudo do trabalho.
 
 \newpage
 
 \section*{Influenciadores Antifeministas do Twitter}
 Figuras públicas antifeministas, que têm alinhamento com os eixos que os caracterizam\footnote{Ver caracterização feita no capítulo 1 do texto principal.}, ou uma quantidade expressiva deles, além de mídias, coletivos e partidos que os apoiam.
 
  
\begin{enumerate}

 \subsection*{Pessoas Públicas}
 \item Abraham Weintraub - @AbrahamWeint\label{abweint}
 
 Economista, professor e acadêmico da Universidade Federal de São Paulo (UNIFESP), Abraham Weintraub\footnote{\url{https://twitter.com/AbrahamWeint}. Último acesso em 20/02/2021.} esteve envolvido em várias controvérsias envolvendo racismo, incluindo ataques contra indígenas\footnote{Ver \url{https://cimi.org.br/2020/05/nota-repudio-postura-preconceituosa-ministro-abraham-weintraub/}. Último acesso em 20/02/2021.} e chineses\footnote{Ver \url{https://www.correiobraziliense.com.br/app/noticia/politica/2020/04/04/interna_politica,842431/weintraub-usa-cebolinha-da-turma-da-monica-para-atacar-a-china.shtml}. Último acesso em 20/02/2021.}, além de recorrentemente atacar às feministas em suas redes sociais.
 
 \begin{figure}[!htbp]
    \centering
    \includegraphics[scale=0.48]{abweint_1.png}
    \caption{Postagem no Twitter de Abraham Weintraub ironizando um suposto nome ``feminista'' dado a seu animal de estimação. Disponível em \url{https://twitter.com/AbrahamWeint/status/1178414831541608451}. Último acesso em 20/02/2021.}
 \end{figure}

 \newpage

 \item Ailton Benedito - @AiltonBenedito
 
 Procurador da República, Ailton Benedito\footnote{\url{https://twitter.com/AiltonBenedito}. Último acesso em 20/02/2021.} já usou suas redes sociais diversas vezes para atacar o feminismo e a suposta ``ideologia de gênero''. Envolveu-se também em diversas polêmicas, como a tentativa de derrubar a proibição do Conselho Federal de Psicologia (CFP) à participação de psicólogos em terapias de conversão e readequação de identidade de gênero, que afirmava ser parte resultante de um ``lobby transgênero em instituições''\footnote{Ver \url{https://www.bbc.com/portuguese/brasil-44745964}. Último acesso em 20/02/2021.}.
 
 \begin{figure}[!htbp]
    \centering
    \includegraphics[scale=0.7]{ailtonbenedito_2.png}
    \caption{Postagem no Twitter de Ailton atacando o feminismo. Disponível em \url{https://twitter.com/AiltonBenedito/status/1074273162404265986}. Último acesso em 20/02/2021.}
 \end{figure}

%  \begin{figure}[!htbp]
%     \centering
%     \includegraphics[scale=0.65]{ailtonbenedito_1.png}
%     \caption{Postagem no Twitter de Ailton Benedito criticando a suposta ``ideologia de gênero''. Disponível em \url{https://twitter.com/ailtonbenedito/status/1132340171029012481}. Último acesso em 20/02/2021.}
%  \end{figure}

 \item Alê Silva nem lugar na fila eu quero - @alesilva$\_$38
 
 Advogada, perita contábil e atual deputada federal por Minas Gerais, Alê Silva\footnote{\url{https://twitter.com/alesilva_38}. Último acesso em 20/02/2021.} é filiada ao Partido Social Liberal (PSL) (ver \ref{psl}) e declaradamente apoiadora do atual Presidente, Jair Bolsonaro. Envolvida em controvérsias como a publicação com viés machista de ataque à jornalista Vera Magalhães\footnote{Para informações mais completas sobre o ataque de Bolsonaro ao qual Alê Silva faz referência, ver \url{https://istoe.com.br/jair-bolsonaro-e-processado-por-ofensas-de-conotacao-sexual-a-reporter-patricia-campos-mello/}. Último acesso em 20/02/2021.} e a divulgação de informação falsa sobre projeto de lei da bancada feminista\footnote{Sobre o projeto de lei 1552/2020, ver \url{https://politica.estadao.com.br/blogs/estadao-verifica/pls-aprovados-na-camara-falam-em-protecao-a-vitimas-de-violencia-domestica-e-nao-alteram-legislacao-sobre-aborto/}. Último acesso em 20/02/2021.}, a deputada defende uma visão conservadora contrária às pautas feministas.

 \begin{figure}[!htbp]
    \centering
    \includegraphics[scale=0.9]{alesilva_1.png}
    \caption{Postagem no Twitter de Vera Magalhães denunciando publicação de Alê Silva. Deputada fazia, com sua ironização de duplo sentido, uma referência a um ataque semelhante que Bolsonaro disferiu contra a jornalista Patrícia Campos Mello, do jornal Folha de S.Paulo, em ocasião de matéria de Patrícia sobre disparo ilegal de mensagens em massa, via Whatsapp, em campanha presidencial de Bolsonaro. Disponível em \url{https://twitter.com/alesilva_38/status/1232489089946963968}. Último acesso em 20/02/2021.}
    \label{fig:alesilvafuro}
 \end{figure}

 \begin{figure}[!htbp]
    \centering
    \includegraphics[scale=0.7]{alesilva_2.png}
    \caption{Postagem no Twitter de Alê Silva. Deputada faz denúncia falsa de que o Projeto de lei 1552/2020, proposto por Sâmia Bomfim, seria usado para construção de supostos ``abortódromos''. O projeto de lei não faz qualquer menção ao direito ao aborto, e propõe direito de acolhimento temporário a vítimas de violência doméstica. Disponível em \url{https://twitter.com/alesilva_38/status/1268520976603648004}. Último acesso em 20/02/2021.}
    \label{fig:alesilvapl}
 \end{figure}

 \newpage
 
 \item Alexandre Aleluia - @AlexAleluia
 
 Bacharel em direito e atual vereador de Salvador pelo Partido Democratas (DEM), Alexandre Aleluia\footnote{\url{https://twitter.com/AlexAleluia}. Último acesso em 20/02/2021.} é apoiador declarado do governo Bolsonaro e filho de José Carlos Aleluia, também político e apoiador de Bolsonaro. Alexandre teve muitos posicionamentos contrários às pautas feministas, como o direito ao aborto, e também tentou revogar a Lei Teu Nascimento (PL 9498/2019), que pune com multas estabelecimentos que cometam ato de homofobia. O vereador ainda apresentou projeto de lei contra a suposta ``ideologia de gênero'' em Salvador e demonstra contrariedade ao feminismo em redes sociais.

 \begin{figure}[!htbp]
    \centering
    \includegraphics[scale=0.8]{alexandreal_1.png}
    \caption{Postagem no Twitter de Alexandre Aleluia. Vereador acusa Professora Dayane Pimental (ver \ref{depdp}), também conservadora, de performar ``vitimismo feminista''. Disponível em \url{https://twitter.com/alexaleluia/status/1331363942430633989}. Último acesso em 20/02/2021.}
    \label{fig:aleluia}
 \end{figure}

 \item Alexandre Garcia - @alexandregarcia
 
 Jornalista, apresentador e colunista, Alexadre Garcia\footnote{\url{https://twitter.com/alexandregarcia}. Último acesso em 20/02/2021.} foi porta-voz do último presidente da ditadura militar do Brasil, general João Batista Figueiredo, e atualmente escreve para o jornal A Gazeta do Povo, que o listou entre os dez maiores influenciadores digitais da direita política no Brasil. Envolveu-se em várias controvérsias, como elogiar João Dória por combate à suposta ``ideologia de gênero'', criticar o termo ``feminicídio'' e por reação à declaração da atriz Jane Fonda de ter sido estuprada quando era criança.

%  \begin{figure}[!htbp]
%     \centering
%     \includegraphics[scale=0.8]{alexgarcia_1.png}
%     \caption{Postagem no Twitter de Alexandre Garcia. Disponível em \url{https://twitter.com/alexandregarcia/status/826949617300422656}. Último acesso em 20/02/2021.}
%     \label{fig:garciafem}
%  \end{figure}

 \begin{figure}[!htbp]
    \centering
    \includegraphics[scale=0.8]{alexgarcia_2.png}
    \caption{Postagem no Twitter de Alexandre Garcia. Disponível em \url{https://twitter.com/alexandregarcia/status/837438307663560704}. Último acesso em 20/02/2021.}
    \label{fig:garciafonda}
 \end{figure}

 \item Aline Barros - @alinebarros
 
 Cantora, compositora, pastora, escritora, empresária e bióloga, Aline Barros\footnote{\url{https://twitter.com/alinebarros}. Último acesso em 20/02/2021.} é conhecida principalmente por sua carreira enquanto cantora gospel. Em entrevista, disse pensar que Deus poderia transformar gays: ``eu amo o ser humano, mas... o meu posicionamento é sempre amar o homem e abominar o pecado''\footnote{Ver \url{https://www.youtube.com/watch?v=-dTkjqFcmdM}. Último acesso em 20/02/2021.}. Em outra entrevista, afirmou que mulheres não podem esquecer do seu papel dentro de casa, como mãe e esposa\footnote{Ver \url{https://noticias.gospelmais.com.br/conversa-com-bial-aline-barros-mulher-papel-casa-100617.html}. Último acesso em 20/02/2021.}.
 
 \newpage

 \item Ana Paula Henkel - @AnaPaulaVolei
 
 Ana Paula\footnote{\url{https://twitter.com/AnaPaulaVolei}. Último acesso em 20/02/2021.}, é arquiteta e ex-jogadora de voleibol. Atualmente é comentarista política e colunista da Revista Oeste e da Rede Jovem Pan, onde também integra o programa Os Pingos nos Is. Costuma usar seu twitter para fazer ataques ao que chama de ``feminismo atual'', que teria supostamente perdido sua essência. Também costuma fazer críticas à inclusão de mulheres trans em espaços femininos, em especial no esporte feminino.
 
%  \begin{figure}[!htbp]
%     \centering
%     \includegraphics[scale=0.7]{anapaulah_1.png}
%     \caption{Postagem no Twitter de Ana Paula. Disponível em \url{https://twitter.com/AnaPaulaVolei/status/1085284528518660096}. Último acesso em 20/02/2021.}
%  \end{figure}
 
%  \begin{figure}[!htbp]
%     \centering
%     \includegraphics[scale=0.7]{anapaulah_2.png}
%     \caption{Postagem no Twitter de Ana Paula. Disponível em \url{https://twitter.com/AnaPaulaVolei/status/1059162243206533121}. Último acesso em 20/02/2021.}
%  \end{figure}
 
 \begin{figure}[!htbp]
    \centering
    \includegraphics[scale=0.7]{anapaulah_3.png}
    \caption{Postagem no Twitter de Ana Paula Henkel. Disponível em \url{https://twitter.com/AnaPaulaVolei/status/1059172169488326656}. Último acesso em 20/02/2021.}
 \end{figure}

 \item Ana Paula Valadão Bessa - @anapaulavaladao\label{anapaulav}
 
 Cantora, compositora, arranjadora de música, escritora, pastora e apresentadora, Ana Paula Valadão\footnote{\url{https://twitter.com/anapaulavaladao}. Último acesso em 20/02/2021.} é conhecida principalmente por suas canções do gênero gospel. Envolvida em polêmicas envolvendo sobretudo a comunidade LGBTQIA+, Ana Paula já fez falas defendendo que ``relacionamentos não heterossexuais não são normais'' e que levam a ``consequências naturais'' como o HIV\footnote{\url{https://twitter.com/delucca/status/1304790255128776705}. Último acesso em 20/02/2021.}. Em congressos registrados pelo próprio \textit{blog} de seu ``ministério de louvor'' (banda de música cristã), chamado \textit{Diante do Trono}, a cantora e pastora tece comentários que tangem o feminismo. Em uma publicação referente ao Congresso de Homens e Mulheres do Diante do Trono 2017, Ana Paula responsabiliza as mulheres por supostos pesos em seus relacionamentos, quando ``arrancam dos homens a liderança e assumem esse compromisso para elas''. No mesmo congresso, ela ainda afirma que o fato de as mães estarem criando os filhos sem pais tem como consequência homens cada vez mais afeminados\footnote{Ver \url{https://diantedotrono.com/noite-de-abertura-por-que-e-melhor-serem-dois-do-que-um/}. Último acesso em 20/02/2021.}. Em outra postagem com sua assinatura\footnote{Ver \url{https://diantedotrono.com/desrespeito-e-discordia-sem-fim-dia-7-congresso-homens-e-mulheres-dt/}. Último acesso em 20/02/2021.}, Ana Paula denuncia supostos ``extremos do movimento feminista'' e os relaciona ao ``desrespeito à figura do homem no lar e na sociedade''. Outro posicionamento famoso da cantora foi contra a loja C&A por ocasião de uma propaganda da loja em que namorados homossexuais apareciam. Chamando seu incômodo de ``Santa indignação'', ela instruiu seus seguidores a boicotar a marca, alegando ``ideologia de gênero''\footnote{Ver \url{https://www.ligadonogospel.com/2016/05/ana-paula-valadao-comenta-sobre.html}. Último acesso em 21/02/2021.}.

 \item André Fernandes - @andrefernm
 
 Político, André Fernandes\footnote{\url{https://twitter.com/andrefernm}. Último acesso em 20/02/2021.} é atual deputado estadual pelo Ceará. Palestrante do movimento Direita Ceará e colunista do Conexão Política (ver \ref{conexpol}), André já se envolveu em várias controvérsias envolvendo a comunidade LGBTQIA+, ativistas negros e feministas. Entre os ataques desferidos pelo deputado estão sua fala contra a jornalista Patrícia Lélis\footnote{``Quem é tu pra achar que eu vou descer de nível pra debater contigo? Tu só veio aparecer na mídia depois que disse que foi estuprada por Marco Feliciano. (...) Deus me livre chegar perto pra dias depois tu dizer que tá grávida de mim ou que eu te estuprei também''. Ver \url{https://www.opovo.com.br/noticias/politica/2018/02/chamada-de-doida-patricia-lelis-diz-que-cearense-e-mais-um-palhaco.html}. Último acesso em 21/02/2021.}, seu projeto de lei contrário a mulheres trans no esporte feminino e suas várias postagens contra ativistas feministas e LGBTQIA+.
 
%  \begin{figure}[!htbp]
%     \centering
%     \includegraphics[scale=0.7]{andrefern_1.png}
%     \caption{Postagem no Facebook de André Fernandes. Disponível em \url{https://www.facebook.com/andrefernm/posts/905966652887174/}. Último acesso em 20/02/2021.}
%  \end{figure}
 
%  \begin{figure}[!htbp]
%     \centering
%     \includegraphics[scale=0.8]{andrefern_2.png}
%     \caption{Postagem no Twitter de André Fernandes. Disponível em \url{https://twitter.com/andrefernm/status/1334648716746240001?}. Último acesso em 20/02/2021.}
%  \end{figure}
 
%  \begin{figure}[!htbp]
%     \centering
%     \includegraphics[scale=0.7]{andrefern_3.png}
%     \caption{Postagem no Facebook de André Fernandes. Disponível em \url{https://www.facebook.com/andrefernm/posts/1023057037844801/}. Último acesso em 20/02/2021.}
%  \end{figure}
 
 \begin{figure}[!htbp]
    \centering
    \includegraphics[scale=0.7]{andrefern_4.png}
    \caption{Postagem no Twitter de André Fernandes. Disponível em \url{https://twitter.com/andrefernm/status/1012064012828278792?}. Último acesso em 20/02/2021.}
 \end{figure}
 
%  \begin{figure}[!htbp]
%     \centering
%     \includegraphics[scale=0.8]{andrefern_5.png}
%     \caption{Postagem no Twitter de André Fernandes. Disponível em \url{https://twitter.com/andrefernm/status/1134586787068829703?}. Último acesso em 20/02/2021.}
%  \end{figure}
 
 \item Andre Valadao - @andrevaladao
 
 Pastor evangélico, cantor da música cristã, compositor, ator e apresentador, André Valadão\footnote{\url{https://twitter.com/andrevaladao}. Último acesso em 20/02/2021.} é irmão da também cantora Ana Paula Valadão Bessa (Ver \ref{anapaulav}). Contrário ao casamento homossexual, já usou seus cultos para criticar esse tipo de união, e chegou a dizer, na rede social \textit{Instagram}, que igreja não é lugar para gays\footnote{Ver \url{https://catracalivre.com.br/entretenimento/andre-valadao-diz-que-casal-gay-nao-pode-ir-a-igreja/}. Último acesso em 20/02/2021.}. Em relação ao que tange as questões de gênero, também usou seu Instagram para criticar o que classifica como ``igualdade sem lógica'' entre homens e mulheres, afirmando que só a mulher pode gerar e que o homem tem força física superior à da mulher --- e, por isso, deveria ser provedor. No mesmo texto ainda ataca uma suposta geração masculina afeminada\footnote{Ver \url{https://www.instagram.com/p/ByDQG-NBPwa}. Último acesso em 20/02/2021.}.

 \item Arthur Weintraub - @ArthurWeint
 
 Acadêmico especializado em direito, Arthur Weintraub\footnote{\url{https://twitter.com/ArthurWeint}. Último acesso em 20/02/2021.} é irmão mais novo de Abraham Weintraub (ver \ref{abweint}), e, assim como o irmão, trabalhou conjuntamente com o governo Bolsonaro. Abertamente contrário ao direito ao aborto, Arthur já utilizou suas redes sociais para criticar a suposta ``ideologia de gênero'' e tentar ofender uma usuária do Twitter chamando-a de ``feminista suvaqueira''.
 
%  \begin{figure}[!htbp]
%     \centering
%     \includegraphics[scale=0.6]{arthurweint_3.png}
%     \caption{Postagem no Twitter de Arthur Weintraub. Disponível em \url{https://twitter.com/ArthurWeint/status/1253455977841602560}. Último acesso em 20/02/2021.}
%  \end{figure}
 
%  \begin{figure}[!htbp]
%     \centering
%     \includegraphics[scale=0.8]{arthurweint_1.png}
%     \caption{Postagem no Twitter de Arthur Weintraub. Disponível em \url{https://twitter.com/arthurweint/status/1207985637586087936}. Último acesso em 20/02/2021.}
%  \end{figure}
 
 \begin{figure}[!htbp]
    \centering
    \includegraphics[scale=0.55]{arthurweint_2.png}
    \caption{Postagem no Twitter de Arthur Weintraub, em resposta a usuária que o chamou de ``babaca''. Disponível em \url{https://twitter.com/ArthurWeint/status/1190701724014727169}. Último acesso em 20/02/2021.}
 \end{figure}
 
  \item Bia Kicis - @Biakicis
  
  Advogada, procuradora aposentada do Distrito Federal, \textit{youtuber} e política, Bia Kicis\footnote{\url{https://twitter.com/Biakicis}. Último acesso em 20/02/2021.} é atualmente deputada federal pelo Distrito Federal, de atuação conservadora e alinhada ao governo Bolsonaro. Segundo levantamento do \textit{Aos Fatos}, mídia especializada em verificação de integridade de informação e detecção de \textit{fake news}, Bia Kicis foi a quarta maior disseminadora de desinformação sobre Covid-19 entre parlamentares no Twitter\footnote{Segundo artigo, a deputada só disseminou menos informações falsas sobre Covid-19 que Osmar Terra, Eduardo Bolsonaro e Carla Zambeli. \url{https://www.aosfatos.org/noticias/deputados-governistas-lideram-desinformacao-sobre-covid-19-entre-parlamentares-no-twitter/}. Último acesso em 20/02/2021.}. Apoiadora do Escola Sem Partido, é irmã da esposa de Miguel Nagib, criador do projeto. Bia Kicis já se posicionou inúmeras vezes em suas redes contra a descriminalização do aborto e contra a ``ideologia de gênero'', além de fazer uma postagem contra Moro e Mandetta que foi considerada racista\footnote{Ver \url{https://www.poder360.com.br/midia/bia-kicis-e-acusada-de-racismo-por-montagem-de-moro-e-mandetta/}. Último acesso em 20/02/2021.}.
 
%  \begin{figure}[!htbp]
%     \centering
%     \includegraphics[scale=0.6]{biak_1.png}
%     \caption{Postagem no Facebook de Bia Kicis. Disponível em \url{https://www.facebook.com/biakicisoficial/posts/1882109305288956}. Último acesso em 20/02/2021.}
%  \end{figure}
 
%  \begin{figure}[!htbp]
%     \centering
%     \includegraphics[scale=0.5]{biak_2.png}
%     \caption{Postagem no Twitter de Bia Kicis. Disponível em \url{https://twitter.com/Biakicis/status/1149451785259638784}. Último acesso em 20/02/2021.}
%  \end{figure}
 
 \begin{figure}[!htbp]
    \centering
    \includegraphics[scale=0.6]{biak_3.png}
    \caption{Postagem no Twitter de Bia Kicis. Disponível em \url{https://twitter.com/Biakicis/status/831530454100623361}. Último acesso em 20/02/2021.}
 \end{figure}
  
  \item Bibo Nunes - @bibonunes1
  
  Jornalista, apresentador e político, Bibo Nunes\footnote{\url{https://twitter.com/bibonunes1}. Último acesso em 20/02/2021.} é atualmente deputado federal pelo Rio Grande do Sul e um dos grandes aliados governistas de Jair Bolsonaro. Declaradamente contra a ``ideologia de gênero'' e a descriminalização do aborto, o deputado foi recentemente classificado como misógino ao chamar deputadas de oposição de ``histéricas'' e ``deputéricas''\footnote{Declarações das deputadas feministas sobre o episódio e a resposta de Bibo estão registradas em \url{https://www.camara.leg.br/noticias/712629-bibo-nunes-acusa-deputadas-de-oposicao-de-histeria-e-causa-revolta-no-plenario}. Último acesso em 20/02/2021.}.
 
 \begin{figure}[!htbp]
    \centering
    \includegraphics[scale=0.55]{bibonunes_1.png}
    \caption{Postagem no Twitter de Bibo Nunes. Disponível em \url{https://twitter.com/bibonunes1/status/1337465387454242817}. Último acesso em 20/02/2021.}
 \end{figure}
 
  \item Bruna Karla - @brunakarlabk
  
  Cantora de música cristã, Bruna Karla\footnote{\url{https://twitter.com/brunakarlabk}. Último acesso em 20/02/2021.} já se posicionou contra a ``ideologia de gênero'' e criticou famílias não heterossexuais e casais homossexuais\footnote{A entrevista completa pode ser acessada em \url{https://pleno.news/fe/gravida-bruna-karla-atesta-nao-aceito-ensinar-ideologia-de-genero-ao-meu-filho.html}. Último acesso em 21/02/2021.}.

  \item Bruno Engler - @BrunoEnglerDM\label{engler}
  
  Bruno Engler\footnote{\url{https://twitter.com/BrunoEnglerDM}. Último acesso em 20/02/2021.} é político e coordenador das redes sociais do Movimento Direita Minas (ver \ref{dirmin}). Atualmente exerce o cargo de deputado estadual por Minas Gerais. Ligado ao governo Bolsonaro, o deputado em várias postagens critica o movimento feminista e a ``ideologia de gênero'' e se posiciona contra a descriminalização do aborto.
 
 \begin{figure}[!htbp]
    \centering
    \includegraphics[scale=0.7]{brunoe_1.png}
    \caption{Postagem no Twitter de Bruno Engler. Disponível em \url{https://twitter.com/BrunoEnglerDM/status/1311013791228887043}. Último acesso em 20/02/2021.}
 \end{figure}
 
%  \begin{figure}[!htbp]
%     \centering
%     \includegraphics[scale=0.8]{bruno1_2.png}
%     \caption{Postagem no Twitter de Bruno Engler. Disponível em \url{https://twitter.com/BrunoEnglerDM/status/1205536819884052480}. Último acesso em 20/02/2021.}
%  \end{figure}
 
%  \begin{figure}[!htbp]
%     \centering
%     \includegraphics[scale=0.8]{brunoe_3.png}
%     \caption{Postagem no Twitter de Bruno Engler. Disponível em \url{https://twitter.com/BrunoEnglerDM/status/1353842260589346816}. Último acesso em 20/02/2021.}
%  \end{figure}
 
 \newpage
 
  \item Capitão Derrite - @capitaoderrite
  
  Militar e político, Capitão Derrite\footnote{\url{https://twitter.com/capitaoderrite}. Último acesso em 20/02/2021.} é atualmente deputado federal por São Paulo. Apoiador de Bolsonaro, costuma atacar a ``ideologia de gênero'' em seu Twitter.
 
 \begin{figure}[!htbp]
    \centering
    \includegraphics[scale=0.8]{derr_1.png}
    \caption{Postagem no Twitter de Capitão Derrite. Disponível em \url{https://twitter.com/capitaoderrite/status/1341067846697271299}. Último acesso em 20/02/2021.}
 \end{figure}
 
%  \begin{figure}[!htbp]
%     \centering
%     \includegraphics[scale=0.8]{derr_2.png}
%     \caption{Postagem no Twitter de Capitão Derrite. Disponível em \url{https://twitter.com/capitaoderrite/status/1169271238138695680}. Último acesso em 20/02/2021.}
%  \end{figure}
 
  \item Carla Zambelli - @CarlaZambelli38
  
  Gerente de projetos, fundadora e ex-líder do movimento de combate à impunidade e à corrupção NasRuas, e atual deputada federal eleita por São Paulo, Carla Zambelli\footnote{\url{https://twitter.com/CarlaZambelli38}. Último acesso em 20/02/2021.} já foi apontada como participante do controverso movimento Femen Brasil\footnote{Conferir notícia do portal Terra, de 2012, que aponta Zambelli como porta voz do coletivo \url{https://www.terra.com.br/noticias/brasil/cidades/ato-do-femen-na-avenida-paulista-e-recebido-com-indiferenca,1320dc840f0da310VgnCLD200000bbcceb0aRCRD.html}, site de apoio e apresentação do coletivo \url{https://supportfemen.com/pages/about-us} e questionamentos ao feminismo do Femen Brasil \url{https://operamundi.uol.com.br/politica-e-economia/24385/femen-brazil-nao-tem-propostas-feministas-acusa-ex-numero-2-do-grupo}. Último acesso em 21/02/2021.} no passado, apesar de suas ideias contrárias ao feminismo hoje. No congresso e em redes sociais, Carla costuma se posicionar contrariamente às pautas feministas, inclusive se valendo de informações de veracidade questionável (ver figura \ref{fig:carlaquest}), chegando a ataques diretos às ativistas\footnote{Carla chega a dizer que a rede social Instagram deixou de exibir a quantidade de \textit{likes} que uma foto recebeu por conta de ``gorda feminista peluda do cabelo roxo'' que fica deprimida. Disponível em \url{https://catracalivre.com.br/cidadania/carla-zambelli-e-detonada-na-web-apos-atacar-mulheres-gordas/}. Último acesso em 21/02/2021.}.
 
 \begin{figure}[!htbp]
    \centering
    \includegraphics[scale=0.9]{carlaz_1.png}
    \caption{Postagem no Facebook de Carla Zambelli com informação falsa sobre a descriminalização do aborto até os 9 meses de gestação na França. Filtro de informação falsa aplicado pelo próprio Facebook. Disponível em \url{https://www.facebook.com/ZambelliOficial/posts/3288472751243208/}. Último acesso em 20/02/2021.}
    \label{fig:carlaquest}
 \end{figure}

\newpage

  \item Carlos Bolsonaro - @CarlosBolsonaro
  
  Filho do atual presidente Jair Bolsonaro (ver \ref{jbols}), formado em ciências aeronáuticas e político, Carlos Bolsonaro\footnote{\url{https://twitter.com/CarlosBolsonaro}. Último acesso em 20/02/2021.} atualmente exerce o cargo de vereador pelo Rio de Janeiro. Declara-se contra o aborto e a ``ideologia de gênero'', e já comparou feministas a nazistas através do termo ``feminazi''.
 
 \begin{figure}[!htbp]
    \centering
    \includegraphics[scale=0.95]{carlosb_2.png}
    \caption{Postagem no Twitter de Carlos Bolsonaro. Disponível em \url{https://twitter.com/CarlosBolsonaro/status/879787055483637760}. Último acesso em 20/02/2021.}
    \label{fig:carlosb1}
 \end{figure}
 
%  \begin{figure}[!htbp]
%     \centering
%     \includegraphics[scale=0.9]{carlosb_3.png}
%     \caption{Postagem no Twitter de Carlos Bolsonaro. Disponível em \url{https://twitter.com/CarlosBolsonaro/status/869323590835077120}. Último acesso em 20/02/2021.}
%     \label{fig:carlosb2}
%  \end{figure}
 
%  \begin{figure}[!htbp]
%     \centering
%     \includegraphics[scale=0.9]{carlosb_4.png}
%     \caption{Postagem no Twitter de Carlos Bolsonaro. Disponível em \url{https://twitter.com/CarlosBolsonaro/status/804798452257460224}. Último acesso em 20/02/2021.}
%     \label{fig:carlosb3}
%  \end{figure}
 
%  \begin{figure}[!htbp]
%     \centering
%     \includegraphics[scale=0.9]{carlosb_5.png}
%     \caption{Postagem no Twitter de Carlos Bolsonaro. Disponível em \url{https://twitter.com/CarlosBolsonaro/status/938395228829970432}. Último acesso em 20/02/2021.}
%     \label{fig:carlosb4}
%  \end{figure}
  
  \item Carlos Jordy - @carlosjordy
  
  Formado em turismo e hotelaria e pós-graduado em gestão pública municipal, Carlos Jordy\footnote{\url{https://twitter.com/carlosjordy}. Último acesso em 20/02/2021.} trabalhou como funcionário público concursado e atualmente é deputado federal pelo Rio de Janeiro. Alinhado ao governo Bolsonaro, Jordy expressou diversas vezes em seu Twitter sua contrariedade ao movimento feminista e LGBT e suas pautas.
 
%  \begin{figure}[!htbp]
%     \centering
%     \includegraphics[scale=1]{cjordy_1.png}
%     \caption{Postagem no Twitter de Carlos Jordy. Disponível em \url{https://twitter.com/carlosjordy/status/1217505282273751040}. Último acesso em 20/02/2021.}
%     \label{fig:carlosj1}
%  \end{figure}
 
 \begin{figure}[!htbp]
    \centering
    \includegraphics[scale=0.9]{cjordy_2.png}
    \caption{Postagem no Twitter de Carlos Jordy. Disponível em \url{https://twitter.com/carlosjordy/status/1160231238134849537}. Último acesso em 20/02/2021.}
    \label{fig:carlosj2}
 \end{figure}
 
%  \begin{figure}[!htbp]
%     \centering
%     \includegraphics[scale=1]{cjordy_3.png}
%     \caption{Postagem no Twitter de Carlos Jordy. Disponível em \url{https://twitter.com/carlosjordy/status/1220848680867127302}. Último acesso em 20/02/2021.}
%     \label{fig:carlosj3}
%  \end{figure}
 
 \newpage
  
  \item Carlos Viana - @carlosaviana (Com ressalvas)
  
   Jornalista e político, Carlos Viana\footnote{\url{https://twitter.com/carlosaviana}. Último acesso em 20/02/2021.} exerce atualmente o cargo de senador por Minas Gerais. Apesar de apoiador do governo Bolsonaro e de ter se posicionado contrariamente à suposta ``ideologia de gênero''\footnote{Posicionamento disponível em \url{https://jornalistacarlosviana.com.br/blog/viana-e-contra-a-ideologia-de-genero/}. Último acesso em 21/02/2021.} em 2018 durante as eleições para senador, Viana não se declara radicalmente contrário às pautas do movimento feminista e está nessa seção \textit{com ressalvas}. Em seu \textit{site}\footnote{Disponível em \url{https://jornalistacarlosviana.com.br/blog/pergunte-para-o-viana/}. Último acesso em 21/02/2021.}, Viana diz entender a descriminalização do aborto como questão de saúde pública, que deve ser avaliada caso a caso. Além disso, não costuma se posicionar de maneira ofensiva ou em ataque direto às feministas, ao feminismo ou mesmo ao movimento LGBTQIA+.
 
  \item Carmelo Neto - @carmelonetobr
  
  Carmelo Neto\footnote{\url{https://twitter.com/carmelonetobr}. Último acesso em 20/02/2021.}, ex-integrante do Movimento Brasil Livre (MBL) e do movimento NasRuas, fundado por Carla Zambelli. Ex-conselheiro de juventude do Governo Federal, atualmente é vereador de Fortaleza. No que tange às pautas feministas, se posiciona contrariamente à descriminalização do aborto e também se coloca contrário à ``ideologia de gênero'' e favorável ao Escola sem partido.
 
%  \begin{figure}[!htbp]
%     \centering
%     \includegraphics[scale=1]{cneto_1.png}
%     \caption{Postagem no Twitter de Carmelo Neto. Disponível em \url{https://twitter.com/carmelonetobr/status/950803372885585920}. Último acesso em 20/02/2021.}
%  \end{figure}
 
 \begin{figure}[!htbp]
    \centering
    \includegraphics[scale=0.8]{cneto_2.png}
    \caption{Postagem no Twitter de Carmelo Neto. Disponível em \url{https://twitter.com/carmelonetobr/status/1254203047468818434}. Último acesso em 20/02/2021.}
 \end{figure}
  

  \item Caroline De Toni - @CarolDeToni
  
  Política, advogada e mestre em direito público, Caroline De Toni\footnote{\url{https://twitter.com/CarolDeToni}. Último acesso em 20/02/2021.} é atualmente deputada federal por Santa Catarina. Declaradamente contrária ao movimento feminista, a deputada apresentou em 2020 proposta para extinguir a reserva mínima de $30\%$ das vagas para mulheres nas candidaturas para mandatos eletivos preenchidos pelo sistema proporcional. Em sua argumentação da proposta, ela afirma que há ``carga ideológica que cerca o tema igualdade de gênero'' e que ``infelizmente apenas uma parcela muito pequena das mulheres de fato, se interessa por desenvolver atividade político-partidária''\footnote{Veja mais sobre a proposta em \url{https://www.cfemea.org.br/index.php/alerta-feminista/4831-deputada-do-psl-e-mais-uma-na-fila-de-quem-quer-menos-mulheres-na-politica}. Último acesso em 21/02/2021.}.
 
 \begin{figure}[!htbp]
    \centering
    \includegraphics[scale=0.65]{carolt_1.png}
    \caption{Postagem no \textit{Instagram} de Caroline De Toni. Disponível em \url{https://www.instagram.com/p/CLHwT31DB0O/}. Último acesso em 20/02/2021.}
 \end{figure}
 
%  \begin{figure}[!htbp]
%     \centering
%     \includegraphics[scale=0.7]{carolt_2.png}
%     \caption{Postagem no Twitter de Caroline De Toni. Disponível em \url{https://twitter.com/CarolDeToni/status/1280683594411921410}. Último acesso em 20/02/2021.}
%  \end{figure}
 
%  \begin{figure}[!htbp]
%     \centering
%     \includegraphics[scale=0.9]{carolt_3.png}
%     \caption{Postagem no Twitter de Caroline De Toni. Disponível em \url{https://twitter.com/CarolDeToni/status/1321817462179069955}. Último acesso em 20/02/2021.}
%  \end{figure}
 
  \item Chris Tonietto - @ToniettoChris
  
  Advogada e política, Chris Tonietto\footnote{\url{https://twitter.com/ToniettoChris}. Último acesso em 20/02/2021.} é atual deputada federal pelo Rio de Janeiro. Conservadora, já foi processada pelo Ministério Público Federal por crime de homofobia por uma de suas postagens, quando procuradores compreenderam que ela associava homossexuais à pedofilia\footnote{Uma cobertura mais completa pode ser encontrada em \url{https://www.poder360.com.br/congresso/mpf-move-acao-contra-deputada-do-psl-por-postagem-homofobica/}. Último acesso em 21/02/2021.}. Integrante da chamada Frente parlamentar mista contra o aborto e em defesa da vida, Chris é contrária à descriminalização do aborto e já fez publicações contra a ``ideologia de gênero'' e o movimento feminista.
 
%  \begin{figure}[!htbp]
%     \centering
%     \includegraphics[scale=0.75]{christ_2.png}
%     \caption{Postagem no Twitter de Chris Tonietto. Disponível em \url{https://twitter.com/ToniettoChris/status/1324138859123408896}. Último acesso em 20/02/2021.}
%  \end{figure}
 
%  \begin{figure}[!htbp]
%     \centering
%     \includegraphics[scale=0.65]{christ_1.png}
%     \caption{Postagem no \textit{Instagram} de Chris Tonietto. Disponível em \url{https://www.instagram.com/p/CLXg4IOpFV5/}. Último acesso em 20/02/2021.}
%  \end{figure}
 
%  \begin{figure}[!htbp]
%     \centering
%     \includegraphics[scale=0.9]{christ_3.png}
%     \caption{Postagem no Twitter de Chris Tonietto. Disponível em \url{https://twitter.com/ToniettoChris/status/1236798582457597952}. Último acesso em 20/02/2021.}
%  \end{figure}
  
  \item Clau de Luca - @ClaudeLuca$\_$
  
  Cantora, publicitária e autodeclarada bolsonarista, Clau de Luca\footnote{\url{https://twitter.com/ClaudeLuca_}. Último acesso em 20/02/2021.} foi candidata a deputada federal por São Paulo em 2018 e a vereadora em 2020, não sendo eleita em nenhum dos casos. Contrária às pautas feministas e a favor do Escola sem Partido, Clau já usou o twitter para atacar feministas.
 
%  \begin{figure}[!htbp]
%     \centering
%     \includegraphics[scale=0.95]{clau_1.png}
%     \caption{Postagem no Twitter de Clau de Luca. Disponível em \url{https://twitter.com/ClaudeLuca_/status/1040978025939193860}. Último acesso em 20/02/2021.}
%  \end{figure}
 
 \begin{figure}[!htbp]
    \centering
    \includegraphics[scale=0.75]{clau_2.png}
    \caption{Postagem no \textit{Instagram} de Clau de Luca. Disponível em \url{https://twitter.com/ClaudeLuca_/status/1272159662582968320}. Último acesso em 20/02/2021.}
 \end{figure}
 
  
 \item Damares Alves - @DamaresAlves
 
 Damares Alves\footnote{\url{https://twitter.com/DamaresAlves}. Último acesso em 20/02/2021.} é advogada e pastora evangélica, exercendo atualmente o cargo de ministra da Mulher, da Família e dos Direitos Humanos (MMFDH) do governo Bolsonaro. Envolvida em muitas controvérsias, em dezembro de 2019, em entrevista ao jornal O Estado de São Paulo, afirmou que ``se lutar pelo direito das mulheres é ser feminista, eu sou''\footnote{Veja entrevista completa em \url{https://brasil.estadao.com.br/noticias/geral,mamae-damares-vai-mandar-bola-livro-arroz-e-feijao-camisinha-nao-afirma-ministra,70003136949}. Último acesso em 22/02/2021.}. Sua fala supostamente abraça as ideias do movimento feminista. No entanto, no dia 8 de março de 2020, menos de três meses depois da entrevista ao jornal, Damares afirmou que não é feminista, mas ``feminina'', ao R7\footnote{Veja entrevista completa em \url{https://noticias.r7.com/prisma/r7-planalto/eu-nao-sou-feminista-sou-feminina-diz-ministra-damares-alves-09032020}. Último acesso em 22/02/2021.}. A entrevista ao R7 não surpreende, uma vez que em várias oportunidades anteriores, Damares atacou tanto o movimento feminista\footnote{Em março de 2019, em ocasião do encontro O Protagonismo da Mulher Jovem no Brasil, organizado pelo MMFDH, houve um painel dedicado a criticar o movimento, chamado ``As armadilhas do feminismo''. Mais detalhes sobre o encontro e o painel podem ser encontrados no próprio \textit{site} do Ministério \url{https://www.gov.br/mdh/pt-br/assuntos/noticias/todas-as-noticias/2019/marco/ministerio-promove-encontro-201co-protagonismo-da-mulher-jovem-no-brasil201d}. Último acesso em 22/02/2021.} quanto as próprias feministas\footnote{Veja a declaração ``feministas não gostam de homens porque são feias'' de Damares em \url{https://catracalivre.com.br/dimenstein/damares-alves-feministas-nao-gostam-de-homens-porque-sao-feias/}. Último acesso em 22/02/2021.}. Damares também se posiciona contra a ``ideologia de gênero'' e favoravelmente ao Escola sem partido, além de ter feito declarações contrárias aos relacionamentos homossexuais\footnote{Em artigo, a revista Fórum organizou algumas falas de Damares sobre homossexuais; em uma delas, a atual ministra classifica as relações homossexuais como ``aberrações''. Veja artigo completo em  \url{https://revistaforum.com.br/lgbt/exclusivo-em-clinica-de-restauracao-de-sexualidade-damares-classifica-homossexualidade-como-aberracao/}. Último acesso em 22/02/2021.}.
 
 \begin{figure}[!htbp]
    \centering
    \includegraphics[scale=0.9]{damares_1.png}
    \caption{Postagem no Twitter da Vide Editorial sobre recomendação de Damares do livro ``Feminismo: perversão e subversão''. A autora é a apresentadora do painel ``As armadilhas do feminismo'', que ocorreu em evento organizado pelo MMFDH. Disponível em \url{https://twitter.com/videeditorial/status/1096232055925547008}. Último acesso em 20/02/2021.}
 \end{figure}
  
 \item Daniel Freitas - @DFDanielFreitas
 
 Empresário e político, Daniel Freitas\footnote{\url{https://twitter.com/DFDanielFreitas}. Último acesso em 20/02/2021.} exerce atualmente o cargo de deputado federal por Santa Catarina. Em 2020, a esposa do deputado registrou boletim de ocorrência de violência doméstica contra o mesmo; o processo investigativo foi arquivado um mês depois por falta de provas\footnote{A notícia completa pode ser acessada em  \url{https://ndmais.com.br/justica/justica-arquiva-processo-contra-deputado-daniel-freitas-por-violencia-domestica/}. Último acesso em 22/02/2021.}. Posiciona-se contrariamente à descriminalização do aborto e à ``ideologia de gênero'' e favoravelmente ao Escola sem partido.
 
 \begin{figure}[!htbp]
    \centering
    \includegraphics[scale=0.9]{dfreitas_1.png}
    \caption{Postagem no Twitter de Daniel Freitas. Disponível em \url{https://twitter.com/DFDanielFreitas/status/1293682254200410121}. Último acesso em 20/02/2021.}
 \end{figure}
 
%  \begin{figure}[!htbp]
%     \centering
%     \includegraphics[scale=0.9]{dfreitas_2.png}
%     \caption{Postagem no Facebook de Daniel Freitas. Disponível em \url{https://www.facebook.com/depdanielfreitas/videos/3996400493703707/}. Último acesso em 20/02/2021.}
%  \end{figure}
  
 \item Daniel Silveira - @danielPMERJ
 
 Daniel Silveira\footnote{\url{https://twitter.com/danielPMERJ}. Último acesso em 20/02/2021.} é ex-policial militar e político, atualmente no cargo de deputado federal pelo Rio de Janeiro. Daniel foi preso durante a execução desse trabalho, após publicar um vídeo de apologia ao AI-5 --- o decreto do período militar que fechou o congresso nacional e permitiu a institucionalização da tortura. Daniel ficou conhecido principalmente após a extrema popularização de um vídeo em que quebrava uma placa de homenagem à vereadora Marielle Franco. Abertamente contrário às feministas, Daniel também se declara contrário à ``ideologia de gênero'', à descriminalização do aborto e favorável ao Escola sem partido.
 
 \begin{figure}[!htbp]
    \centering
    \includegraphics[scale=0.9]{daniels_1.png}
    \caption{Postagem no Twitter de Daniel Freitas. Disponível em \url{https://twitter.com/danielPMERJ/status/1359824329370959873}. Último acesso em 16/02/2021. Daniel teve sua conta suspensa por ordem do STF em 19/02/2021.}
 \end{figure}
 
%  \begin{figure}[!htbp]
%     \centering
%     \includegraphics[scale=0.7]{daniels_2.png}
%     \caption{Postagem no Twitter de Daniel Freitas. Disponível em \url{https://twitter.com/danielPMERJ/status/1238656330673643520}. Último acesso em 16/02/2021. Daniel teve sua conta suspensa por ordem do STF em 19/02/2021.}
%  \end{figure}
 
%  \begin{figure}[!htbp]
%     \centering
%     \includegraphics[scale=0.7]{daniels_3.png}
%     \caption{Postagem no Twitter de Daniel Freitas. Disponível em \url{https://twitter.com/danielPMERJ/status/1309150308367568899}. Último acesso em 16/02/2021. Daniel teve sua conta suspensa por ordem do STF em 19/02/2021.}
%  \end{figure}
 
%  \begin{figure}[!htbp]
%     \centering
%     \includegraphics[scale=0.9]{daniels_4.png}
%     \caption{Postagem no Twitter de Joice Hasselmann. Em vídeo, Daniel Silveira diz não contratar negros para seu gabinete, e chama os mesmos de principais perpetuadores do racismo hoje, por serem ``racistas contra brancos''. Disponível em \url{https://twitter.com/joicehasselmann/status/1363272594250731520}. Último acesso em 16/02/2021. Daniel teve sua conta suspensa por ordem do STF em 19/02/2021.}
%  \end{figure}
 
 \newpage
 
 \item David M. Quinlan - @DavidMQuinlan
 
 Cantor e missionário, David Martin Quinlan \footnote{\url{https://twitter.com/DavidMQuinlan}. Último acesso em 20/02/2021.} é irlandês e veio ao Brasil em missão religiosa com os pais. Fundou o ministério ``Paixão, Fogo e Glória'' e através dele faz conferências e eventos ao redor do país e do mundo. Em seu twitter, conta com quase seiscentos mil seguidores\footnote{Verificado em 22/02/2021.} e se posiciona contra o ``kit gay'', a ``ideologia de gênero'', a descriminalização do aborto e favoravelmente ao Escola sem partido e à ``família tradicional'' (cisgênera e heterossexual).
 
%  \begin{figure}[!htbp]
%     \centering
%     \includegraphics[scale=0.65]{dq_5.png}
%     \caption{Postagem no Twitter de David M. Quinlan. Disponível em \url{https://twitter.com/DavidMQuinlan/status/72679402005209089}. Último acesso em 22/02/2021.}
%  \end{figure}
 
%  \begin{figure}[!htbp]
%     \centering
%     \includegraphics[scale=0.65]{dq_4.png}
%     \caption{Postagem no Twitter de David M. Quinlan. Disponível em \url{https://twitter.com/DavidMQuinlan/status/920019158540185601}. Último acesso em 22/02/2021.}
%  \end{figure}
 
%  \begin{figure}[!htbp]
%     \centering
%     \includegraphics[scale=0.65]{dq_2.png}
%     \caption{Postagem no Twitter de David M. Quinlan. Disponível em \url{https://twitter.com/DavidMQuinlan/status/1025383118327431169}. Último acesso em 22/02/2021.}
%  \end{figure}
 
%  \begin{figure}[!htbp]
%     \centering
%     \includegraphics[scale=0.65]{dq_1.png}
%     \caption{Postagem no Twitter de David M. Quinlan. Disponível em \url{https://twitter.com/DavidMQuinlan/status/1058910647696744448}. Último acesso em 22/02/2021.}
%  \end{figure}
 
 \begin{figure}[!htbp]
    \centering
    \includegraphics[scale=0.65]{dq_3.png}
    \caption{Postagem no Twitter de David M. Quinlan. Disponível em \url{https://twitter.com/DavidMQuinlan/status/211082911175286787}. Último acesso em 22/02/2021.}
 \end{figure}
 
 \newpage
 
 \item Davy Albuquerque - @AlbuquerqueDavy
 
 Davy Albuquerque\footnote{\url{https://twitter.com/AlbuquerqueDavy}. Último acesso em 20/02/2021.} é chefe de redação da mídia conservadora Conexão Política (ver \ref{conexpol}) e suposto assessor do deputado estadual Alexandre Knoploch\footnote{Em matéria da revista Época, a Conexão Política é investigada em relação à origem de seu financiamento. A investigação feita pela revista aponta Knoploch como responsável pelo funcionamento da mídia conservadora. Veja matéria completa em \url{https://epoca.globo.com/opiniao-como-nasce-um-embuste-23397102}. Último acesso em 22/02/2021.}. Candidato a vereador do Rio de Janeiro em 2020, recebeu pouco mais de dois mil votos e acabou não sendo eleito. Declarado conservador, é contra o movimento feminista, suas ativistas e suas pautas.
 
 \begin{figure}[!htbp]
    \centering
    \includegraphics[scale=1]{da_3.png}
    \caption{Postagem no Twitter de Davy Albuquerque. Disponível em \url{https://twitter.com/AlbuquerqueDavy/status/1344301499602661376}. Último acesso em 22/02/2021.}
 \end{figure}
 
%  \begin{figure}[!htbp]
%     \centering
%     \includegraphics[scale=0.9]{da_2.png}
%     \caption{Postagem no Twitter de Davy Albuquerque. Disponível em \url{https://twitter.com/AlbuquerqueDavy/status/1022621797055963138}. Último acesso em 22/02/2021.}
%  \end{figure}
 
%  \begin{figure}[!htbp]
%     \centering
%     \includegraphics[scale=0.9]{da_1.png}
%     \caption{Postagem no Twitter de Davy Albuquerque. Disponível em \url{https://twitter.com/AlbuquerqueDavy/status/1263557228700872708}. Último acesso em 22/02/2021.}
%  \end{figure}
 
 \item Delegado Éder Mauro - @EderMauroPA
 
 Éder Mauro\footnote{\url{https://twitter.com/EderMauroPA}. Último acesso em 20/02/2021.} é delegado de polícia e político. Atualmente exerce o cargo de deputado federal pelo Pará. Em várias postagens em sua página do Twitter se posiciona contra a descriminalização do aborto, o ``kit gay'' e a ``ideologia de gênero'', além de costumar fazer piadas sobre pessoas supostamente LGBTQIA+. O deputado já foi acusado de homofobia por uma de suas postagens no Facebook\footnote{Uma descrição maior do caso está disponível em \url{https://www.itafm.com.br/2019/07/05/lideranca-estudantil-denuncia-deputado-federal-eder-mauro-a-pgr-por-homofobia/}. Último acesso em 22/02/2021.} e também respondeu por agressão contra uma servidora pública transexual\footnote{A notícia completa está disponível em \url{https://g1.globo.com/pa/para/noticia/2019/05/28/deputado-federal-eder-mauro-e-acusado-de-agredir-servidora-transexual-durante-votacao-em-belem.ghtml}. Último acesso em 22/02/2021.}.
 
 \begin{figure}[!htbp]
    \centering
    \includegraphics[scale=0.65]{dem_2.png}
    \caption{Postagem no Twitter do Delegado Éder Mauro. Disponível em \url{https://twitter.com/EderMauroPA/status/1196977528432398336}. Último acesso em 22/02/2021.}
 \end{figure}
 
 \begin{figure}[!htbp]
    \centering
    \includegraphics[scale=0.8]{dem_3.png}
    \caption{Postagem no Twitter do Delegado Éder Mauro. Disponível em \url{https://twitter.com/EderMauroPA/status/1213460593363030018}. Último acesso em 22/02/2021.}
 \end{figure}
 
%  \begin{figure}[!htbp]
%     \centering
%     \includegraphics[scale=0.9]{dem_1.png}
%     \caption{Postagem no Twitter do Delegado Éder Mauro. Disponível em \url{https://twitter.com/EderMauroPA/status/1255243019017687041}. Último acesso em 22/02/2021.}
%  \end{figure}
 
%  \newpage
  
 \item Dom - @domlancellotti
 
 Dom Lancellotti\footnote{\url{https://twitter.com/domlancellotti}. Último acesso em 20/02/2021.} é fundador do movimento \textit{Gays com Bolsonaro}. Tentou a candidatura à vereador de Fortaleza em 2020, e conseguiu se classificar para suplente com os seus 285 votos. Se posiciona contra o ``kit gay'', a ideologia de gênero, a descriminalização do aborto, o casamento igualitário, o movimento LGBTQIA+ e feminista, e se identifica com a teoria da conspiração de direita QAnon\footnote{Para mais detalhes sobre a QAnon e brasileiros que a seguem, ver \url{https://theintercept.com/2020/10/07/qanon-quatro-candidatos-a-vereador-mostram-como-conspiracao-invadiu-estas-eleicoes-municipais/}. Último acesso em 22/02/2021.}.
 
 \begin{figure}[!htbp]
    \centering
    \includegraphics[scale=0.9]{dom_1.png}
    \caption{Postagem no Twitter de Dom. Disponível em \url{https://twitter.com/domlancellotti/status/1010387642083667968}. Último acesso em 22/02/2021.}
 \end{figure}
 
%  \begin{figure}[!htbp]
%     \centering
%     \includegraphics[scale=0.9]{dom_3.png}
%     \caption{Postagem no Twitter de Dom. Disponível em \url{https://twitter.com/domlancellotti/status/1168695822709645312}. Último acesso em 22/02/2021.}
%  \end{figure}
 
%  \begin{figure}[!htbp]
%     \centering
%     \includegraphics[scale=0.6]{dom_4.png}
%     \caption{Postagem no Twitter de Dom. Disponível em \url{https://twitter.com/domlancellotti/status/1053011246172332035}. Último acesso em 22/02/2021.}
%  \end{figure}
 
%  \begin{figure}[!htbp]
%     \centering
%     \includegraphics[scale=0.9]{dom_5.png}
%     \caption{Postagem no Twitter de Dom. Disponível em \url{https://twitter.com/domlancellotti/status/1170433702381400065}. Último acesso em 22/02/2021.}
%  \end{figure}
 
 \begin{figure}[!htbp]
    \centering
    \includegraphics[scale=0.9]{dom_7.png}
    \caption{Postagem no Twitter de Dom. Disponível em \url{https://twitter.com/domlancellotti/status/1257900928558080000}. Último acesso em 22/02/2021.}
 \end{figure}
 
 \item Douglas Garcia - @DouglasGarcia\label{douglasgarcia}
 
 Ex-estudante de Relações Internacionais e vice-presidente do Movimento Conservador (antigo Direita São Paulo), Douglas Garcia\footnote{\url{https://twitter.com/DouglasGarcia}. Último acesso em 20/02/2021.} é político, exercendo atualmente o cargo de deputado estadual por São Paulo. Conhecido por tentar criar um bloco carnavalesco chamado \textit{Porão do Dops}, proibido pela justiça por ``apologia à tortura''\footnote{Veja mais sobre o bloco e sobre Douglas em \url{https://www1.folha.uol.com.br/poder/2018/10/direita-avanca-nas-periferias-a-reboque-do-conservadorismo-da-favela.shtml}. Último acesso em 22/02/2021.}, Douglas também criou controvérsia ao divulgar um documento que listava supostos antifascistas, sendo condenado a pagar, até o fechamento desse trabalho, duas indenizações a pessoas que constavam no suposto dossiê\footnote{Para mais informações do caso e das condenações, ver \url{https://revistaforum.com.br/noticias/douglas-garcia-aliado-de-bolsonaro-sofre-nova-condenacao-por-dossie-sobre-antifascistas/}. Último acesso em 22/02/2021.}. Alinhado com Bolsonaro, foi um dos oito deputados investigados no inquérito das \textit{fake news}\footnote{Veja mais sobre o inquérito e os deputados envolvidos em  \url{https://www.aosfatos.org/noticias/deputados-investigados-por-fake-news-publicam-dois-tweets-criticos-ao-stf-por-dia-em-tres-meses/}. Último acesso em 22/02/2021.}. Douglas se declara contrário à ``ideologia de gênero'', à descriminalização do aborto, ao movimento feminista e a favor do Escola sem Partido. 
 
 Em 2017, ainda estudante, Douglas Garcia fez um discurso na Câmara dos Deputados, ao lado de Fernanda Salles (ver \ref{salles}), onde afirmou: ``Gênero, de acordo com a biologia, é XX e XY. (...) Quando Platão escreveu A República, ele disse que o homem e a mulher têm naturezas diferentes e que, por possuírem naturezas diferentes, um tem determinada propensão a executar determinada função com maior facilidade do que o outro. (...) Costumam dizer que não existe mulher dentro das faculdades de exatas por causa disso, por causa daquilo. Não! As mulheres têm talento sim para exercer funções que o homem não tem, e vice-versa''\footnote{Discurso completo disponível em \url{https://www.camara.leg.br/internet/sitaqweb/TextoHTML.asp?etapa=11&nuSessao=0075/17}. Último acesso em 20/02/2021.}.
 
 \begin{figure}[!htbp]
    \centering
    \includegraphics[scale=0.9]{dg_1.png}
    \caption{Postagem no Twitter de Douglas Garcia. Disponível em \url{https://twitter.com/DouglasGarcia/status/1103989634990772224}. Último acesso em 22/02/2021.}
 \end{figure}
 
 \begin{figure}[!htbp]
    \centering
    \includegraphics[scale=0.8]{so_1.png}
    \caption{Postagem no \textit{instagram} de Dra. Soraya Manato. Disponível em \url{https://www.instagram.com/tv/CHJi6rODUQp/}. Último acesso em 22/02/2021.}
 \end{figure}
 
 \item Dra. Soraya Manato - @DraManato
 
 Política, médica ginecologista e obstetra, Soraya Manato\footnote{\url{https://twitter.com/DraManato}. Último acesso em 20/02/2021.} é atual deputada federal pelo Espírito Santo. Contrária à descriminalização do aborto, à inclusão de mulheres na política por meio de reserva de cadeiras no parlamento\footnote{Veja formulário respondido pelas deputadas federais eleitas em 2018 --- disponível em \url{https://arte.estadao.com.br/focas/capitu/materia/entre-as-deputadas-federais-eleitas-consenso-so-no-que-ja-e-consenso}. Último acesso em 22/02/2021.} e à ``ideologia de gênero'', Soraya é alinhada ao governo Bolsonaro e se envolveu em controvérsias após acessar de maneira duvidosa o laudo médico de uma criança que engravidou após ser estuprada. A parlamentar se colocou contrária ao aborto legal que a menina obteve\footnote{Veja mais sobre o caso e sobre o posicionamento da deputada em \url{https://www.metropoles.com/brasil/justica/menina-estuprada-deputada-do-psl-admite-ter-recebido-laudo-pelo-whatsapp}. Último acesso em 22/02/2021.}.
 
 \item Edir Macedo - @BispoMacedo\label{macedo}
 
 Edir Macedo\footnote{\url{https://twitter.com/BispoMacedo}. Último acesso em 20/02/2021.} é bispo evangélico, escritor, teólogo e empresário. Fundador e líder da Igreja Universal do Reino de Deus (IURD) e proprietário do Grupo Record e RecordTV, o bispo acumula uma fortuna avaliada em mais de 1 bilhão de dólares. Figuras bastante controversas em relação às questões feministas, tanto o bispo quanto a própria IURD não se posicionam contrários à descriminalização do aborto, mas mantêm outras várias características típicas de antifeminismo. O bispo \textit{aceita} pessoas das mais diferentes orientações sexuais e identificações de gênero na IURD, supostamente\footnote{Conferir posicionamento do próprio Edir Macedo sobre aceitação de homossexuais na IURD em \url{https://www.universal.org/bispo-macedo/post/homossexualismo/}. Ver também outro texto do \textit{blog} do bispo sobre o assunto em \url{https://www.universal.org/bispo-macedo/post/nossos-filhos-nao-vao-virar-gays/}. Último acesso em 22/02/2021.}. Por outro lado, \textit{não existe uma aceitação permanente} da pessoa não cisgênera ou não heterossexual, uma vez que, segundo eles, deve-se aceitar o \textit{homossexual}, não a \textit{homossexualidade}. Além disso, compreendem que homens e mulheres são biologicamente distintos e que cada um tem seu papel, tanto na sociedade quanto no único casamento possível para eles --- o cisgênero e heterossexual. Não é por acaso que no \textit{site} da IURD, em especial no espaço de \textit{blog} do próprio Edir Macedo, existem vários \textit{testemunhos} de supostos ex-homossexuais ou ex-travestis\footnote{Exemplos de testemunhos em \url{https://www.universal.org/bispo-macedo/post/libertas-do-homossexualismo/} e em \url{https://www.universal.org/bispo-macedo/post/historia-de-um-ex-travesti/}. Último acesso em 22/02/2021.}. 
 
 Há também, dentro da igreja, uma divisão baseada em biologia que define em quais espaços físicos e em quais funções eclesiásticas pode cada sexo transitar\footnote{Ver Martinez (\citeyear{martinez2018}, p. 6). Último acesso em 22/02/2021.}. Entre os papéis de gênero femininos esperados ou, em alguns casos, exigidos pela IURD e pelo bispo estão o da submissão e da obediência\footnote{Veja matéria completa, que comenta expectativas em relação às mulheres evangélicas da IURD em \url{https://exame.com/brasil/obediencia-e-submissao-o-que-se-espera-das-mulheres-evangelicas-no-brasil/}. Último acesso em 22/02/2021.}. Em um de seus cultos que ficaram mais conhecidos nas redes sociais, o bispo comenta que não permitiu que as filhas prosseguissem os estudos além do ensino médio antes de casarem, para não correrem o risco de se tornarem a ``cabeça da família''\footnote{Ver \url{https://www.correiobraziliense.com.br/app/noticia/brasil/2019/09/24/interna-brasil,789307/bispo-edir-macedo-diz-que-mulher-nao-pode-ter-mais-estudo-que-o-marido.shtml}. Último acesso em 23/02/2021.}. A igreja de Edir Macedo ainda se posiciona contrária à ``ideologia de gênero''\footnote{Veja matéria completa da IURD em \url{https://www.universal.org/noticias/post/pais-cuidado-ideologia-de-genero-chegou-aos-desenhos-animados/}. Último acesso em 23/02/2021.}.
 
 Assim, pela crença de que a biologia é a única responsável pela diferença sexual, pela oposição ao casamento igualitário homossexual, por acreditar que qualquer orientação sexual diversa ao heterossexualismo seja incorreta, pela contrariedade às \textit{filosofias do feminismo}\footnote{Posicionamento de uma das maiores representantes da comunidade feminina da IURD e filha de Edir Macedo, Cristiane, está disponível em Souza (\citeyear{souza2017}, p. 31).}, pela ideia de que o interesse primário da mulher é o lar, que é sua responsabilidade cuidar do marido e dos filhos, e que sua vocação é a maternidade, Edir Macedo foi enquadrado, no escopo desse trabalho, no grupo antifeminista. É interessante saber, porém, que o discurso de sua igreja é considerado muitas vezes politicamente liberal em relação ao meio evangélico, e que há relações complexas em relação a mulheres autodeclaradas feministas dentro da igreja. O trabalho da pesquisadora Jacqueline Teixeira é bastante recomendado e esmiúça de maneira muito rica o assunto\footnote{Uma entrevista com ela sobre o tema pode ser encontrado em \url{https://brasil.elpais.com/brasil/2019/05/11/politica/1557527356_335349.html}. Último acesso em 23/02/2021.}.
  
 \item Edson Salomão - @edsonsalomaosp
 
 Presidente do Movimento Conservador (o mesmo de \ref{douglasgarcia}) e \textit{youtuber}, Edson Salomão\footnote{\url{https://twitter.com/edsonsalomaosp}. Último acesso em 20/02/2021.} também foi chefe de gabinete de Douglas Garcia (\ref{douglasgarcia}), mas se afastou para concorrer, sem sucesso, ao cargo de vereador em São Paulo. Com histórico de controvérsias, Edson já foi investigado no inquérito de \textit{fake news} no STF em 2020. De acordo com o Estatuto do Movimento Conservador\footnote{Disponível em \url{https://www.movimentoconservador.com/missao-visao-e-valores/}. Último acesso em 23/02/2021.} que preside, os membros são contrários à descriminalização do aborto, às ``ideologias de gênero'' e à doutrinação ideológica em ambiente estudantil (sendo próximos do Escola sem partido).
 
 \begin{figure}[!htbp]
    \centering
    \includegraphics[scale=0.8]{es_1.png}
    \caption{Postagem no Twitter de Edson Salomão. Disponível em \url{https://twitter.com/edsonsalomaosp/status/1295469349667590144}. Último acesso em 23/02/2021.}
 \end{figure}
 
 \newpage
  
 \item Eduardo Bolsonaro - @BolsonaroSP
 
 Advogado, policial federal e político, Eduardo Bolsonaro\footnote{\url{https://twitter.com/BolsonaroSP}. Último acesso em 20/02/2021.} é filho do atual presidente Jair Bolsonaro (ver \ref{jbols}), e atual deputado federal por São Paulo, eleito em 2018 com a maior votação da história do país. Eduardo é contrário ao casamento igualitário homossexual\footnote{Veja declaração completa dada em entrevista a uma televisão israelita \url{https://www.sabado.pt/mundo/detalhe/filho-de-bolsonaro-compara-casamento-gay-a-relacao-entre-um-homem-e-um-cao}. Último acesso em 23/02/2021.}; ao movimento feminista, suas pautas e militantes; à ``ideologia de gênero'', e é a favor do Escola sem partido.
 
 \begin{figure}[!htbp]
    \centering
    \includegraphics[scale=0.6]{eb_1.png}
    \caption{Postagem no Twitter de Eduardo Bolsonaro. Apesar de o vídeo focar na agitação do deputado, o suposto ``barraco'' que ele se refere não é esse, mas a suposta movimentação de deputadas de oposição que aparecem apenas no primeiro segundo do vídeo. Disponível em \url{https://twitter.com/BolsonaroSP/status/1229934375854366721}. Último acesso em 23/02/2021.}
 \end{figure}
  
 \item Ernesto Araújo - @ernestofaraujo
 
 Escritor e atual Ministro das Relações Exteriores, Ernesto Araújo\footnote{\url{https://twitter.com/ernestofaraujo}. Último acesso em 20/02/2021.} seguiu carreira diplomática após preparação no Instituto Rio Branco. Declarado contra o que chama de ``climatismo'' e negacionista do aquecimento global, Ernesto também se coloca contra o ``marxismo cultural'', a descriminalização do aborto e a ``ideologia de gênero''.
 
 \begin{figure}[!htbp]
    \centering
    \includegraphics[scale=0.9]{ea_1.png}
    \caption{Postagem no Twitter de Ernesto Araújo. Disponível em \url{https://twitter.com/ernestofaraujo/status/1159301828334575616}. Último acesso em 23/02/2021.}
 \end{figure}
 
 \newpage
 
 \item Fabiana Barroso - @fabifbbr
 
 Advogada especializada em Direito Tributário pela Pontifícia Universidade Católica de São Paulo (PUC-SP), consultora e membra do Articulação Conservadora, Iluminadas e Movimento Avança Brasil (ver \ref{mab}), Fabiana Barroso\footnote{\url{https://twitter.com/fabifbbr}. Último acesso em 20/02/2021.} também é \textit{youtuber} e comentarista política, com participação em jornais como Brasil Sem Medo (ver \ref{bsm}). Em suas redes sociais, já se posicionou contrariamente ou criticou a descriminalização do aborto, o feminismo, a ``ideologia de gênero'', e o ``kit gay'', além de se colocar favorável ao programa Escola sem partido.
 
 \begin{figure}[!htbp]
    \centering
    \includegraphics[scale=0.9]{fb_1.png}
    \caption{Postagem no Twitter de Fabiana Barroso. Disponível em \url{https://twitter.com/fabifbbr/status/1044942826168090624}. Último acesso em 23/02/2021.}
 \end{figure}
 
 \item Felipe C. Pedri - @FelipePedri
 
 Felipe Pedri \footnote{\url{https://twitter.com/FelipePedri}. Último acesso em 20/02/2021.} é atual secretário de Comunicação Institucional do governo Bolsonaro. Um dos fundadores do Aliança pelo Brasil (ver \ref{alianca}), Felipe foi um dos responsáveis pela confecção do manifesto de fundação do mesmo. Abertamente contrário ao feminismo, Felipe também se declara contrário à ``ideologia de gênero'' e à descriminalização do aborto em qualquer circunstância.
 
 \begin{figure}[!htbp]
    \centering
    \includegraphics[scale=0.9]{fp_1.png}
    \caption{Postagem no Twitter de Felipe C. Pedri. Disponível em \url{https://twitter.com/FelipePedri/status/1237808492276301824}. Último acesso em 23/02/2021.}
 \end{figure}
 
 \newpage
  
 \item Felipe Francischini - @FFrancischini$\_$
 
 Advogado e político, Felipe Francischini\footnote{\url{https://twitter.com/FFrancischini_}. Último acesso em 20/02/2021.} exerce atualmente o cargo de deputado federal pelo Paraná. Já foi autor de projeto de lei que visava implementar o programa Escola sem partido em 2016, enquanto era deputado estadual pelo Paraná. Recusado pela câmara, o projeto valorizava a identidade biológica de sexo e vetava o que o texto chamava de ``aplicação da ideologia de gênero''\footnote{Uma matéria completa sobre o episódio está disponível em \url{https://bandnewsfmcuritiba.com/deputados-estaduais-do-parana-rejeitam-escola-sem-partido/}. Último acesso em 23/02/2021.}. Abertamente contra a descriminalização do aborto, o deputado compõe a Frente Parlamentar Mista contra o Aborto e em Defesa da Vida.
  
 \item Fernanda Barth vereadora - @fernandabarth
 
 Mestre em Ciência Política e Jornalista, Fernanda Barth\footnote{\url{https://twitter.com/fernandabarth}. Último acesso em 20/02/2021.} é atualmente vereadora em Porto Alegre. Líder do Movimento Avança Brasil (ver \ref{mab}) no Rio Grande do Sul, participante do movimento Livre Iniciativa para Todos e do Acorda Brasil, Fernanda é também presidente do grupo Mulheres de Direita de Porto Alegre. Declaradamente contrária à descriminalização do aborto, à ``ideologia de gênero'' e a favor do programa Escola sem partido.
 
 \begin{figure}[!htbp]
    \centering
    \includegraphics[scale=0.75]{fba_1.png}
    \caption{Postagem no Twitter de Fernanda Barth. Disponível em \url{https://twitter.com/fernandabarth/status/1293748223069257728}. Último acesso em 23/02/2021.}
 \end{figure}
  
  \newpage
  
 \item Fernanda Brum - @PraFeBrum
 
 Cantora de música gospel, Fernanda Brum\footnote{\url{https://twitter.com/PraFeBrum}. Último acesso em 20/02/2021.} é também escritora, atriz, compositora, produtora e pastora evangélica. Em um de seus discos, contou com a participação de Ana Paula Valadão (\ref{anapaulav}). Contrária à descriminalização do aborto e praticante da máxima ``feminina, não feminista'', Fernanda também já se posicionou contrariamente à ``ideologia de gênero''\footnote{Veja notícia de entrevista de Fernanda em rádio disponível em \url{https://noticias.gospelmais.com.br/avon-flordelis-fernanda-brum-ideologia-de-genero-93485.html}. Último acesso em 23/02/2021.}. Em outro posicionamento, se referindo a homossexuais, comentou que ``o que sara é o amor e o diálogo (...) Não é repudiar''\footnote{O vídeo com entrevista completa com a pastora está disponível em \url{https://noticias.gospelmais.com.br/fernanda-brum-papel-igreja-receber-homossexuais-ouvi-los-87465.html}. Último acesso em 23/02/2021.}.
  
 \item Fernanda Salles - @reportersalles\label{salles}
 
 Jornalista, \textit{youtuber}, repórter, assessora de imprensa e colaboradora de jornais como o Terça Livre, Fernanda Salles\footnote{\url{https://twitter.com/reportersalles}. Último acesso em 20/02/2021.} é declaradamente contra a ``ideologia de gênero'', a descriminalização do aborto, o movimento feminista, e favorável ao projeto Escola sem Partido. Em 2019, foi apontada como assessora do deputado Bruno Engler\footnote{Veja notícia disponível em \url{https://politica.estadao.com.br/noticias/geral,fernanda-salles-que-assina-texto-com-dados-falsos-sobre-reporter-do-estado-trabalha-para-o-psl,70002751019}. Último acesso em 25/02/2021.} (ver \ref{engler}).
 
 \begin{figure}[!htbp]
    \centering
    \includegraphics[scale=0.7]{fs_4.png}
    \caption{Postagem no Twitter de Fernanda Salles. Disponível em \url{https://twitter.com/reportersalles/status/1179506520779870208}. Último acesso em 23/02/2021.}
 \end{figure}
 
 \begin{figure}[!htbp]
    \centering
    \includegraphics[scale=0.9]{fs_1.png}
    \caption{Postagem no Twitter de Fernanda Salles. Disponível em \url{https://twitter.com/reportersalles/status/1186645698109956097}. Último acesso em 23/02/2021.}
 \end{figure}
 
%  \begin{figure}[!htbp]
%     \centering
%     \includegraphics[scale=0.9]{fs_2.png}
%     \caption{Postagem no Twitter de Fernanda Salles. Disponível em \url{https://twitter.com/reportersalles/status/1286341377106890757}. Último acesso em 23/02/2021.}
%  \end{figure}
 
%  \begin{figure}[!htbp]
%     \centering
%     \includegraphics[scale=0.9]{fs_3.png}
%     \caption{Postagem no Twitter de Fernanda Salles. Disponível em \url{https://twitter.com/reportersalles/status/1180496214103662593}. Último acesso em 23/02/2021.}
%  \end{figure}
 
 \newpage
  
 \item Fernando Holiday - @FernandoHoliday
 
 Estudante de história, \textit{youtuber} e político, Fernando Holiday\footnote{\url{https://twitter.com/FernandoHoliday}. Último acesso em 20/02/2021.} exerce atualmente o cargo de vereador por São Paulo. Ex-integrante do Movimento Brasil Livre, o vereador é assumidamente gay e a favor do casamento igualitário para homossexuais. Por outro lado, é contrário à descriminalização do aborto, à ``ideologia de gênero'', ao movimento feminista e ao feminismo ``atual'', e favorável ao projeto Escola sem partido.
 
%  \begin{figure}[!htbp]
%     \centering
%     \includegraphics[scale=0.8]{fh_1.png}
%     \caption{Postagem no Twitter de Fernando Holiday. Disponível em \url{https://twitter.com/FernandoHoliday/status/1220413392416055302}. Último acesso em 23/02/2021.}
%  \end{figure}
 
 \begin{figure}[!htbp]
    \centering
    \includegraphics[scale=0.6]{fh_2.png}
    \caption{Postagem no Twitter de Fernando Holiday. Disponível em \url{https://twitter.com/FernandoHoliday/status/918265588815945728}. Último acesso em 23/02/2021.}
 \end{figure}
 
 \begin{figure}[!htbp]
    \centering
    \includegraphics[scale=0.8]{fh_3.png}
    \caption{Postagem no Twitter de Fernando Holiday. Disponível em \url{https://twitter.com/FernandoHoliday/status/839557274213486595}. Último acesso em 23/02/2021.}
 \end{figure}
  
  \newpage
  
 \item Filipe Barros - @filipebarrost
 
 Advogado e político, Filipe Barros\footnote{\url{https://twitter.com/filipebarrost}. Último acesso em 20/02/2021.} atualmente exerce cargo de deputado federal pelo Paraná e é aliado ao governo Bolsonaro. Ex-integrante do MBL, saiu do grupo para se integrar ao Partido Social Liberal (PSL), antiga agremiação da qual o atual presidente Bolsonaro fazia parte. É posicionado contra a ``ideologia de gênero'', a descriminalização do aborto e o feminismo, e é favorável ao programa Escola sem partido. 
 
 \begin{figure}[!htbp]
    \centering
    \includegraphics[scale=0.85]{fib_1.png}
    \caption{Postagem no Twitter de Filipe Barros. Disponível em \url{https://twitter.com/filipebarrost/status/1080644179208876032}. Último acesso em 23/02/2021.}
 \end{figure}
 
 \begin{figure}[!htbp]
    \centering
    \includegraphics[scale=0.85]{fib_2.png}
    \caption{Postagem no Twitter de Filipe Barros. Disponível em \url{https://twitter.com/filipebarrost/status/1210702261837582338}. Último acesso em 23/02/2021.}
 \end{figure}
  
 \item Filipe G. Martins - @filgmartin
 
 Professor e analista político, Filipe G. Martins\footnote{\url{https://twitter.com/filgmartin}. Último acesso em 20/02/2021.} é atual Assessor Especial para Assuntos Internacionais do governo Bolsonaro. Conservador, se declara contrário à ``ideologia de gênero'', à descriminalização do aborto e ao feminismo, e favorável ao programa Escola sem partido.
 
 \begin{figure}[!htbp]
    \centering
    \includegraphics[scale=0.85]{fig_1.png}
    \caption{Postagem no Twitter de Filipe G. Martins. Disponível em \url{https://twitter.com/filgmartin/status/1031195934376513542}. Último acesso em 23/02/2021.}
 \end{figure}
 
 \begin{figure}[!htbp]
    \centering
    \includegraphics[scale=0.85]{fig_2.png}
    \caption{Postagem no Twitter de Filipe G. Martins. Disponível em \url{https://twitter.com/filgmartin/status/1040642485196914688}. Último acesso em 23/02/2021.}
 \end{figure}
 
   \newpage

 \item Flavio Bolsonaro - @FlavioBolsonaro
 
 Advogado, empresário, político e filho do atual presidente Jair Bolsonaro (ver \ref{jbols}), Flavio Bolsonaro\footnote{\url{https://twitter.com/FlavioBolsonaro}. Último acesso em 20/02/2021.} atualmente exerce o cargo de senador pelo Rio de Janeiro. Envolvido em diversas controvérsias, já afirmou que ``O normal não é ser homossexual. O normal é ser heterossexual. Duvido que algum pai tenha orgulho de ter um filho gay''.\footnote{Matéria e fala completa disponíveis em  \url{https://politica.estadao.com.br/noticias/geral,para-filhos-bolsonaro-diz-o-que-a-maioria-pensa,700618}. Último acesso em 23/02/2021.} Contrário ao ``kit gay'', à ``ideologia de gênero'', à descriminalização do aborto, ao movimento feminista e favorável ao programa Escola sem partido.
 
%  \begin{figure}[!htbp]
%     \centering
%     \includegraphics[scale=0.8]{flb_1.png}
%     \caption{Postagem no Twitter de Flavio Bolsonaro. Disponível em \url{https://twitter.com/FlavioBolsonaro/status/73461290647035904}. Último acesso em 23/02/2021.}
%  \end{figure}
 
 \begin{figure}[!htbp]
    \centering
    \includegraphics[scale=0.8]{flb_2.png}
    \caption{Postagem no Twitter de Flavio Bolsonaro. Disponível em \url{https://twitter.com/FlavioBolsonaro/status/933311306068905984}. Último acesso em 23/02/2021.}
 \end{figure}
 
 \begin{figure}[!htbp]
    \centering
    \includegraphics[scale=0.8]{flb_3.png}
    \caption{Postagem no Twitter de Flavio Bolsonaro. Disponível em \url{https://twitter.com/FlavioBolsonaro/status/1007747149126361089}. Último acesso em 23/02/2021.}
 \end{figure}
  
  \newpage
  
 \item Gil Diniz - @carteiroreaca
 
 Carteiro e político, Gil Diniz\footnote{\url{https://twitter.com/carteiroreaca}. Último acesso em 20/02/2021.} ficou conhecido como ``Carteiro Reaça'' ainda quando exercia a profissão de carteiro, por conta de seus posicionamentos políticos. Atualmente, Gil Diniz é deputado estadual por São Paulo. Se declara contra a ``ideologia de gênero'', o feminismo, o ``kit gay'' e defende o programa Escola sem partido.
 
 \begin{figure}[!htbp]
    \centering
    \includegraphics[scale=0.55]{gil_3.png}
    \caption{Postagem no Twitter de Gil Diniz. Disponível em \url{https://twitter.com/carteiroreaca/status/704167614117974016}. Último acesso em 24/02/2021.}
 \end{figure}
  
 \item Heitor Freire - @HeitorFreireCE
 
 Heitor Freire\footnote{\url{https://twitter.com/HeitorFreireCE}. Último acesso em 20/02/2021.} é administrador e político, exercendo atualmente o cargo de deputado federal pelo Ceará. Favorável ao Escola sem partido, Heitor Freire se declara contrário à ``ideologia de gênero'' e à descriminalização do aborto, além de ter votado contra a inclusão de políticas para a comunidade LGBTI no Ministério da Mulher, da Família e dos Direitos Humanos. Essa decisão faria com que, na teoria, nenhuma das pastas ministeriais tivesse políticas para comunidade LGBTI entre suas competências\footnote{Veja mais sobre a decisão e a proposta em \url{https://revistahibrida.com.br/2019/10/17/248-deputados-votaram-contra-politicas-lgbt-na-pasta-de-direitos-humanos/}. Último acesso em 24/02/2021.}.
 
 \begin{figure}[!htbp]
    \centering
    \includegraphics[scale=0.85]{hfr_1.png}
    \caption{Postagem no Twitter de Heitor Freire. Disponível em \url{https://twitter.com/HeitorFreireCE/status/1255232145372700675}. Último acesso em 24/02/2021.}
 \end{figure}
 
 \item Helio Lopes - @depheliolopes
 
 Militar e político, Helio Lopes\footnote{\url{https://twitter.com/depheliolopes}. Último acesso em 20/02/2021.} é também formado em gestão pública e financeira e exerce atualmente o cargo de deputado federal pelo Rio de Janeiro. Apoiador do governo Bolsonaro, Hélio se declara contrário à ``ideologia de gênero'', à descriminalização do aborto e às feministas, e favorável ao escola sem partido.
 
 \begin{figure}[!htbp]
    \centering
    \includegraphics[scale=0.85]{hel_1.png}
    \caption{Postagem no Twitter de Helio Lopes. Disponível em \url{https://twitter.com/depheliolopes/status/1323356376169631744}. Último acesso em 24/02/2021.}
 \end{figure}
  
 \item Italo Lorenzon - @LorenzonItalo
 
 Analista político, Italo Lorenzon\footnote{\url{https://twitter.com/LorenzonItalo}. Último acesso em 20/02/2021.} é também apresentador e co-fundador do Terça Livre. É favorável ao programa Escola sem partido e contrário ao feminismo, à ``ideologia de gênero'' e também à descriminalização do aborto.
 
 \begin{figure}[!htbp]
    \centering
    \includegraphics[scale=0.85]{il_1.png}
    \caption{Postagem no Twitter de Italo Lorenzon. Disponível em \url{https://twitter.com/LorenzonItalo/status/1193200004585984001}. Último acesso em 24/02/2021.}
 \end{figure}
  
 \item Jair M. Bolsonaro - @jairbolsonaro\label{jbols}
 
 Militar e político inveterado, Jair Bolsonaro\footnote{\url{https://twitter.com/jairbolsonaro}. Último acesso em 20/02/2021.} é atual presidente do Brasil, depois de ter exercido sete mandatos como deputado federal. Envolvido em diversas controvérsias, que vão desde a defesa da tortura e da ditadura militar\footnote{Veja participação de Jair Bolsonaro em manifestação que pedia intervenção militar, em 2020, durante a pandemia de COVID-19, disponível em \url{https://www.youtube.com/watch?v=O5ZWNM3rkpA}. Em outra entrevista mais antiga, de 1999, Bolsonaro já defendia a tortura e o fechamento do congresso. Disponível em \url{http://www.diariodecuiaba.com.br/video-tv/bolsonaro-fala-em-golpes-e-fechar-congresso-desde-anos-90/530225}. Último acesso em 24/02/2021.}, até a contrariedade às conquistas da comunidade LGBTQIA+ e das mulheres. 
 
 Em relação às pessoas LGBTQIA+, Jair Bolsonaro já condenou a homossexualidade em várias situações\footnote{Em entrevista à revista masculina Playboy, afirmou que ``prefiro que um filho meu morra num acidente do que apareça com um bigodudo por aí''. Disponível em \url{http://noticias.terra.com.br/brasil/bolsonaro-quotprefiro-filho-morto-em-acidente-a-um-homossexualquot,cf89cc00a90ea310VgnCLD200000bbcceb0aRCRD.html}. Último acesso em 24/02/2021.}, se declarou contrário ao casamento homossexual igualitário, à adoção de filhos por casais homossexuais\footnote{Sobre posicionamento de Jair Bolsonaro em relação ao casamento igualitário e à adoção por casal homossexual, veja \url{http://noticias.terra.com.br/brasil/politica/bolsonaro-sobre-casamento-gay-nao-querem-igualdade-e-sim-privilegios,99ff52d635aae310VgnVCM4000009bcceb0aRCRD.html}. Último acesso em 24/02/2021.} e, ao se colocar contra a suposta ``ideologia de gênero'', afirma a máxima ``não se torna mulher (ou homem), nasce-se''\footnote{Veja matéria completa sobre declaração do presidente de que ``ou se nasce homem, ou se nasce mulher'' em \url{https://www.terra.com.br/noticias/brasil/politica/bolsonaro-inaugura-no-rj-colegio-para-filhos-de-pms-e-critica-ideologia-de-genero,eab0e2453a644bd0703ef7695768962593y66dr1.html}. Último acesso em 24/02/2021.}.
 
 Em relação às mulheres, além das famosas afirmações de que uma deputada não merecia ser estuprada porque seria supostamente ``muito feia''\footnote{Veja mais sobre o assunto em \url{http://g1.globo.com/politica/noticia/2016/06/bolsonaro-vira-reu-por-falar-que-maria-do-rosario-nao-merece-ser-estuprada.html}. Último acesso em 24/02/2021.} e de que sua quinta filha seria fruto de uma ``fraquejada'' após quatro filhos homens\footnote{Veja matéria completa em \url{https://revistaforum.com.br/noticias/bolsonaro-eu-tenho-5-filhos-foram-4-homens-a-quinta-eu-dei-uma-fraquejada-e-veio-uma-mulher-3/}. Último acesso em 24/02/2021.}, Jair Bolsonaro também já justificou em entrevista que mulher recebe salário menor porque engravida\footnote{Veja matéria completa em \url{https://revistacrescer.globo.com/Familia/Maes-e-Trabalho/noticia/2015/02/jair-bolsonaro-diz-que-mulher-deve-ganhar-salario-menor-porque-engravida.html}. Último acesso em 24/02/2021.}. Contrário à descriminalização do aborto, também diz não saber e não se importar com que quer o movimento feminista, um movimento de ``mulher com braço cabeludo''\footnote{Ver entrevista ao jornal O Globo em \url{https://oglobo.globo.com/brasil/2018/07/21/3046-nao-entendo-mesmo-de-economia-diz-bolsonaro-ao-globo}. Último acesso em 24/02/2021.}.
  
 \item JOSÉ MEDEIROS - @JoseMedeirosMT
 
 Policial rodoviário, professor e político, José Medeiros\footnote{\url{https://twitter.com/JoseMedeirosMT}. Último acesso em 20/02/2021.} é atualmente deputado federal pelo Mato Grosso e vice-líder do governo Bolsonaro. Se declara contrário à descriminalização do aborto e à ``ideologia de gênero'' e favorável ao programa Escola sem partido.
 
 \begin{figure}[!htbp]
    \centering
    \includegraphics[scale=0.85]{jm_1.png}
    \caption{Postagem no Twitter de JOSÉ MEDEIROS. Disponível em \url{https://twitter.com/JoseMedeirosMT/status/1355709706917056515}. Último acesso em 24/02/2021.}
 \end{figure}
 
%  \begin{figure}[!htbp]
%     \centering
%     \includegraphics[scale=0.85]{jm_2.png}
%     \caption{Postagem no Twitter de Italo Lorenzon. Disponível em \url{https://twitter.com/LorenzonItalo/status/1193200004585984001}. Último acesso em 24/02/2021.}
%  \end{figure}
 
  \item Junio Amaral - @cabojunioamaral
  
  Policial militar reformado e político, Junio Amaral\footnote{\url{https://twitter.com/cabojunioamaral}. Último acesso em 20/02/2021.} atualmente exerce o cargo de deputado federal por Minas Gerais e é alinhado ao governo Bolsonaro. Contrário à descriminalização do aborto e à ``ideologia de gênero'', é favorável ao programa Escola sem partido.
 
 \begin{figure}[!htbp]
    \centering
    \includegraphics[scale=0.8]{jua_1.png}
    \caption{Postagem no Twitter de Junio Amaral. Disponível em \url{https://twitter.com/cabojunioamaral/status/1240103163824406528}. Último acesso em 24/02/2021.}
 \end{figure}
 
%  \begin{figure}[!htbp]
%     \centering
%     \includegraphics[scale=0.85]{jua_2.png}
%     \caption{Postagem no Twitter de Junio Amaral. Disponível em \url{https://twitter.com/cabojunioamaral/status/1240103163824406528}. Último acesso em 24/02/2021.}
%  \end{figure}
  
 \item Kim Kataguiri - @KimKataguiri
 
 Ex-colunista do jornal Folha de São Paulo e do \textit{The Huffington Post} Brasil, Kim Kataguiri\footnote{\url{https://twitter.com/KimKataguiri}. Último acesso em 20/02/2021.} é mais conhecido por ser cofundador e coordenador do Movimento Brasil Livre. Atualmente, é deputado federal por São Paulo. Contrário à ``ideologia de gênero'' e à descriminalização do aborto, acredita que a biologia define o gênero e é favorável ao programa Escola sem partido. Costuma fazer postagens em suas redes ironizando feministas e o próprio feminismo.
 
 \begin{figure}[!htbp]
    \centering
    \includegraphics[scale=0.8]{kk_1.png}
    \caption{Postagem no Facebook de Kim Kataguiri. Disponível em \url{https://www.facebook.com/kataguiri.kim/posts/1846740348710489}. Último acesso em 24/02/2021.}
 \end{figure}
 
 \begin{figure}[!htbp]
    \centering
    \includegraphics[scale=0.8]{kk_2.png}
    \caption{Postagem no Facebook de Kim Kataguiri. Disponível em \url{https://www.facebook.com/833053646745836/videos/421391445325536}. Último acesso em 24/02/2021.}
 \end{figure}
 
 \item Leandro Ruschel - @leandroruschel
 
 Empreendedor e especialista em investimentos, Leandro Ruschel\footnote{\url{https://twitter.com/leandroruschel}. Último acesso em 20/02/2021.} é também \textit{youtuber} e usa suas redes sociais para falar sobre política. Declarado conservador, é contrário à ``ideologia de gênero'' e à descriminalização do aborto, favorável ao programa Escola sem partido, e acredita que gênero é definido pela biologia.
 
 \begin{figure}[!htbp]
    \centering
    \includegraphics[scale=0.8]{ler_1.png}
    \caption{Postagem no Twitter de Leandro Ruschel. Disponível em \url{https://twitter.com/leandroruschel/status/1359893602713956360}. Último acesso em 24/02/2021.}
 \end{figure}
 
 \item Leticia Aguiar - @letsaguiar
 
 Política, Leticia Aguiar\footnote{\url{https://twitter.com/letsaguiar}. Último acesso em 20/02/2021.} atualmente exerce o cargo de deputada estadual por São Paulo. Contrária ao feminismo ``atual'', à ``ideologia de gênero'', ao ``marxismo cultural'' e à descriminalização do aborto, é a favor do programa Escola sem partido e alinhada com o governo Bolsonaro.
 
%  \begin{figure}[!htbp]
%     \centering
%     \includegraphics[scale=0.8]{lets_1.png}
%     \caption{Postagem no Twitter de Leticia Aguiar. Disponível em \url{https://twitter.com/leandroruschel/status/1359893602713956360}. Último acesso em 24/02/2021.}
%  \end{figure}
 
 \begin{figure}[!htbp]
    \centering
    \includegraphics[scale=0.8]{lets_2.png}
    \caption{Postagem no Twitter de Leticia Aguiar. Disponível em \url{https://twitter.com/letsaguiar/status/1344400991073038336}. Último acesso em 24/02/2021.}
 \end{figure}
 
 \begin{figure}[!htbp]
    \centering
    \includegraphics[scale=0.8]{lets_3.png}
    \caption{Postagem no Twitter de Leticia Aguiar. Disponível em \url{https://twitter.com/letsaguiar/status/1187804120289296385}. Último acesso em 24/02/2021.}
 \end{figure}

 \item Liomar de Oliveira - @PastorLiomar
 
 Político e pastor evangélico, Liomar de Oliveira\footnote{\url{https://twitter.com/PastorLiomar}. Último acesso em 20/02/2021.} é atualmente suplente de vereador no Rio de Janeiro. Coordenador estadual do Movimento Brasil Conservador (MBC) e apoiador de Bolsonaro, Liomar segue agenda conservadora. Contrário à ``militância gay'', à ``ideologia de gênero'', à descriminalização do aborto e ao casamento igualitário homossexual, Liomar é favorável ao programa Escola sem partido.
 
 \begin{figure}[!htbp]
    \centering
    \includegraphics[scale=0.8]{lio_4.png}
    \caption{Postagem no Twitter de Liomar de Oliveira. Disponível em \url{https://twitter.com/PastorLiomar/status/1015770683937316864}. Último acesso em 24/02/2021.}
 \end{figure}
 
 \newpage

 \item LUIZ CAMARGO vlog - @LuizCamargoVlog
 
 \textit{Youtuber}, Luiz Camargo\footnote{\url{https://twitter.com/LuizCamargoVlog}. Último acesso em 20/02/2021.} usa seu canal e suas redes sociais para comentar sobre política e espiritualidade. Se posiciona contrariamente ao feminismo e às feministas, ironizando-as em postagens de seu Twitter. Também se coloca contrário à descriminalização do aborto e defende que ``casamento gay não existe''.
 
 \begin{figure}[!htbp]
    \centering
    \includegraphics[scale=0.8]{luic_1.png}
    \caption{Postagem no Twitter de LUIZ CAMARGO vlog. Disponível em \url{https://twitter.com/LuizCamargoVlog/status/1361401079100043264}. Último acesso em 24/02/2021.}
 \end{figure}
 
 \newpage
  
 \item magno malta - @MagnoMalta
 
 Magno Malta\footnote{\url{https://twitter.com/MagnoMalta}. Último acesso em 20/02/2021.} é pastor evangélico, cantor e político. Foi senador de 2003 a 2019, não conseguindo vencer a eleição para prosseguir no cargo. Alinhado às pautas conservadoras, é contrário à descriminalização do aborto, à ``ideologia de gênero'', ao casamento igualitário homossexual, à adoção de crianças por homossexuais e favorável ao projeto Escola sem partido.
 
 \begin{figure}[!htbp]
    \centering
    \includegraphics[scale=0.8]{mag_3.png}
    \caption{Postagem no Twitter de Magno Malta, provavelmente se referindo à presidenciável a époa, Marina Silva. Disponível em \url{https://twitter.com/MagnoMalta/status/1030751924260417536}. Último acesso em 24/02/2021.}
 \end{figure}
 
 \begin{figure}[!htbp]
    \centering
    \includegraphics[scale=0.9]{mag_4.png}
    \caption{Postagem no Twitter de Magno Malta. Disponível em \url{https://twitter.com/MagnoMalta/status/956198596533260290}. Último acesso em 24/02/2021.}
 \end{figure}
  
  \newpage
  
 \item Marcel van Hattem - @marcelvanhattem
 
 Marcel van Hattem\footnote{\url{https://twitter.com/marcelvanhattem}. Último acesso em 20/02/2021.} é graduado em relações internacionais, mestre em ciência política e em jornalismo, mídia e globalização, e atualmente exerce o cargo de deputado federal pelo Rio Grande do Sul. Contrário à ``ideologia de gênero'' e à descriminalização do aborto, Marcel é também a favor do programa Escola sem partido.
 
 \begin{figure}[!htbp]
    \centering
    \includegraphics[scale=1]{mvh_1.png}
    \caption{Postagem no Twitter de Marcel van Hattem. Disponível em \url{https://twitter.com/marcelvanhattem/status/1255258275601317889}. Último acesso em 24/02/2021.}
 \end{figure}
 
 \begin{figure}[!htbp]
    \centering
    \includegraphics[scale=0.8]{mvh_2.png}
    \caption{Postagem no Twitter de Marcel van Hattem. Disponível em \url{https://twitter.com/marcelvanhattem/status/613509063121444864}. Último acesso em 24/02/2021.}
 \end{figure}
 
  
 \item Marcelo Crivella - @MCrivella
 
 Político e bispo licenciado pela IURD, Marcelo Crivella\footnote{\url{https://twitter.com/MCrivella}. Último acesso em 20/02/2021.} é também cantor e compositor, e sobrinho de Edir Macedo (ver \ref{macedo}). Envolvido em polêmicas como a ordem de recolhimento de exemplares de um romance gráfico com dois personagens homossexuais na Bienal do Livro\footnote{O episódio completo está descrito em matéria do G1 disponível em \url{https://g1.globo.com/rj/rio-de-janeiro/noticia/2019/09/08/nao-foi-encontrada-nenhuma-violacao-diz-subsecretario-do-rio-apos-fiscalizacao-na-bienal-do-livro.ghtml}. Último acesso em 24/02/2021.}, Marcelo é contrário à descriminalização do aborto, ao casamento igualitário homossexual, ao ``kit gay'' e à ``ideologia de gênero''.
 
 \begin{figure}[!htbp]
    \centering
    \includegraphics[scale=0.7]{mc_1.png}
    \caption{Postagem no Twitter de Marcelo Crivella. Disponível em \url{https://twitter.com/MCrivella/status/1324364397725310976}. Último acesso em 24/02/2021.}
 \end{figure}
  
  \newpage
  
 \item Marcio Labre Oficial - @marciolabre
 
 Político, jornalista, empresário e \textit{youtuber}, Marcio Labre\footnote{\url{https://twitter.com/marciolabre}. Último acesso em 20/02/2021.} é atual deputado federal eleito pelo Rio de Janeiro. Alinhado a Bolsonaro e autodeclarado conservador, Marcio Labre se posiciona contrário à ``ideologia de gênero'', ao aborto, ao feminismo e ao movimento feminista. É partidário do programa Escola sem partido.
 
 \begin{figure}[!htbp]
    \centering
    \includegraphics[scale=0.9]{ml_1.png}
    \caption{Postagem no Twitter de Marcio Labre Oficial. Disponível em \url{https://twitter.com/marciolabre/status/1028808666441568256}. Último acesso em 24/02/2021.}
 \end{figure}
 
 \begin{figure}[!htbp]
    \centering
    \includegraphics[scale=0.9]{ml_2.png}
    \caption{Postagem no Twitter de Marcio Labre Oficial. Disponível em \url{https://twitter.com/marciolabre/status/1044959034686795777}. Último acesso em 24/02/2021.}
 \end{figure}
 
%  \begin{figure}[!htbp]
%     \centering
%     \includegraphics[scale=0.7]{ml_3.png}
%     \caption{Postagem no Twitter de Marcio Labre Oficial. Disponível em \url{https://twitter.com/MCrivella/status/1324364397725310976}. Último acesso em 24/02/2021.}
%  \end{figure}
  
 \item MARISA LOBO-Supl.Dep.Federal - @marisa$\_$lobo
 
 Política e psicóloga, Marisa Lobo\footnote{\url{https://twitter.com/marisa_lobo}. Último acesso em 20/02/2021.} teve seu registro no Conselho Regional de Psicologia (CRP) cassado em 2014 por supostamente aplicar a ``cura gay''. Representada pela advogada Damares Alves, atual ministra do governo Bolsonaro, conseguiu reverter a cassação em 2015. Hoje, considera-se uma ``psicóloga cristã'' e divulga sua religião através de seu Ministério Marisa Lobo\footnote{Veja mais sobre sua atuação em \url{https://psicologiacrista.minhalojanouol.com.br/Paginas/51431/quem_somos}. Último acesso em 24/02/2021.}. É também suplente de deputada federal pelo Paraná. Marisa Lobo é contrária ao feminismo, à ``ideologia de gênero'', à descriminalização do aborto, ao casamento igualitário homossexual e à adoção por casais homossexuais\footnote{Veja publicação feita por Marisa em \url{https://www.facebook.com/photo.php?fbid=10202365182357042&l=64838b2c5a}. Último acesso em 24/02/2021.}.
 
 \begin{figure}[!htbp]
    \centering
    \includegraphics[scale=0.9]{mlo_1.png}
    \caption{Postagem no Twitter de MARISA LOBO-Supl.Dep.Federal. Disponível em \url{https://twitter.com/marisa_lobo/status/951120339316113413}. Último acesso em 24/02/2021.}
 \end{figure}
 
 \begin{figure}[!htbp]
    \centering
    \includegraphics[scale=0.9]{mlo_2.png}
    \caption{Postagem no Twitter de MARISA LOBO-Supl.Dep.Federal. Disponível em \url{https://twitter.com/marisa_lobo/status/1272348658932101121}. Último acesso em 24/02/2021.}
 \end{figure}
 
 \begin{figure}[!htbp]
    \centering
    \includegraphics[scale=0.9]{mlo_3.png}
    \caption{Postagem no Twitter de MARISA LOBO-Supl.Dep.Federal. Disponível em \url{https://twitter.com/marisa_lobo/status/282478911890288640}. Último acesso em 24/02/2021.}
 \end{figure}
 
 \newpage
 
 \item Maurício Costa - @MauricioCostaRS
 
 Maurício Costa\footnote{\url{https://twitter.com/MauricioCostaRS}. Último acesso em 20/02/2021.} é \textit{podcaster} e fundador do Movimento Brasil Conservador. Em suas redes, se posiciona contrário à descriminalização do aborto e à ``ideologia de gênero''.
 
 \begin{figure}[!htbp]
    \centering
    \includegraphics[scale=0.9]{mco_1.png}
    \caption{Postagem no Twitter de Maurício Costa. Disponível em \url{https://twitter.com/MauricioCostaRS/status/1346090946748940294}. Último acesso em 24/02/2021.}
 \end{figure}
 
  \newpage
  
 \item Olavo de Carvalho - @OdeCarvalho
 
 Filósofo autodidata, Olavo de Carvalho\footnote{\url{https://twitter.com/OdeCarvalho}. Último acesso em 20/02/2021.} é também escritor, professor e jornalista, além de ser considerado uma espécie de ``guru'' do atual presidente Bolsonaro\footnote{Veja matéria do Globo que considera Olavo de Carvalho ``guru'' do bolsonarismo; disponível em \url{https://oglobo.globo.com/brasil/guru-do-bolsonarismo-olavo-de-carvalho-orienta-alunos-deixarem-governo-23507185}. Último acesso em 24/02/2021.} --- título que ele recusa\footnote{Ver matéria da revista veja em que Olavo de Carvalho nega ser guru do governo Bolsonaro, disponível em \url{https://veja.abril.com.br/politica/olavo-de-carvalho-eu-sou-o-guru-dessa-porcaria/}. Último acesso em 24/02/2021.}. Em suas redes, Olavo de Carvalho costuma criticar o feminismo e provocar feministas. Já se posicionou contrariamente ao aborto e favoravelmente ao Escola sem Partido, e denunciou a existência do que chamou de ``marxismo cultural''.
 
 \begin{figure}[!htbp]
    \centering
    \includegraphics[scale=0.9]{oc_1.png}
    \caption{Postagem no Twitter de Olavo de Carvalho. Disponível em \url{https://twitter.com/OdeCarvalho/status/639230026995503104}. Último acesso em 24/02/2021.}
 \end{figure}
 
 \begin{figure}[!htbp]
    \centering
    \includegraphics[scale=0.9]{oc_2.png}
    \caption{Postagem no Twitter de Olavo de Carvalho. Disponível em \url{https://twitter.com/OdeCarvalho/status/826866916887257089}. Último acesso em 24/02/2021.}
 \end{figure}
 
 \begin{figure}[!htbp]
    \centering
    \includegraphics[scale=0.9]{oc_4.png}
    \caption{Postagem no Twitter de Olavo de Carvalho. Disponível em \url{https://twitter.com/OdeCarvalho/status/795769595231698945}. Último acesso em 24/02/2021.}
 \end{figure}
 
 \begin{figure}[!htbp]
    \centering
    \includegraphics[scale=0.9]{oc_5.png}
    \caption{Postagem no Twitter de Olavo de Carvalho. Disponível em \url{https://twitter.com/OdeCarvalho/status/853381631461163009}. Último acesso em 24/02/2021.}
 \end{figure}
 
 \begin{figure}[!htbp]
    \centering
    \includegraphics[scale=0.9]{oc_6.png}
    \caption{Postagem no Twitter de Olavo de Carvalho. Disponível em \url{https://twitter.com/OdeCarvalho/status/907765691880796160}. Último acesso em 24/02/2021.}
 \end{figure}
 
%  \begin{figure}[!htbp]
%     \centering
%     \includegraphics[scale=0.9]{oc_7.png}
%     \caption{Postagem no Twitter de Olavo de Carvalho. Disponível em \url{https://twitter.com/MauricioCostaRS/status/1346090946748940294}. Último acesso em 24/02/2021.}
%  \end{figure}
 
  
 \item Onyx Lorenzoni - @onyxlorenzoni
 
 Empresário, veterinário e político, Onyx Lorenzoni\footnote{\url{https://twitter.com/onyxlorenzoni}. Último acesso em 20/02/2021.} exerce hoje o cargo de ministro-chefe da Secretaria-Geral da Presidência do Brasil, sob o governo Bolsonaro. Contrário à ``ideologia de gênero'' e ao feminismo, fez um discurso na Câmara dos deputados, em 2017, no qual liga o movimento feminista e LGBT à pedofilia e à destruição da família. Em suas palavras: ``Não há separação entre ideologia de gênero, agendas LGBT, feministas ou pedófilas. Compõem o mesmo vértice destrutivo, com o mesmo objetivo: destruir a família''\footnote{Veja discurso completo disponível em \url{https://www.camara.leg.br/internet/sitaqweb/TextoHTML.asp?etapa=3&nuSessao=302.3.55.O&nuQuarto=11&nuOrador=2&nuInsercao=0&dtHorarioQuarto=14:30&sgFaseSessao=PE        &Data=11/10/2017&txApelido=ONYX LORENZONI&txEtapa=Com redação final}. Último acesso em 24/02/2021.}.
  
 \item Otoni de Paula - @OtoniDepFederal
 
 Pastor evangélico e político, Otoni de Paula\footnote{\url{https://twitter.com/OtoniDepFederal}. Último acesso em 20/02/2021.} é atual deputado federal pelo Rio de Janeiro. Contrário à descriminalização do aborto e à ``ideologia de gênero'', Otoni já usou seu Twitter também para ironizar feministas e criticar o feminismo.
 
 \begin{figure}[!htbp]
    \centering
    \includegraphics[scale=0.9]{op_3.png}
    \caption{Postagem no Twitter de Otoni de Paula. Disponível em \url{https://twitter.com/OtoniDepFederal/status/843613583258910721}. Último acesso em 24/02/2021.}
 \end{figure}
 
 \newpage
  
 \item Padre Paulo Ricardo - @padre$\_$paulo
 
 Professor, escritor e sacerdote católico, Padre Paulo Ricardo\footnote{\url{https://twitter.com/padre_paulo}. Último acesso em 20/02/2021.} é um líder religioso que, em sua página do Twitter, entre outros assuntos, aborda o feminismo de maneira crítica. Em suas publicações costuma remeter ao seu blog, onde esmiúça pensamentos antifeministas, como o da vocação da mulher para a maternidade e o matrimônio heterossexual. Já palestrou criticamente sobre o ``marxismo cultural'', a ``ideologia de gênero'', e se posicionou favoravelmente ao programa Escola sem partido.
 
 \begin{figure}[!htbp]
    \centering
    \includegraphics[scale=0.9]{ppr_1.png}
    \caption{Postagem no Twitter de Padre Paulo Ricardo. Disponível em \url{https://twitter.com/padre_paulo/status/1131281092537913344}. Último acesso em 24/02/2021.}
 \end{figure}
  
  \newpage
  
 \item Paula Marisa - @profpaulamarisa
 
 Professora, jornalista, especialista em educação e \textit{youtuber}, Paula Marisa\footnote{\url{https://twitter.com/profpaulamarisa}. Último acesso em 20/02/2021.} usa seu Twitter e seus vídeos para promover pautas conservadoras. Em suas publicações, ela se declara contrária à descriminalização do aborto, à ``ideologia de gênero'' e ao feminismo, e favorável ao programa Escola sem partido.
 
 \begin{figure}[!htbp]
    \centering
    \includegraphics[scale=0.9]{ppm_1.png}
    \caption{Postagem no Twitter de Paula Marisa. Disponível em \url{https://twitter.com/profpaulamarisa/status/1104094135315763200}. Último acesso em 24/02/2021.}
 \end{figure}
  
 \item Paulo A. Briguet - @PauloBriguet
 
 Escritor e jornalista, Paulo A. Briguet\footnote{\url{https://twitter.com/PauloBriguet}. Último acesso em 20/02/2021.} é editor-chefe do jornal conservador Brasil Sem Medo (ver \ref{bsm}). É declaradamente contra a descriminalização do aborto, a ``ideologia de gênero'' e o feminismo. Posiciona-se favoravelmente ao programa Escola sem partido.
 
 \begin{figure}[!htbp]
    \centering
    \includegraphics[scale=0.9]{pb_1.png}
    \caption{Postagem no Twitter de Paulo A. Briguet. Disponível em \url{https://twitter.com/PauloBriguet/status/1356401828762710024}. Último acesso em 24/02/2021.}
 \end{figure}
 
  \item Paulo Eduardo Martins - @PauloMartins10
 
 Jornalista e político, Paulo Eduardo Martins\footnote{\url{https://twitter.com/PauloMartins10}. Último acesso em 20/02/2021.} é atual deputado federal pelo Paraná. Conservador, em suas redes sociais se posiciona contrariamente à descriminalização do aborto, à ``ideologia de gênero'' e ao feminismo. É também favorável ao programa Escola sem partido.
 
 \begin{figure}[!htbp]
    \centering
    \includegraphics[scale=0.9]{pem_1.png}
    \caption{Postagem no Twitter de Paulo Eduardo Martins. Disponível em \url{https://twitter.com/PauloMartins10/status/902959371856080896}. Último acesso em 24/02/2021.}
 \end{figure}
  
 \newpage
 
 \item paulo eneas - @pauloeneas
 
 Editor do jornal conservador Crítica Nacional, Paulo Eneas\footnote{\url{https://twitter.com/pauloeneas}. Último acesso em 20/02/2021.} é declaradamente contrário à descriminalização do aborto, ao feminismo e à ``ideologia de gênero'', além de apoiador do programa Escola sem partido.
 
 \begin{figure}[!htbp]
    \centering
    \includegraphics[scale=0.9]{pe_1.png}
    \caption{Postagem no Twitter de paulo eneas. Disponível em \url{https://twitter.com/pauloeneas/status/1320243135000215553}. Último acesso em 24/02/2021.}
 \end{figure}
 
 \item Professora Dayane Pimentel - @deppimentel\label{depdp}
 
 Professora e política, Dayane Pimentel\footnote{\url{https://twitter.com/deppimentel}. Último acesso em 20/02/2021.} exerce atualmente o cargo de deputada federal pela Bahia. Em seus \textit{tweets}, Dayane se define não feminista e se coloca contra a ``ideologia de gênero'' e favorável ao programa Escola sem partido.
 
 \begin{figure}[!htbp]
    \centering
    \includegraphics[scale=0.9]{dep_1.png}
    \caption{Postagem no Twitter de Professora Dayane Pimentel. Disponível em \url{https://twitter.com/deppimentel/status/1188472288200613894}. Último acesso em 24/02/2021.}
 \end{figure}
  
 \item Rodrigo Constantino - @Rconstantino
 
 Rodrigo Constantino\footnote{\url{https://twitter.com/Rconstantino}. Último acesso em 20/02/2021.} é escritor, colunista, presidente do Instituto Liberal e membro-fundador do instituto Millenium. Segundo o próprio, é ``liberal com viés conservador''. Em seus \textit{tweets}, já criticou por diversas vezes as feministas e o feminismo da terceira onda. Também se declara contrário à ``ideologia de gênero''. Já se declarou favorável ao programa Escola sem partido.
 
 \begin{figure}[!htbp]
    \centering
    \includegraphics[scale=0.9]{rc_1.png}
    \caption{Postagem no Twitter de Rodrigo Constantino. Disponível em \url{https://twitter.com/Rconstantino/status/1339558481087574016}. Último acesso em 24/02/2021.}
 \end{figure}
  
  \newpage
  
 \item Rogério Marinho - @rogeriosmarinho
 
 Economista e político, Rogério Marinho\footnote{\url{https://twitter.com/rogeriosmarinho}. Último acesso em 20/02/2021.} é o atual Ministro do Desenvolvimento Regional do Brasil do governo Bolsonaro. Em sua página do Twitter, já fez postagens contrárias à descriminalização do aborto, criticou a ``ideologia de gênero'', provocou feministas e declarou-se favorável ao programa Escola sem partido.
 
 \begin{figure}[!htbp]
    \centering
    \includegraphics[scale=0.9]{rm_1.png}
    \caption{Postagem no Twitter de Rogério Marinho. Disponível em \url{https://twitter.com/rogeriosmarinho/status/723867827392917504}. Último acesso em 24/02/2021.}
 \end{figure}
  
 \item Sarita Coelho - @saritacoelho
 
 Jornalista e mestre em Sociologia, Sarita Coelho\footnote{\url{https://twitter.com/saritacoelho}. Último acesso em 20/02/2021.} também é servidora federal concursada.
 Contrária à ``ideologia de gênero'' e à descriminalização do aborto e crítica ao feminismo, que entende como distorcido pela esquerda, Sarita já se posicionou como favorável ao programa Escola sem partido.
 
 \begin{figure}[!htbp]
    \centering
    \includegraphics[scale=0.9]{sc_1.png}
    \caption{Postagem no Twitter de Sarita Coelho. Disponível em \url{https://twitter.com/saritacoelho/status/1017922925909995520}. Último acesso em 24/02/2021.}
 \end{figure}
  
 \item Sérgio Camargo - @sergiodireita1
 
 Sérgio Camargo\footnote{\url{https://twitter.com/sergiodireita1}. Último acesso em 20/02/2021.} é jornalista e atual presidente da Fundação Cultural Palmares. Contrário ao feminismo e à ``ideologia de gênero'', Sérgio também faz críticas às feministas, por vezes com \textit{tweets} provocativos.
 
 \begin{figure}[!htbp]
    \centering
    \includegraphics[scale=0.9]{sca_1.png}
    \caption{Postagem no Twitter de Sérgio Camargo. Disponível em \url{https://twitter.com/sergiodireita1/status/1216490025745092609}. Último acesso em 24/02/2021.}
 \end{figure}
  
 \item Silas Malafaia - @PastorMalafaia
 
 Psicólogo e pastor protestante, Silas Malafaia\footnote{\url{https://twitter.com/PastorMalafaia}. Último acesso em 20/02/2021.} é bastante presente na discussão política e de gênero no país. Contrário ao casamento igualitário homossexual\footnote{Veja declarações públicas de Malafaia sobre o assunto em uma manifestação cristã em 2011 em \url{https://ultimosegundo.ig.com.br/brasil/marcha-para-jesus-vira-ato-contra-uniao-homoafetiva/n1597044443203.html}. Último acesso em 24/02/2021.}, às feministas e ao feminismo, Silas já se manifestou contrariamente à descriminalização do aborto e favorável ao programa Escola sem partido.
 
 \begin{figure}[!htbp]
    \centering
    \includegraphics[scale=0.9]{sm_1.png}
    \caption{Postagem no Twitter de Silas Malafaia. Disponível em \url{https://twitter.com/PastorMalafaia/status/1042513046629965825}. Último acesso em 24/02/2021.}
 \end{figure}
  
 \item Tamires De Paula - @tamires$\_$scpaula
 
 Estudante de Direito, Tamires de Paula\footnote{\url{https://twitter.com/tamires_scpaula}. Último acesso em 20/02/2021.} é ex-secretária-geral do Partido Social Liberal (PSL) em Itapeva, antigo partido do atual presidente Jair Bolsonaro. Em sua página do Twitter, Tamires se posiciona contrariamente ao feminismo, às feministas, à ``ideologia de gênero'', à descriminalização do aborto, e favoravelmente ao programa Escola sem partido.
 
 \begin{figure}[!htbp]
    \centering
    \includegraphics[scale=0.9]{tp_1.png}
    \caption{Postagem no Twitter de Tamires De Paula. Disponível em \url{https://twitter.com/tamires_scpaula/status/1062589739377872896}. Último acesso em 24/02/2021.}
 \end{figure}
  
 \item TeAtualizei - @taoquei1
 
 \textit{Youtuber} e colunista do jornal conservador Brasil sem medo (ver \ref{bsm}), Bárbara\footnote{\url{https://twitter.com/taoquei1}. Último acesso em 20/02/2021.} comenta política em todas suas redes sociais. Em seu twitter, costuma criticar o feminismo e as ativistas feministas. Já se posicionou favoravelmente ao programa Escola sem partido.
 
%  \begin{figure}[!htbp]
%     \centering
%     \includegraphics[scale=0.9]{ta_1.png}
%     \caption{Postagem no Twitter de TeAtualizei. Disponível em \url{https://twitter.com/taoquei1/status/702538017244839941}. Último acesso em 24/02/2021.}
%  \end{figure}
 
 \begin{figure}[!htbp]
    \centering
    \includegraphics[scale=0.9]{ta_2.png}
    \caption{Postagem no Twitter de TeAtualizei. Disponível em \url{https://twitter.com/taoquei1/status/776118649438560256}. Último acesso em 24/02/2021.}
 \end{figure}
 
 \item Tenente Santini - @TenenteSantini
 
 Policial Militar, empresário e ex-vereador de Campinas, Tenente Santini\footnote{\url{https://twitter.com/TenenteSantini}. Último acesso em 20/02/2021.} já se posicionou em sua página do Twitter como contrário à descriminalização do aborto e favorável ao programa Escola sem partido. Em alguns \textit{tweets}, criticou o movimento feminista e ativistas feministas.
 
 \begin{figure}[!htbp]
    \centering
    \includegraphics[scale=0.9]{ts_1.png}
    \caption{Postagem no Twitter de Tenente Santini. Disponível em \url{https://twitter.com/TenenteSantini/status/1258174877586141184}. Último acesso em 24/02/2021.}
 \end{figure}
 
 \begin{figure}[!htbp]
    \centering
    \includegraphics[scale=0.9]{ts_2.png}
    \caption{Postagem no Twitter de Tenente Santini. Disponível em \url{https://twitter.com/TenenteSantini/status/1203613755961159680}. Último acesso em 24/02/2021.}
 \end{figure}
  
  \newpage
  
 \item Vítor Hugo - @MajorVitorHugo
 
 Militar, advogado e político, Vítor Hugo\footnote{\url{https://twitter.com/MajorVitorHugo}. Último acesso em 20/02/2021.} é atual deputado federal por Goiás. Ativo nas redes sociais, em seus \textit{tweets}, já se posicionou contrariamente à ``ideologia de gênero'', à descriminalização do aborto e ao ``kit gay'', e favoravelmente ao programa Escola sem partido.
 
 \begin{figure}[!htbp]
    \centering
    \includegraphics[scale=0.9]{vh_1.png}
    \caption{Postagem no Twitter de Vítor Hugo. Disponível em \url{https://twitter.com/MajorVitorHugo/status/1035164125628915713}. Último acesso em 24/02/2021.}
 \end{figure}
 
 \newpage

 \subsection*{Mídias, Coletivos e Partidos}

 \item Aliança pelo Brasil - @somosalianca\label{alianca}
 
 A Aliança pelo Brasil\footnote{\url{https://twitter.com/somosalianca}. Último acesso em 20/02/2021.} é uma organização política que pretende se desenvolver em partido político. Foi anunciada por Jair Bolsonaro em 2019, uma vez declarada sua saída do Partido Social Liberal (PSL). Em seu programa publicado, a organização se opõe à descriminalização do aborto e se propõe a combater a ``ideologia de gênero''.
 
 \begin{figure}[!htbp]
    \centering
    \includegraphics[scale=0.8]{alianca_1.png}
    \caption{Postagem no Facebook da Aliança pelo Brasil da Bahia. Disponível em \url{https://m.facebook.com/story.php?story_fbid=209026130562994&id=103489347783340}. Último acesso em 20/02/2021.}
    \label{fig:alianca}
 \end{figure}
 
  \item Brasil Paralelo - @brasil$\_$paralelo
  
  Brasil Paralelo\footnote{\url{https://twitter.com/brasil_paralelo}. Último acesso em 20/02/2021.} é uma mídia e produtora de vídeos, autodeclarada independente, mas que teve a veiculação de uma de suas séries no canal estatal TV Escola durante o governo Bolsonaro\footnote{Detalhes em \url{https://theintercept.com/2020/03/01/allan-terca-livre-governo-bolsonaro/}. Último acesso em 21/02/2021.}. Envolvida em controvérsias nas eleições de 2018 por disseminar informações falsas sobre urnas eletrônicas\footnote{Entre as informações falsas divulgadas estão a de que as urnas teriam mais de $70\%$ de chance de serem fraudadas e de que não seriam auditáveis. Ver \url{https://oglobo.globo.com/fato-ou-fake/mensagens-com-conteudo-fake-sobre-fraude-em-urnas-eletronicas-se-espalham-nas-redes-23134205}. Último acesso em 21/02/2021.}, em 2020 a mídia publicou uma série de 3 capítulos\footnote{Disponível em \url{https://site.brasilparalelo.com.br/series/as-grandes-minorias/}. Último acesso em 21/02/2021.} em que critica e questiona movimentos sociais antifacistas, feministas e negros, os associando ao que chama de "As Grandes Minorias".
 
 \begin{figure}[!htbp]
    \centering
    \includegraphics[scale=0.8]{brasilp_1.png}
    \caption{Postagem no Twitter de Brasil Paralelo. Disponível em \url{https://twitter.com/brasil_paralelo/status/1311332462795124736}. Último acesso em 20/02/2021.}
 \end{figure}

  \item Brasil Sem Medo - @JornalBSM\label{bsm}
  
  Autodeclarado o maior jornal conservador do Brasil, o Brasil Sem Medo\footnote{\url{https://twitter.com/JornalBSM}. Último acesso em 20/02/2021.}, lançado no final de 2019, é uma mídia \textit{online} com textos e \textit{podcasts} associados que publica principalmente artigos de opinião sob um viés conservador. Em alguns de seus artigos, seus colaboradores criticam a ``ideologia de gênero'' e o feminismo (enquanto ``pauta extremista''\footnote{Exemplo em \url{https://brasilsemmedo.com/psol-a-repeticao-da-tragedia/}. Último acesso em 21/02/2021.}).
 
 \begin{figure}[!htbp]
    \centering
    \includegraphics[scale=0.8]{brasils_1.png}
    \caption{Postagem no Twitter de Brasil sem medo. Disponível em \url{https://twitter.com/JornalBSM/status/1230922274318028801}. Último acesso em 20/02/2021.}
 \end{figure}
 
 \newpage
 
  \item Caneta Desesquerdizadora - @Desesquerdizada
  
  Mídia de viés conservador, a Caneta Desesquerdizadora\footnote{\url{https://twitter.com/Desesquerdizada}. Último acesso em 20/02/2021.} conta hoje com site próprio, mas nasceu nas redes sociais. Com o objetivo de expor supostos vieses de esquerda na grande mídia, a página viralizou nas redes com postagens que afirmava ``corrigir'' informações enviesadas de esquerda. Em seu twitter, os responsáveis pela página recorrentemente atacam feministas e suas pautas --- como a descriminalização do aborto.
 
 \begin{figure}[!htbp]
    \centering
    \includegraphics[scale=1]{desesq_5.png}
    \caption{Postagem no Twitter de Caneta Desesquerdizadora. Disponível em \url{https://twitter.com/Desesquerdizada/status/1250573583971225600}. Último acesso em 20/02/2021.}
 \end{figure}
 
 \begin{figure}[!htbp]
    \centering
    \includegraphics[scale=0.95]{desesq_4.png}
    \caption{Postagem no Twitter de Caneta Desesquerdizadora. Disponível em \url{https://twitter.com/Desesquerdizada/status/970505853009883136}. Último acesso em 20/02/2021.}
 \end{figure}
 
%  \begin{figure}[!htbp]
%     \centering
%     \includegraphics[scale=1]{desesq_1.png}
%     \caption{Postagem no Twitter de Caneta Desesquerdizadora. Disponível em \url{https://twitter.com/Desesquerdizada/status/1073270623818924032}. Último acesso em 20/02/2021.}
%  \end{figure}
 
 \newpage
 
  \item Conexão Política - @conexaopolitica\label{conexpol}
  
  Conexão Política\footnote{\url{https://twitter.com/conexaopolitica}. Último acesso em 21/02/2021.} é um jornal digital que faz cobertura e análise política nacional e internacional, em geral sob um viés conservador. Em alguns de seus artigos de opinião e colunas, o jornal divulga críticas ao movimento feminista e à ``ideologia de gênero''\footnote{Veja exemplos de matérias críticas ao movimento feminista em \url{http://conexaopolitica.com.br/artigo/a-vitoria-de-miss-transgenero-e-a-falencia-do-movimento-feminista/} e \url{https://conexaopolitica.com.br/artigo/opiniao/o-odio-esquerdista-e-feminista-contra-o-cristianismo-e-o-ocidente/?}, e à ``ideologia de gênero'', em \url{http://conexaopolitica.com.br/brasil/nossa-colunista-marisa-lobo-foi-processada-por-se-opor-a-ideologia-de-genero/}. Último acesso em 24/02/2021.}. Em 2019, a revista Época publicou um artigo em que analisa as relações políticas dos editores e colaboradores do jornal, que até então se dizia independente. Na matéria, a revista defende que o Conexão Política não faz jornalismo, mas assessoria para políticos conservadores\footnote{A matéria completa está disponível em \url{https://epoca.globo.com/opiniao-como-nasce-um-embuste-23397102}. Último acesso em 24/02/2021.}.
  
 \item Crítica Nacional - @criticanac
 
 Autodeclarado ``um projeto que oferece ao público um conteúdo jornalístico online conservador e de direita da melhor qualidade'', o portal Crítica Nacional\footnote{\url{https://twitter.com/criticanac}. Último acesso em 21/02/2021.} é um site de notícias, análises políticas e artigos de opinião políticos, sob o viés de direita. Os \textit{tweets} da página do portal nem sempre são de divulgação do conteúdo supostamente jornalístico do portal, sendo por vezes apenas publicações de crítica direta a alguma entidade ou movimento que seus editores não tenham acordo.
 
 \begin{figure}[!htbp]
    \centering
    \includegraphics[scale=0.85]{cnac_1.png}
    \caption{Postagem no Twitter de Crítica Nacional. Disponível em \url{https://twitter.com/criticanac/status/823004089235673091}. Último acesso em 20/02/2021.}
 \end{figure}
 
 \begin{figure}[!htbp]
    \centering
    \includegraphics[scale=0.85]{cnac_2.png}
    \caption{Postagem no Twitter de Crítica Nacional. Disponível em \url{https://twitter.com/criticanac/status/736275813121675265}. Último acesso em 20/02/2021.}
 \end{figure}
 
  \newpage
  
 \item Direita Goiás - @direitagoiasofc
 
 Movimento regional político, conservador e de direita, a Direita Goiás\footnote{\url{https://twitter.com/direitagoiasofc}. Último acesso em 20/02/2021.} promove valores conservadores no estado de Goiás e no Brasil. Contando também com um canal no Youtube, o Direita Goiás já usou suas redes sociais para criticar o feminismo e ativistas feministas, para denunciar a suposta ``ideologia de gênero'' e para apoiar o programa Escola sem partido.
 
 \begin{figure}[!htbp]
    \centering
    \includegraphics[scale=0.85]{z_dg_1.png}
    \caption{Postagem no Twitter de Direita Goiás. Disponível em \url{https://twitter.com/direitagoiasofc/status/1007950678000635906}. Último acesso em 20/02/2021.}
 \end{figure}
  
 \item Direita Minas - @direitaminas\label{dirmin}
 
 Movimento regional de direita, Direita Minas\footnote{\url{https://twitter.com/direitaminas}. Último acesso em 20/02/2021.}, em alguns \textit{tweets}, publica críticas ao movimento feminista e às ativistas feministas. O movimento se coloca contrário à ``ideologia de gênero'' e favorável ao programa Escola sem partido.
 
 \begin{figure}[!htbp]
    \centering
    \includegraphics[scale=0.85]{z_dm_1.png}
    \caption{Postagem no Twitter de Direita Minas. Disponível em \url{https://twitter.com/direitaminas/status/1240437794817871879}. Último acesso em 20/02/2021.}
 \end{figure}
 
  \newpage
  
 \item Escola sem Partido - @escolasempartid
 
 Movimento político criado pelo advogado Miguel Nagib, cunhado de Bia Kicis, o Escola Sem Partido\footnote{\url{https://twitter.com/escolasempartid}. Último acesso em 20/02/2021.} é um programa que supostamente visa criar leis que proíbam professores de promover seus interesses e opiniões em sala de aula e os impeçam de desfavorecer estudantes com opiniões contrárias. Em suas postagens e ações, porém, os responsáveis pelo movimento costumam criticar a suposta existência de \textit{doutrinações específicas}: a ``esquerdista'', a de ``ideologia de gênero'' e a feminista. Questionados sobre o porquê de não se manifestarem em casos de supostas doutrinações de direita, respondem que apenas quem veste a camisa do movimento pode exigir posicionamentos.
 
%  \begin{figure}[!htbp]
%     \centering
%     \includegraphics[scale=0.8]{esp_1.png}
%     \caption{Postagem no Twitter de Escola sem Partido e suposta evidência de propaganda do governo petista e da ``ideologia de gênero'': livro didático de sociologia com citação de filósofos e de movimento estudado pelas Ciências Sociais. Disponível em \url{https://twitter.com/escolasempartid/status/974080297942835200}. Último acesso em 23/02/2021.}
%  \end{figure}
 
 \begin{figure}[!htbp]
    \centering
    \includegraphics[scale=0.7]{esp_3.png}
    \caption{Postagem no Twitter de Escola sem Partido. Disponível em \url{https://twitter.com/escolasempartid/status/1039503343217831941}. Último acesso em 23/02/2021.}
 \end{figure}
 
 \begin{figure}[!htbp]
    \centering
    \includegraphics[scale=0.8]{esp_5.png}
    \caption{Postagem no Twitter de Escola sem Partido. Disponível em \url{https://twitter.com/escolasempartid/status/1104367491332096001}. Último acesso em 23/02/2021.}
 \end{figure}
 
  \newpage
  
 \item FamíliaDireitaBrasil - @BrazilFight
 
 Página de divulgação de conteúdo conservador, FamíliaDireitaBrasil\footnote{\url{https://twitter.com/BrazilFight}. Último acesso em 20/02/2021.} já demonstrou em seus \textit{tweets} contrariedade ao feminismo, às feministas e à ``ideologia de gênero'', além de apoio ao programa Escola sem partido.
 
 \begin{figure}[!htbp]
    \centering
    \includegraphics[scale=0.8]{z_fam_1.png}
    \caption{Postagem no Twitter de FamíliaDireitaBrasil. Disponível em \url{https://twitter.com/BrazilFight/status/1161268558833340416}. Último acesso em 25/02/2021.}
 \end{figure}
  
 \item MBC – Movimento Brasil Conservador - @EuSouMBC
 
 Compreendido por seus membros como uma comunidade de conservadores, o Movimento Brasil Conservador\footnote{\url{https://twitter.com/EuSouMBC}. Último acesso em 20/02/2021.} (MBC) se propõe a ``trabalhar pela reconstrução do país, pautados na defesa dos pilares da civilização ocidental e no combate à dominação cultural imposta por ideologias revolucionárias''. Dentro dessa comunidade, os membros têm acesso ao que chamam de ``formação intelectual'' e realizam ``desde cursos, grupos de estudos, seminários e congressos até manifestações de rua e nas câmaras e assembleias legislativas''\footnote{Veja mais sobre o movimento em \url{https://www.eusoumbc.org/}. Último acesso em 24/02/2021.}. Através de \textit{tweets}, o MBC se posiciona favorável ao projeto Escola sem partido, contrário à descriminalização do aborto e à ideologia de gênero. Também já publicaram que compreendem o feminismo como uma ``farsa''.
 
 \begin{figure}[!htbp]
    \centering
    \includegraphics[scale=0.8]{z_mbc_1.png}
    \caption{Postagem no Twitter de MBC – Movimento Brasil Conservador. Disponível em \url{https://twitter.com/EuSouMBC/status/1149712044339240960}. Último acesso em 25/02/2021.}
 \end{figure}
 
  \newpage
  
 \item MBL - Movimento Brasil Livre - @MBLivre
 
 O Movimento Brasil Livre\footnote{\url{https://twitter.com/MBLivre}. Último acesso em 20/02/2021.} é um movimento político que defende principalmente o liberalismo econômico. Em seus \textit{tweets}, com alguma frequência costumam criticar o feminismo e ironizar as ativistas do movimento. Além disso, o MBL já usou o espaço da página para se posicionar contrariamente à descriminalização do aborto e à ``ideologia de gênero'', e favoravelmente ao programa Escola sem partido.
 
 \begin{figure}[!htbp]
    \centering
    \includegraphics[scale=0.8]{z_mbl1.png}
    \caption{Postagem no Twitter de MBL - Movimento Brasil Livre. Disponível em \url{https://twitter.com/MBLivre/status/1138900793250263040}. Último acesso em 25/02/2021.}
 \end{figure}
  
 \item Movimento Avança Brasil  - @MAvancaBrasil\label{mab}
 
 O Movimento Avança Brasil\footnote{\url{https://twitter.com/MAvancaBrasil}. Último acesso em 25/02/2021.} é um movimento político conservador, que promove e defende valores conservadores. Em seus \textit{tweets}, já se posicionou contrário ao feminismo e às feministas, à ``ideologia de gênero'', à descriminalização do aborto, e favorável ao programa Escola sem partido.
 
 \begin{figure}[!htbp]
    \centering
    \includegraphics[scale=0.8]{mab_1.png}
    \caption{Postagem no Twitter de Movimento Avança Brasil. Disponível em \url{https://twitter.com/MAvancaBrasil/status/1050203324987920385}. Último acesso em 25/02/2021.}
 \end{figure}
  
  \newpage
  
 \item Partido Social Liberal - PSL - @PSL$\_$Nacional \label{psl}
 
 Antigo partido de Bolsonaro, o Partido Social Liberal (PSL)\footnote{\url{https://twitter.com/PSL_Nacional}. Último acesso em 25/02/2021.} se define como liberal no âmbito econômico enquanto defende o conservadorismo nos costumes. Em sua página do Twitter, o partido já se posicionou favoravelmente ao programa Escola sem partido e contrariamente à ``ideologia de gênero'' e à descriminalização do aborto.
 
 \begin{figure}[!htbp]
    \centering
    \includegraphics[scale=0.8]{psl_1.png}
    \caption{Postagem no Twitter de Partido Social Liberal - PSL. Disponível em \url{https://twitter.com/PSL_Nacional/status/1046731459073994754}. Último acesso em 25/02/2021.}
 \end{figure}
 
 \begin{figure}[!htbp]
    \centering
    \includegraphics[scale=0.8]{psl_2.png}
    \caption{Postagem no Twitter de Partido Social Liberal - PSL. Disponível em \url{https://twitter.com/PSL_Nacional/status/1030626626101567489}. Último acesso em 25/02/2021.}
 \end{figure}
 
  \newpage
  
 \item Portal Terça Livre - @tercalivre
 
 Terça Livre\footnote{\url{https://twitter.com/tercalivre}. Último acesso em 25/02/2021.} é uma empresa que reúne um veículo jornalístico e uma central de cursos de formação, ambos de viés conservador. Em sua página do Twitter, o Terça Livre já se posicionou favorável ao programa Escola sem partido e contrário à ``ideologia de gênero'' e ao feminismo.
 
 \begin{figure}[!htbp]
    \centering
    \includegraphics[scale=0.8]{ptl_3.png}
    \caption{Postagem no Twitter de Portal Terça Livre. Disponível em \url{https://twitter.com/tercalivre/status/742063869544140800}. Último acesso em 25/02/2021.}
 \end{figure}
  
 \item Senso Incomum - @sensoinc
 
 Senso incomum\footnote{\url{https://twitter.com/sensoinc}. Último acesso em 25/02/2021.} é um portal que discute política e cultura, e que define suas ideias como ``contra as explicações fáceis do jornalismo e da Academia''. Além do portal, o Senso Incomum conta com \textit{podcast} e canal no Youtube. Em sua página do Twitter, o Senso Incomum se coloca como crítico ao feminismo --- supostamente superado--- e às feministas, além de contrário à ``ideologia de gênero'' e à descriminalização do aborto.
 
 \begin{figure}[!htbp]
    \centering
    \includegraphics[scale=0.8]{sic_1.png}
    \caption{Postagem no Twitter de Senso Incomum. Disponível em \url{https://twitter.com/sensoinc/status/972655866204827648}. Último acesso em 25/02/2021.}
 \end{figure}

%   \footnote{Ver \url{}. Último acesso em 25/02/2021.}
%   \footnote{Ver \url{}. Último acesso em 25/02/2021.}
%   \footnote{Ver \url{}. Último acesso em 25/02/2021.}
%   \footnote{Ver \url{}. Último acesso em 25/02/2021.}

\end{enumerate}

 \subsection*{Comentários}
 Assim como aconteceu com as contas feministas, alguns influenciadores foram desconsiderados na análise de resultados por diversos motivos, externos à vontade da pesquisadora. Além da questão dos requisitos do código (como o de que o influenciador deve ser seguido por duzentos entre os dez mil cidadãos que mais seguem contas de influenciadores selecionados para o estudo, sejam feministas ou antifeministas), algumas contas também foram encerradas, retidas ou desativadas, por determinação da justiça, do próprio Twitter ou do influenciador. Deixamos de analisar, por exemplo, o \textit{podcast ShockWaveRadio}\footnote{Conta desativada e indisponível, estava em \url{http://twitter.com/ShockWaveRadio_}. Último acesso em 25/02/2021.}, o co-fundador do Movimento Brasil Conservador, Henrique Olliveira\footnote{Conta desativada e indisponível, estava em \url{http://twitter.com/henriolliveira}. Último acesso em 25/02/2021.} e o Capitão Assumção\footnote{\url{https://twitter.com/capitaoassumcao}. Último acesso em 25/02/2021.}; os dois primeiros pela indisponibilidade das contas, e o último por não seguir o requisito do código do modelo.
 
 Feitas as considerações técnicas, é interessante notar que uma quantidade bastante relevante dos influenciadores que consideramos de antemão como antifeministas é ou foi ligado ao governo Bolsonaro, ou se declara favorável a ele. Esse fenômeno não é inesperado, e há estudiosos do contexto político atual que já fizeram associações semelhantes. Por exemplo, em entrevista ao UOL, a pesquisadora Rosana Pinheiro-Machado declarou que o chamado bolsonarismo\footnote{Na entrevista, a pesquisadora define o bolsonarismo como ``a vitória da extrema-direita, a radicalização de uma política de ``nós contra eles'', cujo inimigo é interno --- e não externo''. Evidentemente, o termo é uma referência à eleição de Jair Bolsonaro, presumivelmente o símbolo/personificação da definição feita por Rosana.} é também uma reação ao fenômeno da nova geração de meninas feministas, inédito no Brasil\footnote{Veja matéria completa em \url{https://noticias.uol.com.br/politica/ultimas-noticias/2019/03/01/e-impossivel-separar-bolsonarismo-do-antifeminismo-diz-antropologa.htm}. Último acesso em 20/02/2021.}, o que torna o antifeminismo e o bolsonarismo inseparáveis. Além dessa questão, precisamos considerar também o fato de que os influenciadores do estudo não partiram do vácuo: como não há um grupo ``oficial'' de antifeministas, a pesquisadora escolheu uma amostra de representantes que conhecia, e que em geral possuíam comprovadamente as características enumeradas na seção 1.2.8 do capítulo 1. A partir daí, foram descobertos outros representantes relacionados aos primeiros, com características semelhantes, através de intensa procura em redes sociais e em notícias sobre figuras públicas que apresentaram contrariedade ao feminismo ou controvérsias com o mesmo e suas ativistas.
 
 \chapter{Método de estimação baseado em simulação: Markov Chain Monte Carlo}\label{mcmc}
 Introduziremos aqui o \emph{Markov Chain Monte Carlo (MCMC)} como um método para calcular expressões como a esperança apresentada na seção 2.2.3 do capítulo 2. Toda essa seção é amplamente baseada em (\citeauthoronline{gilks1996}, \citeyear{gilks1996}).

 Dividiremos essa elucidação em três partes, correspondentes aos métodos constitutivos do MCMC e a sua aplicação prática (computacional). São elas: 1) Integração de Monte Carlo, 2) Cadeias de Markov e 3) o algoritmo de Metropolis-Hastings.
 
 \section{Integração de Monte Carlo}
 Digamos que queremos estimar a esperança anterior
 \begin{equation}
    \begin{aligned}
        E(f(\theta)) = \frac{\int f(\theta) \pi(\theta) d\theta}{\int \pi(\theta) d\theta},
     \end{aligned}
 \end{equation}
 onde a distribuição a \emph{posteriori} está representada como $\pi(\theta)$, sendo
 \begin{equation}
    \begin{aligned}
        \pi(\theta) = p(\theta) p(Y|\theta).
     \end{aligned}
 \end{equation}
 
 A integração de Monte Carlo pode ser usada para calcular $E(f(\theta))$ através do sorteio de amostras $Y = (y_{1}, ..., y{t})$, $t = (1, …, n)$, retiradas de $\pi(\theta)$. Feita a amostragem, fazemos a aproximação
 \begin{equation}
    \begin{aligned}
        E(f(\theta)) \approx \frac{1}{n} \sum_{t = 1}^{n} f(y_{t}).
     \end{aligned}
 \end{equation}
 Dessa maneira, a esperança da população de f($\theta$) é aproximada por uma média simples. Se as amostras $Y$ são independentes, a lei dos grandes números garante que a aproximação será tão acurada quando desejarmos, uma vez que aumentemos o tamanho n da amostra.
 
 Normalmente, o sorteio de amostras independentes $Y_{t}$ de  não é tão simples. Uma das formas de fazê-lo é usando uma cadeia de Markov que tenha $\pi(\theta)$ como sua distribuição estacionária. Usando esse recurso, o processo completo será do tipo Cadeia de Markov-Monte Carlo.
 
 \section{Cadeia de Markov}
 Uma cadeia de Markov é uma sequência de variáveis aleatórias $X = (x_{0}, ..., x_{t})$, tal que, para cada $t \geq 0$, o próximo estado $x_{t+1}$ é amostrado de uma distribuição $p(x_{t+1}|x_{t})$ que depende apenas do atual estado da cadeia, $x_{t}$. Isso significa que, dado $x_{t}$, o próximo estado $x_{t+1}$ não depende de toda a história da cadeia $x_{0}, ..., x_{t-1}$, mas apenas de $x_{t}$. A probabilidade $p(.|.)$ de transição é chamada de \emph{núcleo de transição da cadeia}.
 
 O estado inicial $x_{0}$ afeta $x_{t}$ segundo a distribuição de $x_{t}$ dado $x_{0}$, que denotamos $p^{t}(x_{t}|x_{0})$. Nessa distribuição não consideramos as variáveis intermediárias $x_{1}, ..., x_{t-1}$, e $x_{t}$ depende apenas de $x_{0}$. Sujeita a algumas condições, a cadeia vai ``esquecendo'' gradualmente seu estado inicial, e $p^{t}(.|x_{0})$ eventualmente converge para uma distribuição única estacionária (ou invariante), que não depende de $t$ ou de $x_{0}$, se mantendo homogênea no tempo. Após a convergência, dizemos que a distribuição é uma distribuição estacionária $\psi(.)$. Assim, a medida que $t$ aumenta, os pontos amostrados ${X_{t}}$ vão parecer cada vez mais como amostras independentes de $\psi(.)$. Esse processo é ilustrado em três casos a seguir, na figura \ref{fig:gilksmcmc}, onde $\psi(.)$ tem padrão invariante normal.

 \begin{figure}[!htbp]
    \centering
    \includegraphics[scale=0.8]{gilks_mcmc.png}
    \caption{500 iterações do algoritmo de Metropolis com distribuição estacionária $N(0,1)$ e distribuições propostas (a) $q(.|X) = N(X, 0.5)$; (b) $q(.|X) = N(X, 0.1)$ e (c) $q(.|X) = N(X, 10)$. O período de aquecimento, que visa não pegar amostras antes da convergência, está à esquerda da linha vertical tracejada. Esse exemplo foi retirado de \cite[p.~6]{gilks1996}.}
    \label{fig:gilksmcmc}
 \end{figure}
 
 Após um período de aquecimento suficientemente longo de, digamos, $m$ iterações, os pontos $x_{t}$, dado $t = (m+1, ..., n)$ serão amostras aproximadamente independentes de $\psi(.)$. Podemos, então, usar a saída da cadeia de Markov para estimar a esperança $E(f(\theta))$, que citamos anteriormente, onde $Y$ tem distribuição $\psi(.)$. Normalmente, as amostras do período de aquecimento são desconsideradas nos cálculos. Assim, a aproximação da esperança pode ser representada por
 \begin{equation}
    \begin{aligned}
        f = \frac{1}{n-m} \sum_{t = m+1}^{n} f(y_{t}),
     \end{aligned}
 \end{equation}
 e é a chamada média ergódica. A convergência para a esperança requerida é garantida pelo teorema ergódico.
 
 \section{Algoritmo de Metropolis-Hastings}
 Demonstraremos aqui como construir, na prática, um método usando MCMC tal que encontremos amostras da distribuição estacionária $\psi(.)$, suficientemente parecida com nossa distribuição a \emph{posteriori} de interesse $\pi(.)$, que nos permita calcular as estatísticas necessárias para inferência dos valores de nossos parâmetros $\theta$. Comentaremos o método proposto por \apud[p.~5]{hastings1970}{gilks1996}, por sua vez uma generalização do método proposto anteriormente por \apud[p.~7]{metropolis1953}{gilks1996}.
 
 O algoritmo de Metropolis-Hastings tem o seguinte funcionamento: para cada tempo $t$, o próximo estado $x_{t+1}$ será escolhido através da rejeição ou aceitação de um ponto $Z$ candidato, amostrado de uma distribuição proposta $q(.|x_{t})$, que dependerá apenas do ponto atual $x_{t}$. Por exemplo: podemos propor $q(.|X)$ como uma distribuição normal com média $X$ e uma matriz de covariância fixada. O ponto candidato $Z$ será aceito com probabilidade $\alpha(X, Z)$, tal que
 \begin{equation}
    \begin{aligned}
        \alpha(X, Z) = \min\bigg(1, \frac{\pi(Z) q(X|Z)}{\pi(X) q(Z|X)}\bigg).
     \end{aligned}
 \end{equation}
 
 Se o ponto candidato $Z$ é aceito, o próximo estado se tornará $x_{t+1} = Z$. Caso contrário, $x_{t+1} = x_{t}$. A figura \ref{fig:gilksmcmc} anterior ilustra esse processo para algumas distribuições propostas $q$ diferentes; no caso c), podemos observar, inclusive, várias instâncias do caso em que a cadeia não muda por algumas iterações, por conta da rejeição recorrente de candidatos Z.
 
 Dada a descrição, podemos resumir o algoritmo de Metropolis-Hastings da seguinte maneira:
 \begin{enumerate}
  \item Escolhemos $x_{0}$ e fazemos que $t = 0$;
  \item Amostramos um ponto $Z$ de $q(.|x_{t})$ e calculamos $\alpha(x_{t}, Z)$;
  \item Amostramos uma variável aleatória $U$ de uma distribuição uniforme em $[0,1]$;
  \item Se $U \leq \alpha(x_{t}, Z)$, fazemos $x_{t+1} = Z$. Caso contrário, $x_{t+1} = x_{t}$;
  \item Incrementamos t e voltamos para o passo 2.
\end{enumerate}
 
 Podemos escolher a distribuição proposta $q(.)$ livremente; o algoritmo será mais eficiente, porém, se essa distribuição for uma boa aproximação para a distribuição a \emph{posteriori} $\pi(.)$ que queremos simular.
 
 É interessante adicionar algumas considerações de Mendonça (\citeyear{mendonca2008}). Após a implementação do algoritmo, é necessário verificar se a cadeia gerada converge para a distribuição de interesse. A verificação pode ser feita através de diversos diagnósticos de convergência - métodos numéricos, por exemplo. Demonstraremos uma verificação de convergência em um exemplo, na subseção a seguir.
 
 Outras características do algoritmo na prática que são interessantes de ser melhor discutidas, até para a compreensão do código, são o período de aquecimento (em inglês, \emph{burn-in}), e os saltos (ou, em inglês, \emph{leapfrogs}).

 Um aquecimento de tamanho B eliminará os B primeiros valores gerados da amostra, a contar do valor inicial. Sua finalidade é desconsiderar o que foi gerado antes da convergência da distribuição para a estacionariedade.

 O salto de valor L garantirá que, após o período de aquecimento, os valores serão selecionados para a amostra final a cada L iterações. O intuito desse processo é aumentar a independência entre os valores gerados.

 Assim, o primeiro valor da amostra será aquele gerado depois das B iterações de aquecimento, o segundo, após B+L iterações, o terceiro, após B+2L iterações, e assim por diante. Se quisermos que nossa amostra final tenha 1000 valores, com período de queimada de tamanho 100 e saltos de tamanho de 10, 10000 iterações de amostragem serão necessárias, além das 100 iterações de aquecimento, totalizando 10100 iterações.
 
 \begin{figure}[!htbp]
    \centering
    \includegraphics[scale=0.8]{mendonca_burnin.png}
    \caption{Queimada e saltos, segundo \cite[p.~33]{mendonca2008}.}
    \label{fig:mendoncaburnin}
 \end{figure}
 
 \chapter{Códigos e Implementações}\label{codigoimplementacao}
 A implementação desse modelo não foi muito simples. Apesar de Pablo Barberá disponibilizar os códigos utilizados na confecção do artigo\footnote{Disponíveis em \url{https://github.com/pablobarbera/twitter_ideology/tree/master/replication}. Último acesso em 06/03/2021.}, algumas funções estavam desatualizadas ou tinham sido descontinuadas. Além disso, o tempo de execução era muito grande para algumas partes do código, então fizemos certas melhorias na performance; ainda assim, alguns dos arquivos demoram dias para terminar de rodar. Nossas mudanças estão disponíveis no github\footnote{Códigos disponíveis em \url{https://github.com/MoraisCamila91/TCCBarberaFeminismo}. Os dados gerados estão disponíveis em \url{https://drive.google.com/drive/folders/14NVcD6G6JVX87nRSTXr5Hq_bpgtUxw9V?usp=sharing}, porém. Último acesso em 06/03/2021.} e serão comentadas a seguir.
 
 A implementação foi baseada em cinco estágios, correspondentes aos nossos cinco códigos homônimos aos arquivos de replicação disponibilizados por Barberá: \textit{00-install-packages.R}, \textit{01-get-twitter-data.R}, \textit{02-get-users-data.R}, \textit{03-create-adjacency-matrix.R} e \textit{04-model-first-stage.R}.
 
 No arquivo \textit{00-install-packages.R}, fazemos a instalação no R dos pacotes necessários. Adicionamos ao arquivo de Barberá a instalação da biblioteca \textit{ROAuth}, e alteramos a instalação da \textit{RStan}, que havia mudado desde a implementação do autor.
 
 Usando o arquivo \textit{01-get-twitter-data.R}, fazemos o download dos seguidores dos influenciadores que escolhemos via API do Twitter. Houve vários problemas técnicos nessa parte, que foi bastante lenta. Nossa alteração se deteve, porém, em diminuir o filtro de exigência de quantidade de seguidores do influenciador. Enquanto Barberá exigia que o político tivesse pelo menos cinco mil seguidores, exigimos apenas dois mil seguidores de mínimo. Na prática, isso não mudou nossos resultados finais: nosso influenciador com menos seguidores possuía mais de quinze mil seguidores no momento do estudo.
 
 O estágio correspondente ao arquivo \textit{02-get-users-data.R} foi o mais complexo de replicação. Nesse estágio, compilávamos todos os seguidores de todos os influenciadores em uma tabela, excluindo os repetidos. Usávamos então a API do Twitter para obter uma série de informações desses usuários, das quais mantínhamos: quantas contas seguiam, por quantas eram seguidos, quantos \textit{tweets} haviam publicado, qual a data do último \textit{tweet} e a localização geográfica registrada. Em relação às mudanças que fizemos em relação ao arquivo original, alteramos o formato das tabelas que Barberá usava para o formato \textit{dataframe}, de mais fácil manipulação. Além disso, não conseguimos usar o banco de dados MongoDB, então implementamos um mecanismo de adicionar localmente a um arquivo \textit{.RData} cada conjunto de cem seguidores obtidos via API. Enquanto o mecanismo de Barberá adicionava os seguidores um a um no banco, conseguimos salvar de cem em cem.
 
 Uma vez obtidos os dados dos seguidores, fizemos então a matriz adjacência, usando o arquivo \textit{03-create-adjacency-matrix.R}. Não mudamos praticamente nada do original em nossa adaptação.
 
 Finalmente, utilizamos o código de \textit{04-model-first-stage.R} para a inferência bayesiana de ponto ideal dos influenciadores. Nesse estágio, escolhemos dez mil cidadãos que seguem mais de dez influenciadores do estudo, e usamos essa amostra para estimar o ponto ideal dos influenciadores. Nosso código é mais curto que o de Barberá, porque o do autor usa alguns trechos de códigos para configurar o ponto de partida de Cadeias de Markov de países diferentes. Como temos apenas um cenário, de feminismo e antifeminismo no Brasil, mantivemos apenas um desses trechos. Uma vez definidos os pontos de partida das cadeias de Markov e escolhidos os dez mil cidadãos, é feito um corte nos influenciadores, ficando apenas aqueles que são seguidos por mais de duzentos entre os dez mil cidadãos. Salvamos os nomes desse corte de influenciadores em um arquivo \textit{.csv}, para conseguirmos saber de quem se trata. Barberá não lança mão desse recurso porque tem o nome dos influenciadores na tabela \textit{results}, no último trecho de código. Não conseguimos registrar esses nomes da mesma maneira. Outras alterações nesse código foram o espaçamento das linhas e o aumento dos comentários, para facilitar a legibilidade e a compreensão.
 
 \chapter{Resultados: Tabelas e Gráficos Adicionais}\label{figurascompletas}
 Nesse capítulo, incluiremos algumas análises gráficas que não foram selecionadas para o texto principal.
 
 \section{Exemplo Modelo TRI usando MCMC}\label{trimcmcextra}
 Na seção 2.3.6 do capítulo 2, apresentamos um exemplo de uso de inferência Bayesiana usando MCMC, e fizemos a análise de resultados partindo do uso de três escolhas diferentes para as distribuições a \textit{priori} dos parâmetros. Algumas figuras extras à essa análise podem ser encontradas a seguir.
 
 \subsection{Inferência dos Parâmetros usando distribuições a \emph{priori} Normal padrão}\label{extrapadrao}
 Para analisarmos a convergência da estimação dos parâmetros $\theta$ e garantir a confiabilidade dos resultados, podemos estudar a tabela-resumo das estatísticas das estimativas das distribuições a \textit{posteriori} desses parâmetros.
 
 \begin{figure}[!htbp]
    \centering
    \includegraphics[scale=0.8]{tabela_15_normal01.png}
    \caption{Tabela completa, para distribuições a \textit{priori} N(0,1). Parte I.}
    \label{fig:tabelacompleta01}
 \end{figure}
 
 \begin{figure}[!htbp]
    \centering
    \includegraphics[scale=0.8]{tabela_16_85_normal01.png}
    \caption{Tabela completa, para distribuições a \textit{priori} N(0,1). Parte II.}
    \label{fig:tabelacompleta02}
 \end{figure}
 
 \begin{figure}[!htbp]
    \centering
    \includegraphics[scale=0.8]{tabela_86final_todos_normal01.png}
    \caption{Tabela completa, para distribuições a \textit{priori} N(0,1). Parte III.}
    \label{fig:tabelacompleta03}
 \end{figure}
 
 \newpage
 
 As cadeias de $\theta$ geraram mais de 300 gráficos, todos com quatro cadeias de markov sobrepostas. Não é viável incluirmos mesmo no anexo todas as figuras, então incluímos uma pequena amostra com quatro exemplos na figura \ref{fig:thetasquatro01}.
 
 As cadeias semelhantes geradas para os quatro $b's$ podem ser visualizadas na figura \ref{fig:betasquatro01}.
 
 \begin{figure}[!htbp]
    \centering
    \includegraphics[scale=0.65]{estimacoes_thetas_exemplos_todos_normal01.png}
    \caption{Estimativas de $\theta$: Cadeias de Markov. A área acinzentada corresponde ao período de aquecimento.}
    \label{fig:thetasquatro01}
 \end{figure}
 
 \begin{figure}[!htbp]
    \centering
    \includegraphics[scale=0.8]{estimacoes_betas_todos_normal01.png}
    \caption{Estimativas de $b$: Cadeias de Markov. A área acinzentada corresponde ao período de aquecimento.}
    \label{fig:betasquatro01}
 \end{figure}
 
 \newpage
 
 \subsection{Inferência dos Parâmetros usando distribuições a \emph{priori} escolhidas}\label{extraescolhida}
 A convergência e outras estatísticas das estimativas das distribuições a \textit{posteriori} dos parâmetros $\theta$ utilizando distribuições a \textit{priori} escolhidas $b \sim N(-0.2, 0.8)$ e $\theta \sim N(0.2, 0.8)$ podem ser observadas na tabela a seguir.
 
 \begin{figure}[!htbp]
    \centering
    \includegraphics[scale=0.7]{tabela_0208_pt1.png}
    \caption{Tabela completa, para distribuições a \textit{priori} $b \sim N(-0.2, 0.8)$ e $\theta \sim N(0.2, 0.8)$. Parte I.}
    \label{fig:tabelacompleta0208}
 \end{figure}
 
 \begin{figure}[!htbp]
    \centering
    \includegraphics[scale=0.8]{tabela_0208_pt2.png}
    \caption{Tabela completa, para distribuições a \textit{priori} $b \sim N(-0.2, 0.8)$ e $\theta \sim N(0.2, 0.8)$. Parte II.}
    \label{fig:tabelacompleta02082}
 \end{figure}
 
 Uma amostra das cadeias de Markov geradas para estimar as distribuições a \textit{posteriori} dos parâmetros $\theta$ está a seguir, conjuntamente com as cadeias geradas para estimar as distribuições a \textit{posteriori} dos parâmetros $b$.
 
 \begin{figure}[!htbp]
    \centering
    \includegraphics[scale=1]{thetas_amostra_0208.png}
    \caption{Estimativas de $\theta$: Cadeias de Markov. A área acinzentada corresponde ao período de aquecimento.}
    \label{fig:thetasquatro0208}
 \end{figure}
 
 \begin{figure}[!htbp]
    \centering
    \includegraphics[scale=1]{betas0208.png}
    \caption{Estimativas de $b$: Cadeias de Markov. A área acinzentada corresponde ao período de aquecimento.}
    \label{fig:betasquatro0208}
 \end{figure}
 
 \newpage
 \subsection{Inferência dos Parâmetros usando distribuições a \emph{priori} pouco informativas}\label{extrapoucoinfo}
 Podemos analisar a convergência e outras estatísticas das estimativas das distribuições a \textit{posteriori} dos parâmetros $\theta$ com distribuições a \textit{priori} pouco informativas ($b \sim N(0, 100)$ e $\theta \sim N(0, 100)$) na tabela a seguir.
 
 \begin{figure}[!htbp]
    \centering
    \includegraphics[scale=0.8]{tabela_0100_amostra.png}
    \caption{Tabela completa, para distribuições a \textit{priori} N(0,100). Parte I.}
    \label{fig:tabelacompleta0100}
 \end{figure}
 
 \begin{figure}[!htbp]
    \centering
    \includegraphics[scale=0.95]{tabela_0100_pt1.png}
    \caption{Tabela completa, para distribuições a \textit{priori} N(0,100). Parte II.}
    \label{fig:tabelacompleta01002}
 \end{figure}
 
 \begin{figure}[!htbp]
    \centering
    \includegraphics[scale=0.95]{tabela_0100_pt2.png}
    \caption{Tabela completa, para distribuições a \textit{priori} N(0,100). Parte III.}
    \label{fig:tabelacompleta01002}
 \end{figure}
 
 Uma amostra das Cadeias de Markov das estimativas das distribuições a \textit{posteriori} de $\theta$ e as Cadeias de Markov das estimativas das distribuições a \textit{posteriori} de $b$ estão nas figuras abaixo.
 
 Sabemos que os $b$ não obtiveram uma boa métrica de convergência (veja a convergência na seção 2.3.6 do capítulo 2). Note como o comportamento de suas cadeias é bem diferente do comportamento das cadeias de $b$ anteriores, comparando com os gráficos \ref{fig:betasquatro01} e \ref{fig:betasquatro0208}.
 
 \begin{figure}[!htbp]
    \centering
    \includegraphics[scale=1]{thetas_fit_0100.png}
    \caption{Estimativas de $\theta$: Cadeias de Markov. A área acinzentada corresponde ao período de aquecimento.}
    \label{fig:thetasquatro0100}
 \end{figure}
 
 \begin{figure}[!htbp]
    \centering
    \includegraphics[scale=1]{beta_fit_0100.png}
    \caption{Estimativas de $b$: Cadeias de Markov. A área acinzentada corresponde ao período de aquecimento.}
    \label{fig:betasquatro0100}
 \end{figure}
 
 \newpage
 
 \section{Modelo: Feminismo e Antifeminismo}\label{resultadomodeloextra}
 Dados os resultados gerados na primeira fase do algoritmo de estimação de pontos ideais, podemos observar as estatísticas dos parâmetros estimados de acordo com as tabelas a seguir.
 
  \begin{figure}[!htbp]
    \centering
    \includegraphics[scale=0.6]{tabela_phi_ponto_ideal_politico_incompleta_fase1_c.png}
    \caption{Amostra maior das estatísticas geradas para as estimações feitas para as distribuições \textit{a posteriori} dos parâmetros $\phi$.}
    \label{convphianexo}
 \end{figure}
 
 \begin{figure}[!htbp]
    \centering
    \includegraphics[scale=0.6]{tabela_alpha_popularidade_politico_incompleta_fase1_nc.png}
    \caption{Amostra maior das estatísticas geradas para as estimações feitas para as distribuições \textit{a posteriori} dos parâmetros $\alpha$.}
    \label{convalphaanexo}
 \end{figure}

 \begin{figure}[!htbp]
    \centering
    \includegraphics[scale=0.6]{tabela_theta_ponto_ideal_cidadao_incompleta_fase1_c.png}
    \caption{Amostra maior das estatísticas geradas para as estimações feitas para as distribuições \textit{a posteriori} dos parâmetros $\theta$.}
    \label{convthetaanexo}
 \end{figure}

 \begin{figure}[!htbp]
    \centering
    \includegraphics[scale=0.6]{tabela_beta_interesse_cidadao_incompleta_fase1_c.png}
    \caption{Amostra maior das estatísticas geradas para as estimações feitas para as distribuições \textit{a posteriori} dos parâmetros $\beta$.}
    \label{convbetaanexo}
 \end{figure}
 
 Os resultados completos das estimações dos pontos ideais dos influenciadores podem ser encontrados nos gráficos a seguir.
 
 \begin{sidewaysfigure}[ht]
    \includegraphics[scale=0.35]{pontosideais_comp.png}
    \caption{Gráfico dos pontos ideais dos influenciadores.}
    \label{pontosideaistodos}
 \end{sidewaysfigure}
 
 \begin{sidewaysfigure}[ht]
    \includegraphics[scale=0.34]{pontosideais_feminista.png}
    \caption{Gráfico dos pontos ideais das influenciadoras feministas.}
    \label{pontosideaisfem}
 \end{sidewaysfigure}
 
 \begin{sidewaysfigure}[ht]
    \includegraphics[scale=0.35]{pontosideais_antifeminista.png}
    \caption{Gráfico dos pontos ideais dos influenciadores antifeministas.}
    \label{pontosideaisantifem}
 \end{sidewaysfigure}

 \begin{figure}[!htbp]
    \centering
    \includegraphics[scale=0.75]{legenda_pt1.png}
    \caption{Legenda do gráfico de estimação dos pontos ideais dos influenciadores.\\ Parte I.}
    \label{leg1}
 \end{figure}

 \begin{figure}[!htbp]
    \centering
    \includegraphics[scale=0.75]{legenda_pt2.png}
    \caption{Legenda do gráfico de estimação dos pontos ideais dos influenciadores.\\ Parte II.}
    \label{leg3}
 \end{figure}

 \begin{figure}[!htbp]
    \centering
    \includegraphics[scale=0.75]{legenda_pt3.png}
    \caption{Legenda do gráfico de estimação dos pontos ideais dos influenciadores.\\ Parte III.}
    \label{leg3}
 \end{figure}
 
% ---

%\lipsum[31]

% ---
%\chapter{Fusce facilisis lacinia dui}
% ---

%\lipsum[32]

\end{anexosenv}

 \bibliography{bibliografia}
 
%---------------------------------------------------------------------
% INDICE REMISSIVO
%---------------------------------------------------------------------
%\phantompart
\printindex
%---------------------------------------------------------------------

\end{document}
